\documentclass[palatino,code]{ensaexam}
\usepackage{amsfonts}
\usepackage[french]{babel}
\begin{document}
\ModuleName{Algèbre I}
\ExamCode{Cycle Préparatoire}
\ExamPeriod{Spring 2022}
\TimeAllowed{90}
\Logo{
\begin{center}
  \includegraphics[width=3cm, height=3cm]{ENSA-SAFI.png}
\end{center}
}
\Instructions{
  \begin{itemize}
    \item Vous avez {\bf\TheTimeAllowed\; minutes}. 
    \item Vérifier que vous disposez de toutes les pages. 
    \item L'échange d'outils est strictement \textbf{interdit}.
  \end{itemize}
  
}
\MakeHeading
 \pointsinrightmargin
 \boxedpoints
\vspace*{1cm}

%% For questions use the command 
%% \begin{questions} 
%% \questoin[grade]
%% \end{questions}



 \begin{questions}

   %{{{ Question 1
  \titledquestion{Ensembles}
  Soient $A$, $B$ et $C$ des sous-ensembles de $E$, montrer les affirmations suivantes:

  \begin{parts}
    \part[2] 
    $$ A \cup B \cup C = (A\backslash B)\cup (B \backslash C)\cup (C \backslash A)\cup (A\cup B \cup C)
    $$
    \part[3] Si $A\cup B \subset A \cup C$ et $A\cap B \subset A\cap C$ alors $B \subset C$.\\
    Que peut on dire sur l'implication inverse?
  \end{parts}
  \vspace*{0.5cm}
  %}}}

   %{{{ Question 2
  \titledquestion{Applications}
  Soit $E$ un ensemble et $\mathcal{P}(E)$  l'ensemble de ces parties. On considère l'application  $ f \; \mathcal{P}(E)\longrightarrow \mathbb{R}$ telle que pour toutes parties disjoints de $E$, on ait
  $$
  f(A \cup B ) = f(A) + f(B)
  $$
  \begin{parts}
    \part[2] Démontrer que $f(\emptyset) = 0$
    \part[3] Monter que pour toutes parties $A$ et $B$ de $E$ on  a:
    $$
    f(A\cup B ) - f(A\cap B) = f(A)  + f(B)
    $$
  \end{parts}
  \vspace*{0.5cm}
  %}}}

   %{{{ Question 3
  \titledquestion{Relations}
  Soit $\mathcal{R}$ la relation définie sur $\mathbb{R}$ par:
  $$
  \forall x, y \in \mathbb{R}, \quad x \mathcal{R} y\; \iff x^4 - x^2 = y^4 - y^2
  $$
  \begin{parts}
    \part[2] Démontrer qu'il s'agit d'une relation d'équivalence.
    \part[2] Déterminer la classe d'équivalence de $0$ puis en déduire celle de $1$.
    \part[2] Calculer la classe d'équivalence d'un élément quelconque $a\in \mathbb{R} $.
  \end{parts}
  \vspace*{0.5cm}
  %}}}

   %{{{ Question 4
  \titledquestion{Groupes}
  \begin{parts}
    \part[2] On munit $\mathbb{R}$ de la loi de composition interne $*$ définie par:
    $$
    \forall x, y \in \mathbb{R}\quad x*y = xy + (x^2 -1)(y^2 -1)
    $$
    Montrer que $*$ est commutative, non associative, et que $1$ est un élément neutre.
    \part[2] On munit $\mathbb{R}^{*+}$ de la loi de composition interne $*$ définie par:
    $$
    \forall x, y \in \mathbb{R}^{*+}\quad x*y = \sqrt{x^2 + y^2}
    $$
    Montrer que $*$ est commutative, associative, que $0$ est un élément neutre. Montrer qu'aucun élément de $\mathbb{R}^{*+}$ n'admet de symétrique pour $*$.
\end{parts}
  %}}}

 \end{questions}
 
 
\end{document}


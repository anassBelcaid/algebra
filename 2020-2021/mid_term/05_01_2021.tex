%{{{ Header
\documentclass[exam]{cs188}
\usepackage{verbatim}
\usepackage{fancyhdr}
\usepackage{polynom}
\usepackage{booktabs}
\usepackage{setspace}
\usepackage{amsmath,mathrsfs}
\usepackage{multicol}
\usepackage{amssymb}
\usepackage{algpseudocode}
\usepackage{graphicx}
\usepackage{caption}
% \usepackage{subcaption}
\usepackage{array}
\usepackage{xcolor}
\usepackage{float}
\usepackage{enumitem}
\usepackage{mathcomp}
\usepackage{tabularx}
\usepackage{wasysym}
\usepackage{pbox}
\usepackage{pgfplots}
% \usepackage{algorithmic,algorithm}
\usepackage{tikz}
\usetikzlibrary{matrix,shapes}
\usepackage[normalem]{ulem}
\usepackage{multirow}
% \usepackage[inline]{enumitem}
%}}}

% Linkes {{{ %
%http://vekemans.free.fr/L1_algebre/TD1.pdf
% }}} Linkes %
\title{Contrôle continue}
\begin{document}
\newpage
% Ensemble - Applications - Relations {{{ %
% Proof by double inclusion {{{ %
\begin{problem}[2]{* Egalité Ensemble}
    \label{q1}
    Soit $E$ un ensemble. Soient $A$, $B$ et $C$ trois parties de $E$. Montrer
    (en utilisant une double inclusion) que:

    \begin{equation}
      \label{eq:delta_distribution}
      \left(A\Delta B\right)\cap C = \left(A\cap C \right) \Delta \left( B\cap
      C\right).
    \end{equation}
  \end{problem}
% }}} Proof by double inclusion %
% Same proof using  {{{ %
  \begin{problem}[2]{* Fonction caractéristique}
    On cherche maintenant à prouver l'égalité de l'équation \eqref{eq:delta_distribution}
    en utilisant les fonctions caractéristiques:
    $$\pi_A(x) =\left\{\begin{array}{ll} 1 & x\in A\\ 0 & x\notin A
    \end{array}\right.$$ On accepte que si $\pi_A = \pi_B$, alors $A = B$.
    \begin{enumerate}
      \item Calculer, en fonction de $\pi_A, \pi_B$ et $\pi_C$, la fonction
        caractéristique de $(A\Delta B)\cap C$.
      \item Même question pour l'ensemble $(A\cap C)\Delta (B \cap C)$.
      \item Endéduire le résultat de la question (Q1).
    \end{enumerate}
\end{problem}
% }}} Same proof using  %
% Injection Surjection {{{ %
\begin{problem}[2]{** Injection / Surjection}
 Soit $E$ un ensemble et $f: E\longrightarrow E$ une application tel que 
 \begin{equation}
   \label{eq:equality}
    f\circ f\circ f = f
 \end{equation}
 \begin{itemize}
   \item Montrer que si $f$ est injective, alors aussi $f$ est
     \textbf{surjective}.
   \item Inversement, si $f$ est surjective, alors $f$ est
     \textbf{injective}.
 \end{itemize} 
\end{problem}
% }}} Injection Surjection %
% Relation  {{{ %
\begin{problem}[2]{* Relation d'équivalence}
  On définit sur $\mathbb{R}^2$ la relation $\mathcal{R}$ par:
  \begin{equation}
    \label{eq:relation1}
    (x_1,y_1)\;\mathcal{R}\;(x_2, y_2) \iff x_1-5y_2 = x_2 - 5y_1
  \end{equation}
  \begin{enumerate}
    \item Montrer que $\mathcal{R}$ est une relation
    d'\textbf{équivalence}.
  \item Pour un point $(a,b)$ dans le plan, Déterminer l'ensemble
    $\text{Cl}\left(\left(a,b\right)\right)$.
  \end{enumerate}
\end{problem}
% }}} Relation  %
\newpage
% relation d'ordre total {{{ %
\begin{problem}[2]{* Relation d'ordre}
  Pour deux points $(x_1, y_1)$, $(x_2, y_2)$ dans le plan $\mathbb{R}^2$, on
  définit la relation $\mathcal{R}$ par:

  \begin{equation}
    \label{eq:order}
    \forall (x_1,y_1)\;\;\mathcal{R}\;\;(x_2,y_2)\;\;\iff\;\;x_1\leq
    x_2\;\;\text{et}\;\; y_1\leq y_2
  \end{equation}
  \begin{enumerate}
    \item Démontrer que $\mathcal{R}$ est une relation d'ordre.
    \item Pour un point $(a,b)\in \mathbb{R}^2$, représenter graphiquement l'ensemble des
      éléments supérieurs à $(a,b)$ et ceux  qui sont inférieurs.
    \item Cette relation est elle totale?
  \end{enumerate}
\end{problem}
% }}} relation d'ordre total %
% }}} Ensemble Application Relation %
% Questions groupes {{{ %
% Vérification groupes {{{ %
\begin{problem}[2]{** Vérification groupe}
  Soit $G = \mathbb{R}\times\mathbb{R}^{*}$ et $*$ la loi définie comme suit:

  \begin{equation}
    \label{eq:loi_interne}
    (x_1, y_1)*(x_2,y_2)= \left( x_2 y_1 + \dfrac{x_1}{y_2},\;y_1y_2\right)
  \end{equation}

  \begin{enumerate}
    \item Vérifier que $*$ est un loi interne dans $G$.
    \item La loi $*$ est-elle associative? Elle est commutative?
    \item A-t-on un élément neutre dans $\left(G,*\right)$?
    \item $\left(G, *\right)$ est-il un groupe?
  \end{enumerate}
\end{problem}
% }}}  Vérification groupes %
% Automorphisme antérieur {{{ %
\begin{problem}[4]{*** Automorphisme antérieur}
  Soit $(G, .) $ un groupe. Pour chaque élément $a \in G$ on note la fonction
  $\tau_a$ définie comme suit:

  \begin{equation}
    \label{eq:tau}
    \tau_a: \begin{array}{lll}
      G&\longrightarrow  & G\\
      x&\longrightarrow & a^{-1}.x.a
    \end{array}
  \end{equation}

  \begin{enumerate}
    \item Démontrer que $\forall a \in G, \quad \tau_a$ est un morphisme de
      groupe.
    \item Vérifier que $\forall a, b \in G,\quad \tau_a\circ\tau_b=\tau_{a.b}$.

    \item Montrer que $\forall a\in G \quad \tau_a$ est bijective. Déterminer sa
      fonction réciproque.
    \item En déduire que l'ensemble de ces morphismes $\Theta=\left\{\tau_a\;,\;
      a\in G\right\}$ muni de la \textbf{composition}  est un groupe.
  \end{enumerate}

\end{problem}
% }}} Automorphisme antérieur %
% }}} Questions groupes %
% Polynomes {{{ %
% PGCD {{{ %
\begin{problem}[]{** PGCD}
  \begin{question}[2]
  Soit $2x^4+2x^3-2x-2$ et $Q=-x^4+1$ deux polynômes dans $\mathbb{R}[X]$.

  \begin{enumerate}
    \item Soit $D= \text{PGCD}(P,Q)$,  déterminer $D$ en spécifiant les étapes de calcul.
  \end{enumerate}
\end{question}
  % \polylonggcd{2x^4+2x^3-2x-2}{-x^4+1} 

\begin{question}[2]
  On cherche deux polynômes $U$ et $V$ tel que 
  \begin{equation}
    U(2x^4+2x^3-2x-2) + V(-x^4+1) = \underbrace{x^3-x}_{C} 
  \end{equation}
  \begin{enumerate}
    \item Démontrer que $C$ est un \textbf{multiple} de $D$.

    \item Calculer $U$ et $V$.
  \end{enumerate}
 \end{question} 
\end{problem}
% }}} PGCD %
% }}} Polynomes %
\end{document}

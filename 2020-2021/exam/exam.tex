%{{{ Header
\documentclass[exam]{cs188}
\usepackage{verbatim}
\usepackage{fancyhdr}
\usepackage{mynotations}
\usepackage{booktabs}
\usepackage{setspace}
\usepackage{amsmath,mathrsfs}
\usepackage{multicol}
\usepackage{amssymb}
\usepackage{algpseudocode}
\usepackage{graphicx}
\usepackage{caption}
% \usepackage{subcaption}
\usepackage{array}
\usepackage{xcolor}
\usepackage{float}
\usepackage{enumitem}
\usepackage{mathcomp}
\usepackage{tabularx}
\usepackage{wasysym}
\usepackage{pbox}
\usepackage{pgfplots}
% \usepackage{algorithmic,algorithm}
\usepackage{tikz}
\usetikzlibrary{matrix,shapes}
\usepackage[normalem]{ulem}
\usepackage{multirow}
% \usepackage[inline]{enumitem}
%}}}

\title{Examen}
\begin{document}
\newpage

% {{{ Questions de cours %
\begin{problem}[4]{(*) Questions de cours}
    \begin{enumerate}
        \item  Soit $E$ un $\Rr$-espace vectoriel et soit $\mathcal{A}=\left\{v_1,v_2,\ldots,v_p\right\}$ une famille
            de $E$.
            \begin{itemize}
                \item Donner la définition que la famille $\mathcal{A}$ est
                    \textbf{génératrice}.
                \item On suppose que $\mathcal{A}$ est génératrice, donner la
                    relation entre $\Card{\mathcal{A}}=k$ et $\dim E=n$.
            \end{itemize}
        \item Soit $F$ un autre $\Rr$-espace vectoriel et $f:E\longrightarrow F$
            une application.

            \begin{itemize}
                \item Donner la définition de $f$ est une \textbf{application
                    linéaire}. 
                \item On suppose que $f$ est linéaire, Définir $\ker f$.
                \item Donner une condition sur $\ker f$ pour que $f$ soit injective.
            \end{itemize}
        \item Soient $E$ et $F$ deux ensembles tel que $\Card E=3$ et $\Card
            F=5$. 
            \begin{itemize}
                \item Quel est le nombre d'\textbf{injections} de $E$ vers $F$?
                \item Quel sera le cardinal des sous ensembles de $F$ contenant
                    juste \textbf{deux} éléments.
            \end{itemize}
    \end{enumerate}
\end{problem}
% }}} 
% Groupe {{{ %
\begin{problem}[6]{(*) Groupes}
   \begin{enumerate}
       \item Sur $E = \Rr\backslash\{1\}$, on définit sur $E$ la loi interne $*$
           par:
           \begin{equation}
           \forall x,y\in E^2\;\quad     x\;*\;y  = x\;+\;y\;-\;xy
           \end{equation}
        \begin{itemize}
            \item Montrer que $\left(E\;,\;*\right)$ est un groupe.
        \end{itemize}
    \item Montrer que la loi interne $*$ définie sur $\Rr$ par 
        $$
        \forall a,b\in \Rr^2\;\quad\;a\;*\; b = \ln\left(e^a + e^b\right)
        $$
        n'admet pas un élément neutre.
    \item Soit $\left(G,\;.\right)$ un groupe dont l'élément neutre est noté
        $e$. On suppose qu'on possède la propriété suivante:

        \begin{equation*}
           \forall x\in G\;\quad\; x^2 = x\;.\;x = e 
        \end{equation*}
        Montrer que $(G,\;.)$ est abélien.
    \item On considère $\left(G,\;.\right)$ un groupe abélien. Montrer que
        l'application:
        $$
        \begin{array}{llll}
            \Phi: &G&\longrightarrow &G\\
                  &x &\longrightarrow& x^{-1}
    \end{array}
        $$


        est un morphisme de groupe ( $x^{-1}$ est le symétrique de $x$).
   \end{enumerate} 
   
\end{problem}
% }}} Groupe %
\newpage
% Polynomes {{{ %
\begin{problem}[6]{(**) Polynômes}
On considère le polynôme $P=x^3-5x^2 + 8x-4$.

\begin{enumerate}
    \item Montrer que $\alpha=2$ est une racine de $P$.
    \item Quel est le degré de multiplicité de $\alpha$.
    \item Donner une décomposition en $\Rr[X]$ de $P$.  
    \item On considère la fraction rationnelle $$F=\dfrac{x+1}{P}.$$ Donner la
        décomposition \textbf{en éléments simples}  de $F$.
\end{enumerate}
\end{problem}
% }}} Polynomes %
% Espaces vectoriels {{{ %
\begin{problem}[6]{(**) Espaces vectoriels}
  \begin{enumerate}
      \item On considère l'ensemble:
          \begin{equation}
              E = \left\{ (x,y,z,t)\in\Rr^4\;|\; x+y=z=t=0\right\}
          \end{equation}
          \begin{itemize}
              \item Prouvez que $E$ est un sous espace vectoriel de $\Rr^4$.
              \item Trouver un vecteur $u_1\in \Rr^4$ tel que $E =
                  \text{Vect}(u_1)$.
              \item Donner une base et la dimension de $E$.
          \end{itemize}
      \item On considère l'espace vectoriel $F\subset \Rr^4$ défini par:
          \begin{equation}
              F = \left\{(a,\; a+b,\;-a+c,\;c)\;|\;(a,b,c)\in \Rr^3\right\}
          \end{equation}
          \begin{itemize}
              \item Trouver trois vecteurs $u_2, u_3$ et $u_4$ tel que
                  $F=\text{Vect}(u_2,u_3,u_4)$.
              \item Donner une base et la dimension de $F$.
          \end{itemize}
      \item Démontrer que $E$ et $F$ sont \textbf{supplémentaires} dans $\Rr^4$.
  \end{enumerate}  
\end{problem}
% }}} Espaces vectoriels %

\end{document}

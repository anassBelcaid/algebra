\documentclass{report}
\usepackage{pgfplots}
\usepackage{booktabs}

\input{preamble}
\input{macros}
\input{letterfonts}


\title{\Huge{Travaux dirigées}\\ Groupes}
\author{\huge{A.Belcaid}}
\date{\today}

\begin{document}

\maketitle
% \newpage% or \cleardoublepage
% \pdfbookmark[<level>]{<title>}{<dest>}
\pdfbookmark[section]{\contentsname}{toc}
\tableofcontents
\pagebreak
\newcommand{\Nn}{\mathbb{N}}
\newcommand{\Rr}{\mathbb{R}};
\newcommand{\Zz}{\mathbb{Z}}
\newcommand{\Qq}{\mathbb{Q}}




\chapter{}

% Exercice 1 {{{ %
\section{Exercice 1}
\qs{}
{
Pour chaque cas, vérifier si l'ensemble avec la loi proposée est un
\textbf{groupe}:

\begin{enumerate}
  \item $G$ est l'ensemble des applications de $\Rr\longrightarrow\Rr$ définie
    par $f(x) = ax + b$ où $a\in \Rr^{*}$ et $b\in \Rr$, muni de la composition.

  \item $G$ est l'ensemble des fonctions croissantes muni de l'addition.
  \item $G = \{f_1, f_2, f_3, f_4\}$ muni de la composition.
    où:\\
    \begin{itemize}
      \item $f_1 = x$\hspace*{8pt}
      \item $f_2 = -x$\hspace*{8pt}
      \item $f_3 = \dfrac{1}{x}$\hspace*{8pt}
      \item $f_4 = -\dfrac{1}{x}$
    \end{itemize}
\end{enumerate}
}
\begin{myproof}
  Pour la première loi:
  \begin{enumerate}
    \item fonctions linéaires muni de la composition.
      \begin{enumerate}
        \item \textbf{Loi interne}: Pour deux fonctions $f_1:x \longrightarrow
          a_1x + b_1$ et $f_2: x\longrightarrow a_2x+ b_2$
          On as
          $$
          f_1\circ f_2(x) = f_1(f_2(x)) = f_1(a_2x + b_2) = a_1a_2 x + (a_1b_2 + b_1)
          $$
          qui est une fonction linéaire.

        \item \textbf{Associativité}
          Provient directement de l'associativité de la composition.

        \item \textbf{Élément neutre}: Pour la fonction $id_{\Rr}$, on sait déjà
          que  pour toute fonction on as :
          $$
          f \circ id_\Rr = id_\Rr \circ f = f 
          $$
          Comme $id_\Rr = 1x + 0$, c'est une application linéaire.
        \item \textbf{Symétrique} Soit $a_1 \in \Rr^*$  et $b_1\in \Rr$, on calcule l'inverse
          gauche de $f = a_1 x + b_1$
          \begin{eqnarray*}
            f\circ(g(x)) &=& x\\
            a_1(a_2 x + b_2) + b_1 &=& x\\
            a_1a_2 x + (a_1b_2 + b_1)&=& x
        \end{eqnarray*}
        Ainsi on obtient
        $$
        \begin{cases}
          a_1a_2 &= 1 \\
          a1b_2 + b_1 &= 0
        \end{cases}
        $$

        $$
        \begin{cases}
          a_2&= \dfrac{1}{a_1} \\[8pt]
          a_1b_2 + b_1 &= \dfrac{-b_1}{a_1}
        \end{cases}
        $$
        Ainsi pour $g = \dfrac{1}{a_1} x - \dfrac{b_1}{a_1}$ on as $f\circ g =
      \id$. On calcule la composition:
      $$
      g\circ f = \dfrac{1}{a_1}(a_1x + b_1) - \dfrac{-b_1}{a_1} = \text{id}
      $$
      Ainsi chaque élément admet un symétrique.
      \end{enumerate}
      \nt{On pouvait simplement démontrer que l'ensemble des linéaires est un
      sous groupe des fonctions bijectives muni de la composition.}
    \item 
      Pour la composition l'élément neutre est $\text{id}$, cependant pour une
      fonction $f\;x\longrightarrow \text{Cst}$. La fonction est croissante mais elle admet
      pas un élément symétrique (fonction inverse). Car on ne peux trouver $g$ une
      fonction tel que 
      f(g(x)) = \text{id}.
    \item 
      \begin{enumerate}
        \item \textbf{Loi interne}
          Voici un tableau des compositions:
          \begin{table}[h]
         $$
         \begin{array}{ccccc}
            
           \toprule
           \circ & f_1 & f_2 & f_3 & f_4\\
           \bottomrule
           f_1 & f_1 & f_2 & f_3 & f_4  \\
           f_2 & f_2 &  f_1  &  f_4    & f_3    \\ 
           f_3 & f_3 &  f_4 & f_1 & f_2 \\
           f_4 & f_4 &  f_3 & f_2 & f_1\\
           \bottomrule
         \end{array}
         $$
         \caption{Tableau des composition ou la première ligne correspond au
         choix de la fonction $g$, la première colonne montre le choix de la
       fonction $f$, et le reste contient le résultat de $f\circ g$. }
       \end{table}
       Ce qui prouve que la loi est interne.
     \item \textbf{Association} Provient directement de l'association de la
       composition.
     \item \textbf{Élément neutre} $f_1 = \text{id}$ est dans le groupe.

     \item \textbf{Élément symétrique}: 
       Selon la table on as 
       $$
       \forall i\in [1,4]\quad f_i \circ f_i = \text{id}
       $$
       Ce qui prouve que chaque élément et son propre symétrique.
      \end{enumerate}
  \end{enumerate}
  
\end{myproof}

% }}} Exercice 1 %
% Exercice 2 {{{ %
\section{Exercice 2}
\qs{}
{
Pour les deux cas suivants, démontrer que $G$ est un groupe puis vérifier s'il
est \textbf{abélien}.
\begin{enumerate}
  \item $x * y = \dfrac{x+y}{1 + xy}$ sur $G= ]-1,1[$.\\[2pt]
  \item $(x_1,y_1)*(x_2,y_2) = (x_1+x_2\;,\; y_1e^{x_2} + y_2e^{x_1})$ sur
    $\Rr^2$.\\[8pt] 
\end{enumerate}
}

\begin{myproof}
  Pour le premier cas:\\
  \begin{enumerate}
    \item $(G, *)$ est un groupe car:
      \begin{enumerate}
        \item \textbf{loi interne}: en effet, si $x, y\in G$, alors $x \star y \in G$.\\
          Pour cela, etudions la fonction definie sur $[-1,1]$ par:
          $$
          f(t) = \dfrac{t + y}{1 + ty}
          $$
          Elle est derivable sur $[-1,1]$, et sa derivee verifie:
          $$
          f^{'}(t) = \dfrac{1- y^2}{(1 + ty)^2} > 0 \text{ sur } ]-1,1[.
          $$
          f est donc strictement croissante sur $]-1,1[$ et on a 
          $$
          f(-1) = x \star y = f(x) < f(1)
          $$
          Comme $f(-1) = -1$ et $f(1) = 1$, on obtient que 
          $$
          x \star y \in G.
          $$
        \item \textbf{Associative}: Pour tout $(x,y,z)\in G^3$.
          \begin{eqnarray}
            x\star\left(y \star z\right) &=& \dfrac{x + (x\star z)}{ 1 + x(y\star z)}\\
                                         &=& \dfrac{x + \dfrac{x+z}{1 + yz}}{1 + x \dfrac{y + z}{1 + yz}}\\
                                         &=& \dfrac{x + y + z + xyz}{ 1 + xy + xz + yz}
          \end{eqnarray}
          On calcule l'expression $\left(x \star y\right)\star z$ et on trouve le meme resulat.
        \item \textbf{Element neutre}: On montre que $0$ est un element neutre:

          $$
          z \star 0  = 0 \star x = \dfrac{x + 0}{ 1 + 0} = x
          $$
          \item \textbf{inverse} 
            $$
            \forall x\in [-1,1] \quad -x \in [-1,1] \quad \text{ et } x \star (-x) = (-x)\star x = \dfrac{x - x}{ 1 - x^2} = 0
            $$
            De plus, le groupe est \textbf{abelien}.
      \end{enumerate}
    \item Il est clair que $\star$ est une loi de composition interne sur $\mathbb{R}^2$. De plus, 

      \begin{enumerate}
        \item Cette loi est associative:
          \begin{eqnarray}
            (x_1,y_1)\star \big(\left(x_2,y_2\right)\star \left(x_3, y_3\right)\big) &=& (x_1, y_1)\star \left(x_2 + x_3, y_2 e^{x_3} + y_3 e^{-x_2}\right)\\
                                                                                     &=& (x_1, + x_2 + x_3, y_1e^{x_2+ x_3}+ y_2e^{x_3-x_1} + y_3e^{-x_1-x_2})
          \end{eqnarray}
          De meme  on trouve que
          $$
          \big((x_1,y_1)\star \left(x_2,y_2\right)\big) \star \left(x_3, y_3\right) = (x_1, + x_2 + x_3, y_1e^{x_2+ x_3}+ y_2e^{x_3-x_1} + y_3e^{-x_1-x_2})
          $$
        \item \textbf{Element neutre}, On a:
          $$
          (x,y) \star (0,0) = (x + 0, ye^0 + 0 e^{-x}) = (x,y)
          $$
          et 
          $$
          (0,0)\star (x,y) = (0 + x, 0e^x + ye^{-0}) = (x,y)
          $$
        \item \textbf{Element inverse} 
          $$
          (x,y) \star (-x, -y) = (x -x, ye^{-x}- ye^{-x}) = (0,0)
          $$
          $$
          (-x,-y)\star (x,y) = (-x+x, -ye^{x} + ye^{x}) = (0,0)
          $$

        \nt{Le groupe n'est pas abelien car
        $$
        (1,0)\star (0,1) = (1,e^{-1}) 
      $$
    tandis que 
  $$
  (0,1) \star (1,0) = (1, e^1)
$$}
      \end{enumerate}
  \end{enumerate}
\end{myproof}

% }}} Exercice 2 %
% Exercice 3 {{{ %
\section{Exercice 3}
\qs{}
{
Soit $G$ un groupe \textbf{fini} d'élément neutre $e$.\\
\begin{enumerate}
  \item 
Montrer que si cardinal de $G$ est pair, alors il existe $x\in G$ tel que:
$$x\neq
e\;\;\text{et}\;\; x^{-1} = x$$
\end{enumerate}

}

\begin{myproof}
  on definie la relation $\mathcal{R}$ definie sur $G$ par

  $$
  x \mathcal{R} y \iff x = y \quad \text {ou } x = y^{-1}
  $$
  On demontre facilement qu'il s'agit dune relation \textbf{d'equivalence}.
  On suppose que 
  $$
  \not \exists x \ne e \quad \text{ tel que } x = x^{-1}.
  $$
  Ainsi les classes d'equivalences pour la relation $\mathcal{R}$ sont

  $$
  \begin{cases}
    Cl(e) &= \{e\}\\
    Cl(x) &= \{x, x^{-1}\} \quad  \text{ si } x\ne e
  \end{cases}
  $$

  Selon le theoreme de partition  on 

  $$
  G = \bigcup_{x\in G} Cl(x_i) 
  $$
  On decompose les classes:

  $$
  G = \bigcup_{x\ne e} Cl(x_i) \cup \{e\}
  $$
  Puisque les classes sont \textbf{disjoins} on obtient alors que

  $$
  \text{Card}(G) = \sum_{x\ne e} \text{Card}Cl(x) + 1
  $$

  $$
  \text{Card}(G) = 2k + 1
  $$
  ou $k$ est le nombre de classe.\footnote{
    Le 2 vient du fact, que chaque classe contient seulement \textbf{deux} elements.
  }
  Ce qui est \textbf{absurde} car le cardinal de $G$ est pair.
\end{myproof}

% }}} Fold description %
% Exercice 4 {{{ %
\newpage
\section{Exercice 4}
\qs{}
{
Soit $G$ un ensemble \textbf{fini} muni d'une loi de composition interne $*$
associative. On dit qu'un élément $a$ est \textbf{régulier} si les deux
conditions suivantes sont vérifiées:
\begin{itemize}
  \item $a * x = a * y \implies x = y$
  \item $x *a  = y * a \implies x = y$
\end{itemize}
On suppose que tous les éléments de $G$ sont réguliers, et on fixe $a\in G$.

\begin{enumerate}
  \item Démontrer qu'il existe $e\in G$ tel que $a*e = a$.\\[2pt]
  \item Démontrer que, pour tout $x\in G$, on a $e*x = x$.\\[2pt]
  \item Démontrer que, pour tout $x\in G$, on a $x*e = x$.\\[2pt]
  \item Démontrer que $\left(G, *\right)$ est un groupe.
  \item Le résultat subsiste-t-il si $G$ n'est fini?
\end{enumerate}
}
\begin{myproof}
  \begin{enumerate}
    \item On considere l'application
      $$
      \begin{array}{lllll}
        \phi & : & G & \longrightarrow & G\\
             &   & x & \longrightarrow & a \star x
      \end{array}
      $$
      Puisque $a$ est regulier ca prouve que cette aplicatin est \textbf{injective}. Puisque $G$ est fini et $\text{Card} G = \text{Card} G$. alors l'applicatin $phi$ est une \textbf{bijection}.
      Ainsi $a$ admet un antecedant par $\phi$.
      $$
      \exists e \in G \quad a \star e = \phi(e) = a
      $$
    \item En utilisant l' associative de la loi interne:
      $$
      a \star e \star x = a \star x 
      $$
      et Puisque $a$ est regulier on obtient:
      $$
      e \star x = x
      $$
    \item On a:
      $$
      a \star e \star a = x \star a
      $$
      On utilise la question precedante et la regularite de $a$ on aura
      $$
      x \star e = x
      $$
    \item Il suffit desormais de montrer que tout element est inversible. Soit $b\in G$, Puisque l'application $x \rightarrow b \star x$  est bijective \footnote{en utilisant la meme procedure que la premiere question pour $a$} . alors $e$ admet un antecedant unique:
      $$
      \exists c \in G \quad b \star c = e
      $$
      
      De plus on as:
      $$
      c \star b \star c = c \star e = c
      $$
      On deduit alors que $c$ est l'inverse de $b$.
    \item Non, toute l'analyse utilise le fait que l'applicatin $\phi$ est bijective. Chose qui est correcte car le cardinal de $G$ est fini. Sinon elle est seulement \textbf{injective}. Comme contre exemple on prend, $(\mathbb{N}, +)$ qui verifie que tout elements est regulier. Mais c'est pas un groupe.
  \end{enumerate}

\end{myproof}
% }}} Exercice 4 %

% {{{ Exercice 5
\section{Exercice 5} % (fold)
\qs{}
{

% Détermination simple {{{ %
Pour chaque cas,  déterminer si la partie $H$ est un sous groupe de $G$.

\begin{enumerate}
  \item $G = \left(\Zz,+\right)$ et $H =\{2k\;|\; k\in \Zz\}$.
  \item $G = \left(\Zz,+\right)$ et $H =\{2k+1\;|\; k\in \Zz\}$.
  \item $G = \left(\Rr^{*},+\right)$ et $H =]-1,\infty[$.
  \item $G = \left(\Rr^{*},\times\right)$ et $H =\Qq^*$.
  \item $G = \left(\Rr^{*},\times\right)$ et $H \{a + b\sqrt{2}\;|\; a, b\in
    \Qq, (a,b)\neq (0,0)\}$.
\end{enumerate}
}
\begin{myproof}
  \begin{enumerate}
    \item $H$ est un sous-groupe de $G$. En effet:
      \begin{itemize}
        \item $0 \in H$.
        \item si $x,y\in H$ alors $-x$ et $x+y$ sont aussi dans $H$.
      \end{itemize}
      Ainsi $H$ est un \textbf{sous groupe} de $G$.
    \item $0 \not\in H$ et donc $H$ n'est pas un sous groupe.
    \item $2 \in H$ et $-2\not\in H$: $H$ n'est pas un sous-groupe de $G$. 
    \item $H$ est un sous groupe car:
      \begin{enumerate}
        \item $1 \in H$.
        \item si $x = \dfrac{p_1}{q_1}$ et $y=\dfrac{p_2}{q_2}$ alors $x \times y = \dfrac{p_1p_2}{q_1q_2}\in H$.
        \item Aussi on a $\dfrac{1}{x} = \dfrac{q_1}{p_1} \in H$.
      \end{enumerate}
      Ainsi $H$ est un sous groupe.
    \item Il s'agit bien d'un sous groupe car:
      \begin{enumerate}
        \item $1 = 1 + 0\sqrt{2} \in H$
        \item On prend $x = a + b\sqrt{2}$ et $y= c + d\sqrt{2}$ deux elements de $H$.
          alors
          $$xy = (ac + 2bd ) + (ad + bd)\sqrt{2}$$
        \item Aussi on a:
          $$
          \dfrac{1}{x} = \dfrac{a}{a^2 - 2b^2} + \dfrac{-b}{a^2 - 2b^2}\sqrt{2}
          $$
          \nt{On remarque que l'on peut pas voir $ac + 2bd= 0$ et $ad + bd= 0$ sinon on aurait $xy= 0$}
          Ainsi $H$ est un sous groupe de $G$.
      \end{enumerate}

  \end{enumerate}
\end{myproof}

% section section name (end)
% }}}

%{{{ Exercice 6
\section{Execice 6}
\qs{Centre et element de torsion}
{

Soit $\left(G,.\right)$ un groupe. Démontrer que les parties suivantes sont des
sous-groupes de $G$

\begin{enumerate}
  \item $C(G) = \{x\in G\;|\; \forall y \in G\;,\; x.y = y.x\} $, $C(G)$
    s'appelle le \textbf{centre}  de $G$.
  \item $aHa^{-1} = \left\{a h a^{-1}\;, h\in H\right\}$ où $a\in G$ et $H$ est
    un sous groupe.
  \item On suppose que $G$ est abélien. On dit que $x$ est un élément  de
    \textbf{torsion} de $G$ s'il existe $n\in \Nn$ tel que $x^n=e$.

    \begin{itemize}
      \item Démontrer que l'ensemble de torsion forme un sous groupe.
    \end{itemize}
\end{enumerate}
}
\begin{myproof}
  Il suffit, pour chaque cas, d'appliquer le theoreme de caracterisation des sous-groupes. 
  \begin{enumerate}
    \item Pour les centres on a:
      \begin{enumerate}
        \item $e \in C(G)$ car 
          $$
          \forall y \in G \quad ey = ye =y
          $$
        \item Soient $x_1$ et $x_2 \in C(G)$, alors pour tout $y \in G$ on a:
          $$
          x_1x_2y = x_1(x_2y) = (x_1y)x_2 = yx_1x_2
          $$
        \item Soit $x\in C(G)$ alors on a:
          $$
          xy = yx \implies xyx^{-1} = yxx^{-1} = y \implies x^{-1}xyx^{-1} = x^{-1}y  \implies yx^{-1} = x^{-1} =y 
          $$

          Ainsi on deduit que $C(G)$ est sous groupe de $G$.
      \end{enumerate}
      \begin{enumerate}
        \item  Puisque $H$ est un sous groupe, on a $e\in H$ ainsi 
          $$
          aea^{-1} = e \in aHa^{-1}
          $$
        \item Soient $x = ah_1a^{-1}$ et $y = a h_2 a^{-1}$ deux elements de $aHa^{-1}$ avec $h_1, h_2 \in H$.
          $$
          xy = ah_1a^{-1}ah_2 a^{-1} = a\underbrace{h_1h_2}_{\in H}a^{-1} \in aHa^{-1}
          $$
        \item Soit $x = aha^{-1}$ pour $y = ah^{-1}a^{-1}$ on a
          $$
          xy = aha^{-1}ah^{-1}a^{-1} = ahh^{-1}a^{-1} = aa^{-1} = e
          $$
        \nt{Puisque $H$ est un sous groupe, $h^{-1}$ existe et forcement dans $H$}
      \end{enumerate}
  \end{enumerate}
\end{myproof}
%}}}
%{{{ Execice 7
\section{Exercice 7} % (fold)
\qs{Union de sous groupes}
{
Soit $G$ un groupe et $H$ et $K$ deux sous groupes de $G$. Démontrer que $H\cup
K$ est un sous groupe de $G$ si et seulement si $H\subset K$ ou $K\subset H$.
}
\begin{myproof}
On demontre les deux implications:

\begin{enumerate}
  \item Si $H \subset K$ alors $H\cup K = K$ qui est un sous-groupe de $G$.\\
    De meme, si $K \subset H$, $H\cup K = H$ qui est un sous groupe de $G$.
  \item Supposons maintenant que $H\cup K$ est un sous groupe de $G$ et ni $H \subset K$ ni $K \subset H$.\\
    alors on peut trouver 
    $$
    \begin{cases}
      x \in &H\backslash K\\
      y \in &K\backslash H\\
    \end{cases}
    $$
    Maintenant on utilise la fait que $H\cup K$ est un sous groupe, comme $x \in H \cup K$ et $y \in H \cup K$. on aura forcement que 
    $$
    xy \in H \cup K
    $$
    Sans perte de generalite \footnote{Dans le cas inverse $xy \in K$, on change $H$ par $K$ en gardant la meme demonstration}
    on suppose que $xy \in H$. alors on aura que
    $$
    \underbrace{x^{-1}(xy)}_{ = y} \in H
    $$
    Ce qui constitue une contradtion car $y \not \in H$.

\end{enumerate}
\end{myproof}

% section section name (end)


%}}}

%{{{ Exercice 8
\section{Exercice 8}
\qs{}{
Montrer que $H = \left\{x+ \sqrt{3}y\;|\; x\in\Nn, y\in\Zz\;,\; x^2 -3y^2=1
\right\}$ est un sous groupe de $\left(\Rr_+^{*},\times)\right)$.
}
\begin{myproof}
  La premiere chose a remarquer est que $H \subset \Rr_+^{*}$. Car si $x + y\sqrt{3} \in H$ on as alors 
  $$
  x^2 - 3 y^2 > 0 
  $$
  puisque $x\in \Nn$ on a $x > \sqrt{3}\vert y\vert$ et donc $x + y \sqrt{3} > 0$.\\

  on a aussi que $1 = 1 + 0\sqrt{3} \in H$. On montre maintenant la multiplication est une loi interne dans $H$. Soient alors $a = x + y\sqrt{3}$ et $b = u + v\sqrt{3}$ deux elements dans $H$. Alors:

  $$
  ( x + y\sqrt{3})(u + v\sqrt{3}) = (xu + 3yv) + \sqrt{3}( xv + yu)
  $$

  On remarque ensuite que:

  \begin{eqnarray*}
    (xu + 3yv)^2 - 3(xv + yu)^2 &=& x^2u^2 + 9y^2v^2 - 3xx^2v^2 - 3y^2u^2\\
                                &=& x\left(u^2 -3v^2\right) + 3y^2(3v^2 - u^2)\\
                                &=& x^2 - 3y^2\\
                                &=& 1.
  \end{eqnarray*}

  Il nous reste a verifier que $xu + 3yv$  est un element de $\Nn$. Ceci provient du fait que $x \geq \sqrt{3}\vert y \vert $ et $u \geq \sqrt{3}\vert v \vert$.
  Ainsi 
  $$
  ab \in H
  $$

  On demontre maitenant que l'inverse de chaque element est dans $H$.

$$
\dfrac{1}{a} = \dfrac{1}{x + y\sqrt{3}} = \dfrac{x - y\sqrt{3}}{x^2 - 3 y^2} = x - y\sqrt{3} \in H
$$
Ainsi H est bien un sous groupe de $(\Rr_+^{*}, \times)$.
\end{myproof}
%}}}

%{{{
\section{Exercice 9}
\qs{Produit de deux sous groupes}
{
Soit $\left(G, .\right)$ un groupe fini et $A,B$ deux sous groupes de $G$. On
note
\begin{equation*}
  AB = \{a.b\;|\; a\in A, b\in B\}.
\end{equation*}

\begin{enumerate}
  \item Montrer  que $AB$ est un sous groupe de $G$ si et seulement si $AB =
    BA$.
\end{enumerate}
}
\begin{myproof}
  Supposons d'abord que $AB = BA$. Alors $AB$ est un sous groupe de $G$ car:

  \begin{enumerate}
    \item $e \in AB$ car $e= ee$ avec $e \in A$ et $e \in B$.
    \item $AB$ est stable par la loi interne. En effet soit $x=ab \in AB$ et $y = uv \in AB$ alors 
      $$
      xy = abuv
      $$
  On sait que $bu \in AB = BA$. ainsi on conclut qu'il existe $a_1 \in A$ et $b_1 \in B$ tel que
  $$
  bu = a_1 b_1
  $$
  on remplace $bu$ et on trouve que:
  $$
  xy = \underbrace{aa_1}_{\in A} \underbrace{b_1b}_{\in B}
  $$
  Ainsi $xy \in AB$.

\item $AB$ est stable par passage a l'inverse:
  $$
  x = ab \in AB \implies x^{-1} = b^{-1}a^{-1} = a^{'}b^{'}\in AB.
  $$
  \end{enumerate}
  Reciproquement, supposong que $AB$ est un sous groupe de $G$, et prouvons que $AB = BA$.\\

  Soit d'abord $x = ab \in AB$. Alors $x^{-1} = b^{-1}a^{-1} \in AB$.

  Puisque $AB = BA$. on alors l'existence de $a_1\in A$ et $b_1 \in B$ tel que 
  $$
  x^{-1} = a_1 b_1 \implies x = b^{-1} a_1{-1} \in BA
  $$

  On procede la meme facon pour l'autre inclusion.

\end{myproof}
%}}}
\end{document}

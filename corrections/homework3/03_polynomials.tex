\documentclass{report}
\usepackage{pgfplots}
\usepackage{booktabs}
\usepackage{polynom}

\input{preamble}
\input{macros}
\input{letterfonts}


\title{\Huge{Travaux dirigées}\\ Polynômes}
\author{\huge{A.Belcaid}}
\date{\today}

\begin{document}

\maketitle
% \newpage% or \cleardoublepage
% \pdfbookmark[<level>]{<title>}{<dest>}
\pdfbookmark[section]{\contentsname}{toc}
\tableofcontents
\pagebreak
\newcommand{\Rr}{\mathbb{R}};
\chapter{}

%{{{ Exercice 1
\section{Exercice 1} % (fold)
\label{sec:Exercice 1}
\qs{Division Euclidienne}{
Pour les trois cas listés, calculer la division euclidienne de $P$ par
$Q$.

\begin{enumerate}
  \item $P=X^4 + 5X^3 + 12X^2 + 19X - 7$ et $Q=X^2 + 3X-1$.
  \item $P=X^4-4X^3-9X^2+27X+38$ et $Q=X^2-X-7$.
  \item $P=X^5-X^2+2$ et $Q = X^2 +1$.
\end{enumerate}
}

\begin{myproof}

\begin{enumerate}
  \item Pour le premier cas on obtient:

    $$
    \polylongdiv[style=D]{X^4 + 5X^3 + 12X^2 + 19X - 7}{ X^2 + 3X-1}
    $$

  \item Pour le deuxième cas:
    $$
    \polylongdiv[style=D]{X^4-4X^3-9X^2+27X+38}{X^2-X-7}
    $$

  \item Finalement pour le troisième cas:

    $$
    \polylongdiv[style=D]{X^5-X^2+2}{X^2 +1}
    $$
\end{enumerate}

\end{myproof}

% section section name (end)
%}}}

%{{{ Exercice 2
\section{Exercice 2} % (fold)
\label{sec:Execice 2}
\qs{Expression du reste}
{

Soit $P\in \Rr[X]$, $a,b\in \Rr$, $a\neq b$. Sachant que le reste de la
division euclidienne de $P$ par $(X-a)$ vaut $1$ et que celui de $P$ par
$(X-b)$ vaut $-1$.

\begin{enumerate}
  \item Évaluer l'image $P(a)$ et $P(b)$?
  \item On note $R\in \Rr[X]$ le reste de la division euclidienne de $P$ par
    $(X-a)(X-b)$. Quel  sera le degré de $R$?
  \item En déduire l'expression de $R$.
\end{enumerate}
}

\begin{myproof}
  \begin{enumerate}
    \item 
  On sait que $$P(X)=(X-a)Q_1(x) + 1$$
  Ceci implique que 
  $$
  P(a) = 1
  $$

  De meme on obtient que 
  $$
  P(b) = -1
  $$
\item Puisque le cardinal de $(X-a)(x-b)$ est $2$ le reste $R$ doit être au plus de cardinal $1$.

  $$
  P(X) = (X-a)(X-b)Q(x) + \alpha X  + \beta
  $$

  On evalue cette expression en $a$ et en $b$, on trouve le systeme:

  $$
  \begin{cases}
    \alpha a + \beta &= 1\\
    \alpha b + \beta &= -1\\
  \end{cases}
  $$
  La résolution de système nous donne:
  $$
  \alpha = \dfrac{2}{a-b} \text{ et } \beta = \dfrac{-a-b}{a-b}
  $$
  
  Le reste recherché est donc
  $$
  \dfrac{2}{a-b}X + \dfrac{-a-b}{a-b}
  $$


  \end{enumerate}
\end{myproof}

% section section name (end)
%}}}

%{{{ Exercice 3
\newpage
\section{Exercice 3} % (fold)
\label{sec:Exercice 3}

\qs{}
{
On se propose de déterminer l'ensemble

\begin{equation*}
  E = \left\{P\in \Rr[X]\;\; P(X^2)=(X^3+1)P(X)\right\} 
\end{equation*}

\begin{enumerate}
  \item Démontrer que le polynôme nul ainsi\\  que le polynôme $X^3-1$ sont
    dans $E$.\\[4pt]
  \item Soit $P\in E$, non nul.
    \begin{enumerate}
      \small
    \item Démontrer que $P(1)=0$ puis que $P^{'}(0)=P^{''}(0)=0$.
    \item En effectuant la division euclidienne de $P$ par
      $X^3-1$, démontrer qu'il existe $\lambda\in \Rr$ tel que
      $$
      P(X) = \lambda (X^3-1)
      $$
    \end{enumerate}
  \item En déduire l'ensemble $E$.
\end{enumerate}
}
\begin{myproof}
  \begin{enumerate}
    \item On demontre que le polynôme nul et $X^3-1$ sont dans $E$..
      \begin{itemize}
        \item On a 
          $$ 0(X^2) = 0 = (X^3+1)0(X) $$ donc le polynome nul est dans $E$.
        \item Soit $P_1 = X^3 - 1$, on as:
          $$
          P_1(X^2) = X^6 - 1
          $$
          et 
          $$
          (X^3+1)P_1(X) = (X^3+1)(X^3-1) = X^6 - 1
          $$
          Ainsi on as aussi que $P_1$ est dans $E$.
      \end{itemize}
    \item On considere alors $P\in E$. selon la definition on aura:
      $$
      P(1) = (1 + 1) P(1)
      $$
      ce qui implique que 
      $$
      P(1) = 0
      $$
      On dérive la relation de l'équation on obtient:

      $$
      2P^{'}(X^2) = 3X^2P(X) + (X^3+1)P^{'}(X)
      $$
      ce qui prouve que $P^{'}(0)= 0$.

      On dérive on deuxième fois:

      $$
      4P^{''}(X^2) = 6XP(X) + 3X^2P^{''}(X) + 3X^2P^{'}(X) + (X^3+1)P^{''}(X)
      $$

      on injecte $0$ dans cette équation on trouve que:
      $$
      P^{''}(0) = 0
      $$
    \item Une analyse de dégrée de $P$ selon l'équation vérifiée implique que :
      $$
      2deg(P) = 3  + deg(P)
      $$
      Ce qui implique que 
      $$
      deg(P) = 3
      $$
      Ainsi l'expression de la division euclidienne d'un tel polynome par $X^3-1$ donnera :

      $$
      P = \lambda (X^3 - 1) + \beta\quad \lambda,\beta \in \Rr
      $$

      Sachant que $P(1) = 0$ on trouve forcement que $\beta =0$.
      Ainsi

      $$
      P = \lambda (X^3-1)
      $$

  \end{enumerate}
\end{myproof}
% section section name (end)
%}}}

\end{document}

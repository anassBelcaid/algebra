\documentclass{report}
\usepackage{pgfplots}
\usepackage{booktabs}
\usepackage{polynom}

\input{preamble}
\input{macros}
\input{letterfonts}


\title{\Huge{Travaux dirigées}\\ Polynômes}
\author{\huge{A.Belcaid}}
\date{\today}

\begin{document}

\maketitle
% \newpage% or \cleardoublepage
% \pdfbookmark[<level>]{<title>}{<dest>}
\pdfbookmark[section]{\contentsname}{toc}
\tableofcontents
\pagebreak
\newcommand{\Rr}{\mathbb{R}};
\chapter{}

%{{{ Exercice 1
\section{Exercice 1} % (fold)
\label{sec:Exercice 1}
\qs{Division Euclidienne}{
Pour les trois cas listés, calculer la division euclidienne de $P$ par
$Q$.

\begin{enumerate}
  \item $P=X^4 + 5X^3 + 12X^2 + 19X - 7$ et $Q=X^2 + 3X-1$.
  \item $P=X^4-4X^3-9X^2+27X+38$ et $Q=X^2-X-7$.
  \item $P=X^5-X^2+2$ et $Q = X^2 +1$.
\end{enumerate}
}

\begin{myproof}

\begin{enumerate}
  \item Pour le premier cas on obtient:

    $$
    \polylongdiv[style=D]{X^4 + 5X^3 + 12X^2 + 19X - 7}{ X^2 + 3X-1}
    $$

  \item Pour le deuxième cas:
    $$
    \polylongdiv[style=D]{X^4-4X^3-9X^2+27X+38}{X^2-X-7}
    $$

  \item Finalement pour le troisième cas:

    $$
    \polylongdiv[style=D]{X^5-X^2+2}{X^2 +1}
    $$
\end{enumerate}

\end{myproof}

% section section name (end)
%}}}

%{{{ Exercice 2
\section{Exercice 2} % (fold)
\label{sec:Execice 2}
\qs{Expression du reste}
{

Soit $P\in \Rr[X]$, $a,b\in \Rr$, $a\neq b$. Sachant que le reste de la
division euclidienne de $P$ par $(X-a)$ vaut $1$ et que celui de $P$ par
$(X-b)$ vaut $-1$.

\begin{enumerate}
  \item Évaluer l'image $P(a)$ et $P(b)$?
  \item On note $R\in \Rr[X]$ le reste de la division euclidienne de $P$ par
    $(X-a)(X-b)$. Quel  sera le degré de $R$?
  \item En déduire l'expression de $R$.
\end{enumerate}
}

\begin{myproof}
  \begin{enumerate}
    \item 
  On sait que $$P(X)=(X-a)Q_1(x) + 1$$
  Ceci implique que 
  $$
  P(a) = 1
  $$

  De meme on obtient que 
  $$
  P(b) = -1
  $$
\item Puisque le cardinal de $(X-a)(x-b)$ est $2$ le reste $R$ doit être au plus de cardinal $1$.

  $$
  P(X) = (X-a)(X-b)Q(x) + \alpha X  + \beta
  $$

  On evalue cette expression en $a$ et en $b$, on trouve le systeme:

  $$
  \begin{cases}
    \alpha a + \beta &= 1\\
    \alpha b + \beta &= -1\\
  \end{cases}
  $$
  La résolution de système nous donne:
  $$
  \alpha = \dfrac{2}{a-b} \text{ et } \beta = \dfrac{-a-b}{a-b}
  $$
  
  Le reste recherché est donc
  $$
  \dfrac{2}{a-b}X + \dfrac{-a-b}{a-b}
  $$


  \end{enumerate}
\end{myproof}

% section section name (end)
%}}}

%{{{ Exercice 3
\newpage
\section{Exercice 3} % (fold)
\label{sec:Exercice 3}

\qs{}
{
On se propose de déterminer l'ensemble

\begin{equation*}
  E = \left\{P\in \Rr[X]\;\; P(X^2)=(X^3+1)P(X)\right\} 
\end{equation*}

\begin{enumerate}
  \item Démontrer que le polynôme nul ainsi\\  que le polynôme $X^3-1$ sont
    dans $E$.\\[4pt]
  \item Soit $P\in E$, non nul.
    \begin{enumerate}
      \small
    \item Démontrer que $P(1)=0$ puis que $P^{'}(0)=P^{''}(0)=0$.
    \item En effectuant la division euclidienne de $P$ par
      $X^3-1$, démontrer qu'il existe $\lambda\in \Rr$ tel que
      $$
      P(X) = \lambda (X^3-1)
      $$
    \end{enumerate}
  \item En déduire l'ensemble $E$.
\end{enumerate}
}
\begin{myproof}
  \begin{enumerate}
    \item On demontre que le polynôme nul et $X^3-1$ sont dans $E$..
      \begin{itemize}
        \item On a 
          $$ 0(X^2) = 0 = (X^3+1)0(X) $$ donc le polynome nul est dans $E$.
        \item Soit $P_1 = X^3 - 1$, on as:
          $$
          P_1(X^2) = X^6 - 1
          $$
          et 
          $$
          (X^3+1)P_1(X) = (X^3+1)(X^3-1) = X^6 - 1
          $$
          Ainsi on as aussi que $P_1$ est dans $E$.
      \end{itemize}
    \item On considere alors $P\in E$. selon la definition on aura:
      $$
      P(1) = (1 + 1) P(1)
      $$
      ce qui implique que 
      $$
      P(1) = 0
      $$
      On dérive la relation de l'équation on obtient:

      $$
      2P^{'}(X^2) = 3X^2P(X) + (X^3+1)P^{'}(X)
      $$
      ce qui prouve que $P^{'}(0)= 0$.

      On dérive on deuxième fois:

      $$
      4P^{''}(X^2) = 6XP(X) + 3X^2P^{''}(X) + 3X^2P^{'}(X) + (X^3+1)P^{''}(X)
      $$

      on injecte $0$ dans cette équation on trouve que:
      $$
      P^{''}(0) = 0
      $$
    \item Une analyse de dégrée de $P$ selon l'équation vérifiée implique que :
      $$
      2deg(P) = 3  + deg(P)
      $$
      Ce qui implique que 
      $$
      deg(P) = 3
      $$
      Ainsi l'expression de la division euclidienne d'un tel polynome par $X^3-1$ donnera :

      $$
      P = \lambda (X^3 - 1) + \beta\quad \lambda,\beta \in \Rr
      $$

      Sachant que $P(1) = 0$ on trouve forcement que $\beta =0$.
      Ainsi

      $$
      P = \lambda (X^3-1)
      $$

  \end{enumerate}
\end{myproof}
% section section name (end)
%}}}

%{{{ Exercice 4
\section{Exercice 4} % (fold)
\label{sec:Exercice 4}
\qs{Calcul PGCD}{
Pour chaque cas, déterminer le \textbf{PGCD} entre $P$ et $Q$.

\begin{enumerate}
  \item $P=X^4-3X^3 + X^2 + 4$ et $Q=X^3-3X^2+3X-2$.
  \item $P=X^5-X^4+2X^3-2X^2+2X-1$ et $Q=X^5-X^4+2X^2-2X+1$.
  \item $P=X^n -1$ et $Q=(X-1)^n$.
\end{enumerate}
}

\begin{myproof}
  pour chaque cas on as:
  \begin{enumerate}
    \item 
      $$
      \polylonggcd{X^4-3X^3 + X^2 + 4}{X^3-3X^2+3X-2}
      $$
      Ce qui donne que le PGCD est :

      $$
      X - 2
      $$
    \item 
      $$
      \polylonggcd{X^5-X^4+2X^3-2X^2+2X-1}{X^5-X^4+2X^2-2X+1}
      $$
      Ainsi le PGCD est
      $$
      X^2 - X + 1
      $$
    \item 

      On as 
      $$
      (X^n - 1) = (X - 1)\sum_{i=0}^{n-1}X^i
      $$
      Ainsi le seul facteur commun entre ces deux polynomes est:

      $$
      (X - 1)
      $$
  \end{enumerate}

\end{myproof}

% section section name (end)
%}}}

%{{{ Exercice 5
\qs{Formule de Bezout}{
Trouver deux polynômes $U$ et $V$ de $\Rr[X]$ tel que 
$$
AU + BV = 1
$$

où $A = X^7-X-1$ et $B=X^5-1$.

}
\begin{myproof}
  On utilise l'algorithme d'Euclide. On a

  $$
  \polylonggcd{X^7-X -1}{X^5-1}
  $$
  On remonte ensuite les calculs. On va partir plutot de 
  $$
  11 = -25(X^2 - X - 1) + (5X-7)(5X+2)
  $$
  Pour eviter de trainer des fractions. On trouve successivement:

  \begin{eqnarray*}
    11 &= & -25(X^2 - X - 1) + (5X - 7)\left((X^5-1) - (X^2-X-1)(X^3+X^2+2X+3)\right)\\[4pt]
       & =& \left(-5X^4 + 2X^3 - 3X^2-X -4\right)\left(X^2-X-1\right) + \left(5X -7\right)\left(X^5-1\right)\\[4pt]
       &=&(-5X^4 + 2X^3 - 3X^2 -X -4)(X^7-X-1) + (5X^6-2X^5+3X^4+X^3+4X^2+5X-7)(X^5-1)
  \end{eqnarray*}
  Finalement il suffit de diviser par $11$ pour trouver $U$ et $V$.
\end{myproof}
%}}}


%{{{ Exercice 6
\section{Exercice 6} % (fold)
\label{sec:Exercice 6}

% section section name (end)
\qs{}
{

  Soient $P$ et $Q$ des polynômes de $\mathbb{C}[X]$ non constants. Montrer  que
l'équivalence entre:

\begin{enumerate}
  \item $P$ et $Q$ ont un facteur commun.
  \item il existe $A,B\in \mathbb{C}[X]$, $A\neq0$, $B\neq 0$, tel que
    $$
    AP = BQ
    $$
  et $\text{deg}(A)< \text{deg}(Q),\quad \text{deg}(B)<\text{deg}(P)$ 
\end{enumerate}
}

\begin{myproof}
  On commence par la premiere implication
  \begin{enumerate}
    \item On suppose que $P$ et $Q$ ont un facteur commun $D$. On factorise alors 
      $$
      \left\{\begin{array}{lll}
          P&=&DB\\[4pt]
          Q&=&DA
        \end{array}
      \right.
      $$
      Ce qui implique que:
      $$
      AP = ADB = BDA = BQ
      $$
    \item Pour la reciproque, On suppose que $P$ et $Q$ sont premiers entre eux
      $$
      P \wedge Q = 1 \quad \text{ et } AP = BQ
      $$
      alors 
      $$
      P | BQ
      $$
      et par le theoreme de Gauss on aura
      $$
      P | B
      $$
      Ce qui est absurde.
  \end{enumerate}
\end{myproof}
%}}}

%{{{ Exercice 7
\section{Exercice 7} % (fold)
\label{sec:Exercice 7}

% section section name (end)
\qs{Racines}
{
Quel est pour $n\geq 1$ l'ordre de multiplicité de $2$ du polynôme:

$$
P_n(X)= n X^{n+2}-(4n+1)X^{n+1}+4(n+1)X^n-4X^{n-1}
$$
}
\begin{myproof}
  On verifie d'abord que $P_n(2)= 0$ et donc $2$ est une racine de $P_n$.\\

  On calcule maintenant la derivee 
  $$
  P^{'}_n(X) = n(n+2)X^{n+1} - (4n+1)(n+1)X^n + 4n(n+1)X^{n-1} - 4(n-1)X^{n-2}
  $$
  si $n=1$, le dernier terme est interprete comme le polynome nul. En particulier, on a:

  \begin{eqnarray}
    P^{'}_n(2)&=& n(n+2)2^{n+1} - (4n+1)(n+1)2^n + 4n(n+1)2^{n-1} -4(n-1)2^{n-2}\\[4pt]
              &=& 2^{n-2}\left(8n(n+2) -4(4n+1)(n+1) + 8n(n+1) - 4(n-1)\right)\\[4pt]
              &=&0
  \end{eqnarray}
  On derive une fois encore
  $$
  P^{''}_n(X) = n(n+1)(n+2)X^n -(4n+1)n(n+1)X^{n-1} + 4n(n+1)(n-1)X^{n-2} 4(n-1)(n-2)X^{n-3}
  $$
  d'ou l'on tire 
  $$
  P^{''}_n(2) = 2^{n-3}\left(8n(n+1)(n+2) - 4(4n+1)n(n+1)+ 8n(n+1)(n-1) -4(n-1)(n-2)\right)
  $$
  $$
P^{''}_n(2) = 2^n(2n-1)
  $$
  Puisque $2n-1$ ne s'anulle pas quand n est un entier, on as 

  $$
  P_n(2) = P^{'}_n(2) = 0
  $$
  et 
  $$
  P^{''}_n(2) \ne 0.
  $$

  Aini $2$ est une racine de multiplicite $2$.
\end{myproof}
%}}}

%{{{ 
\newpage
\section{Exercie 8} % (fold)
\label{sec:Exercie 8}

% section section name (end)
\qs{}
{
  Soit $P(X)=a_nX^n+\ldots+a_0$ un polynôme dans $\mathbb{Z}[X]$. On suppose aussi
que $P$ admet une racine rationnelle $r=\frac{p}{q}$ tel que $p\wedge q
=1$.
\begin{enumerate}
  \item Développer que la forme $P(r)=0$.
  \item Démontrer que $p\;|\; a_0$.
  \item Prouver que $q\;|\; a_n$
  \item En déduire que $P=X^5 - X^2+1$ n'admet pas de racines dans
    $\mathbb{Q}$.
\end{enumerate}
}
\begin{myproof}
  \begin{enumerate}
    \item 
      \begin{eqnarray*}
        P\left(\dfrac{p}{q}\right) &=& 0\\[4pt]
        a_np^n + a_{n-1}p^{n-1}q+\ldots+a_1pq^{n-1} + a_0q^n&=& 0
      \end{eqnarray*}
      On commence par isoler $a_0q^n$ et on trouve que:
      $$
      p\left(a_n p^{n-1} + a_{n-1}p^{n-2}q + \ldots + a_1q^{n-1}\right) = -a_0q^n
      $$
      Puisque $p \wedge q = 1 $, on aura $p | a_0q^n$.\\

      On isole aussi $a_np^n$, on trouve que:
      $$
      q\left(a_{n-1}p^{n-1} + \ldots+ a_0q^{n-1}\right) = -a_np^n
      $$
      De meme analyse, on conclut que 
      $$
      q | a_np^n \quad \text { puisque } p \wedge q = 1
      $$
      Par consequent, si le polynome $X^5 - X^2 +1$ admet une racine rationnelle $p/q$, alors $p|1$ et $q|1$. Ainsi
      $$
      \vert p \vert = 1 \quad \text{ et } \vert q \vert = 1
      $$
      Ainsi les seuls racines possibles sont $-1$ et $1$. Or, elles ne sont pas des racines de $P$ 
      Ainsi $P$ n'admet pas de racines rationnelle.
  \end{enumerate}
\end{myproof}
%}}}

%{{{Exercice 9
\section{Exercice 9}
\newcommand{\Cc}{\mathbb{C}}
% \newcommand{\Rr}{\mathbb{R}}
\qs{}
{
\begin{enumerate}
  \item Le polynôme $P(X) = X^4 + X^2 + 1$ est il irréductible  dans $\Rr[X]$?
dans $\Cc[X]$?
\item La relation $P\mathcal{R} Q \iff P\; \text{divise}\; Q$  est-elle une
relation d'ordre?
\end{enumerate}
}
\begin{myproof}
  On sait que les seuls polynômes \textbf{irréductibles} dans $\Rr[X]$
  \begin{itemize}
    \item Les polynômes de degré 1.
    \item Les polynômes de degré 2 avec un déterminant négatif.
  \end{itemize}
  et dans $\Cc[X]$ les polynômes de degré $1$.\\

  Ainsi le polynôme $X^4 + X^2 + 1$ ni irréductible ni dans $\Rr[X]$ ni dans $\Cc[X]$.


\end{myproof}
%}}}

%{{{
\section{Exercice 10}
\qs{}{

Pour chaque polynôme, donner la décomposition en facteurs irréductibles dans
$\Rr[X]$

\begin{enumerate}
  \item $P_1(X) = X^4 + 1$
  \item $P_2(X) = X^8 - 1$
  \item $P_3(X) = \left(X^2 - X+1\right)^2+1$
\end{enumerate}

}
\begin{myproof}
  Voici la décomposition  de chaque polynôme:

  \begin{enumerate}
    \item Pour le premier exemple, on résout  l'équation classique
      $$
      X^4 = -1
      $$
      Ça nous donne que 

      $$
      X^4 + 1 = (X-e^{\frac{i\pi}{4}})(X-e^{\frac{3i\pi}{4}})(X-e^{\frac{5i\pi}{4}})(X-e^{\frac{7i\pi}{4}})
      $$

      on regroupe les termes \textbf{conjugués} on trouve que

      $$
      X^4 + 1 =(X-e^{\frac{i\pi}{4}})(X-e^{\frac{7i\pi}{4}}) (X-e^{\frac{5i\pi}{4}})(X-e^{\frac{5i\pi}{4}})
      $$ $$ X^4 + 1 = (X^2 - \sqrt{2}X + 1)(X^2 + \sqrt{2}X +1)
      $$
      Les deux sont de degré deux et de déterminant négatifs. Ainsi ils sont irréductibles dans $\Rr[X]$.
    \item Pour le deuxième exemple on trouve que
      \begin{eqnarray*}
        X^8 - 1 &=& (X^4 - 1)(X^4 + 1 )\\
         &=& (X^2 - 1)(X^2 + 1)(X^4 + 1)\\
         &=& (X-1)(X+1)(X^2+1)(X^4 +1)
    \end{eqnarray*}
    Pour $X^4 + 1$ on utilise le même processus que la question $1$ et on trouve:
    $$
    X^8 - 1 = (X-1)(X+1)(X^2+1)(X^2 - \sqrt{2}X + 1)(X^2 + \sqrt{2}X +1)
    $$
  \item Pour le polynôme final
    $$
    (X^2 - X + 1)^2 + 1 = (X^2 - X + 1 -i)(X^2 - X + 1 + i)
    $$
    On factorise chaque polynôme de degré $2$ dans $\Cc$, on trouve que:

    $$
    (X^2 - X + 1)^2 + 1 = (X+i)(X-1-i)(X-i)(X-1+i)
    $$
    On regroupe les termes conjugués:

    $$
    (X^2 - X + 1)^2 + 1 = (X^2 + 1)(X^2-2X + 2)
    $$
  \end{enumerate}
\end{myproof}
%}}}

%{{{ Exercice 11
\section{Exercice 11} % (fold)
\qs{}{


Soit $P$ le polynôme définit par:

$$
P(X) = 2X^4 + X^2 -3
$$
\begin{enumerate}
  \item 
Décomposer $P$ en facteurs irréductibles dans $\Rr[X]$.
\end{enumerate}

}
\begin{myproof}
  Il s'agit d'un  polynôme spécial \textbf{bicarré} qui s'écrit sous la forme:

  $$
  P(X) = Q(X^2) \quad \text{ ou } Q(X) = X^2 + X - 3
  $$

  On commence alors par la décomposition de $Q$

  $$
  Q(X) = 2(X-1)(x + \frac{3}{2})
  $$

  On en déduit alors que 
  $$
  P(X) = 2(X^2- 1)(X^2 + \frac{3}{2}) = 2(X-1)(X+1)(X^2 + \frac{3}{2})
  $$
\end{myproof}
% section section name (end)
%}}}

%{{{ Exercice 12
\section{Exercice 12} % (fold)

\qs{}
{

Soit le polynôme $P(X) = X^4-6X^3+9X^2+9$.

\begin{enumerate}
  \item Décomposer $X^4-6X^3+9X^2$ en produit de facteurs irréductibles dans
    $\Rr[X]$.
  \item En déduire une décomposition de $P$ dans $\Rr[X]$.

  \item Même question pour $\Cc[X]$.
\end{enumerate}

}
\begin{myproof}
  \begin{enumerate}
\item
  On écrit simplement
  $$
  X^4 - 6X^3 + 9X^2 + 9 = X^2(X^2-6X+9) = X^2 (X-3)^2
  $$

\item L'astuce pour obtenir une identité remarquable est de considèrera que $9 = -(3i)^2$. Avec ceci on trouve que 

  \begin{eqnarray*}
    X^4-6X^3+9X^2 + 9 &=& \left(X(X-3)\right)^2 - (3i)^2\\
                      &=& (X(X-3) - 3i)(X(X-3) + 3i)\\
                      &=& (X^2 - 3X -3i)(X^2-3X+3i)
  \end{eqnarray*}
  Pour finalisme la décomposition il suffit de calculer le discriminant de chaque polynôme pour obtenir les racines.
\item  Dans $\Cc[X]$ la décomposition s'écrit comme
  $$
  P(X) = (X-\alpha_1)(X-\alpha_2)(X - \beta_1)(X- \beta_2)
  $$
  Ou $\alpha_1, \alpha_2, \beta_1$ et $\beta_2$ sont les racines calculées dans la question précédente.
\end{enumerate}
\end{myproof}
% section section name (end)

%}}}
%{{{ Exercice 13
\qs{Factorisation simultanée}{

On considère les deux polynômes suivants:

\begin{itemize}
  \item $P(X) = X^3 - 9X^2 + 26X -24$
  \item $Q(X) = X^3 - 7X^2 + 7X + 15$.
\end{itemize}
\begin{enumerate}
  \item Sachant que $P$ et $Q$ admettent une racine \textbf{commune} $a$,
    Quelle est la relation entre $(X-a)$ et $\text{pgcd}(P,Q)$?
  \item En appliquant l'algorithme d'Euclide, montrer  que le \textbf{pgcd} de $P$
    et $Q$ est $X-3$?
  

  \item Calculer le polynôme $P_1$ tel que
    $$ P = (X-3)P_1$$

  \item Même question pour $Q_1$ tel que:

    $$Q = (X-3)Q_1$$.

  \item En déduire une décomposition en facteurs\\ irréductibles dans
    $\Rr[X]$ de $P$ et $Q$.
\end{enumerate}
}
\begin{myproof}

  Si $a$ possède une racine commune  $a$, alors le polynome $(X-1)$ divise le \textbf{PGCD}(P, Q). Ainsi on calcul le pgcd de ces deux polynômes

$$
\polylonggcd{X^3 - 9X^2 + 26X  - 24}{X^3 - 7X^2 + 7X + 15}
$$

On trouve que le pgcd est $(X-3)$ Ainsi la valeur de $a=3$.\\

Maintenant qu'on as une racine commune on peut diminuer le degré de $P$ et de $Q$ en réalisant la division euclidienne.

$$
\polylongdiv[style=D]{X^3 - 9X^2 + 26 X -24}{X-3}
$$
Ainsi on as:

$$
P = (X-3)(X^2-6X+8) = (X-3)(X-2)(X-4)
$$

On reprend le meme processus pour $Q$

$$
\polylongdiv[style=D]{X^3-7X^2 + 7X + 15}{X-3}
$$

Ainsi 

$$
Q(X) = (X-3)(X^2-6X+8) = (X-3)(X-3)(X-5)
$$



\end{myproof}

%}}}
\end{document}

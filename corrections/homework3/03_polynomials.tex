\documentclass{report}
\usepackage{pgfplots}
\usepackage{booktabs}
\usepackage{polynom}

%%%%%%%%%%%%%%%%%%%%%%%%%%%%%%%%%
% PACKAGE IMPORTS
%%%%%%%%%%%%%%%%%%%%%%%%%%%%%%%%%


\usepackage[tmargin=2cm,rmargin=1in,lmargin=1in,margin=0.85in,bmargin=2cm,footskip=.2in]{geometry}
\usepackage{amsmath,amsfonts,amsthm,amssymb,mathtools}
\usepackage[varbb]{newpxmath}
\usepackage{xfrac}
\usepackage[makeroom]{cancel}
\usepackage{mathtools}
\usepackage{bookmark}
\usepackage{enumitem}
\usepackage{hyperref,theoremref}
\hypersetup{
	pdftitle={Assignment},
	colorlinks=true, linkcolor=doc!90,
	bookmarksnumbered=true,
	bookmarksopen=true
}
\usepackage[most,many,breakable]{tcolorbox}
\usepackage{xcolor}
\usepackage{varwidth}
\usepackage{varwidth}
\usepackage{etoolbox}
%\usepackage{authblk}
\usepackage{nameref}
\usepackage{multicol,array}
\usepackage{tikz-cd}
\usepackage[ruled,vlined,linesnumbered]{algorithm2e}
\usepackage{comment} % enables the use of multi-line comments (\ifx \fi) 
\usepackage{import}
\usepackage{xifthen}
\usepackage{pdfpages}
\usepackage{transparent}

\newcommand\mycommfont[1]{\footnotesize\ttfamily\textcolor{blue}{#1}}
\SetCommentSty{mycommfont}
\newcommand{\incfig}[1]{%
    \def\svgwidth{\columnwidth}
    \import{./figures/}{#1.pdf_tex}
}

\usepackage{tikzsymbols}
\renewcommand\qedsymbol{$\Laughey$}


%\usepackage{import}
%\usepackage{xifthen}
%\usepackage{pdfpages}
%\usepackage{transparent}


%%%%%%%%%%%%%%%%%%%%%%%%%%%%%%
% SELF MADE COLORS
%%%%%%%%%%%%%%%%%%%%%%%%%%%%%%



\definecolor{myg}{RGB}{56, 140, 70}
\definecolor{myb}{RGB}{45, 111, 177}
\definecolor{myr}{RGB}{199, 68, 64}
\definecolor{mytheorembg}{HTML}{F2F2F9}
\definecolor{mytheoremfr}{HTML}{00007B}
\definecolor{mylenmabg}{HTML}{FFFAF8}
\definecolor{mylenmafr}{HTML}{983b0f}
\definecolor{mypropbg}{HTML}{f2fbfc}
\definecolor{mypropfr}{HTML}{191971}
\definecolor{myexamplebg}{HTML}{F2FBF8}
\definecolor{myexamplefr}{HTML}{88D6D1}
\definecolor{myexampleti}{HTML}{2A7F7F}
\definecolor{mydefinitbg}{HTML}{E5E5FF}
\definecolor{mydefinitfr}{HTML}{3F3FA3}
\definecolor{notesgreen}{RGB}{0,162,0}
\definecolor{myp}{RGB}{197, 92, 212}
\definecolor{mygr}{HTML}{2C3338}
\definecolor{myred}{RGB}{127,0,0}
\definecolor{myyellow}{RGB}{169,121,69}


%%%%%%%%%%%%%%%%%%%%%%%%%%%%
% TCOLORBOX SETUPS
%%%%%%%%%%%%%%%%%%%%%%%%%%%%

\setlength{\parindent}{1cm}
%================================
% THEOREM BOX
%================================

\tcbuselibrary{theorems,skins,hooks}
\newtcbtheorem[number within=section]{Theorem}{Theorem}
{%
	enhanced,
	breakable,
	colback = mytheorembg,
	frame hidden,
	boxrule = 0sp,
	borderline west = {2pt}{0pt}{mytheoremfr},
	sharp corners,
	detach title,
	before upper = \tcbtitle\par\smallskip,
	coltitle = mytheoremfr,
	fonttitle = \bfseries\sffamily,
	description font = \mdseries,
	separator sign none,
	segmentation style={solid, mytheoremfr},
}
{th}

\tcbuselibrary{theorems,skins,hooks}
\newtcbtheorem[number within=chapter]{theorem}{Theorem}
{%
	enhanced,
	breakable,
	colback = mytheorembg,
	frame hidden,
	boxrule = 0sp,
	borderline west = {2pt}{0pt}{mytheoremfr},
	sharp corners,
	detach title,
	before upper = \tcbtitle\par\smallskip,
	coltitle = mytheoremfr,
	fonttitle = \bfseries\sffamily,
	description font = \mdseries,
	separator sign none,
	segmentation style={solid, mytheoremfr},
}
{th}


\tcbuselibrary{theorems,skins,hooks}
\newtcolorbox{Theoremcon}
{%
	enhanced
	,breakable
	,colback = mytheorembg
	,frame hidden
	,boxrule = 0sp
	,borderline west = {2pt}{0pt}{mytheoremfr}
	,sharp corners
	,description font = \mdseries
	,separator sign none
}

%================================
% Corollery
%================================
\tcbuselibrary{theorems,skins,hooks}
\newtcbtheorem[number within=section]{Corollary}{Corollary}
{%
	enhanced
	,breakable
	,colback = myp!10
	,frame hidden
	,boxrule = 0sp
	,borderline west = {2pt}{0pt}{myp!85!black}
	,sharp corners
	,detach title
	,before upper = \tcbtitle\par\smallskip
	,coltitle = myp!85!black
	,fonttitle = \bfseries\sffamily
	,description font = \mdseries
	,separator sign none
	,segmentation style={solid, myp!85!black}
}
{th}
\tcbuselibrary{theorems,skins,hooks}
\newtcbtheorem[number within=chapter]{corollary}{Corollary}
{%
	enhanced
	,breakable
	,colback = myp!10
	,frame hidden
	,boxrule = 0sp
	,borderline west = {2pt}{0pt}{myp!85!black}
	,sharp corners
	,detach title
	,before upper = \tcbtitle\par\smallskip
	,coltitle = myp!85!black
	,fonttitle = \bfseries\sffamily
	,description font = \mdseries
	,separator sign none
	,segmentation style={solid, myp!85!black}
}
{th}


%================================
% LENMA
%================================

\tcbuselibrary{theorems,skins,hooks}
\newtcbtheorem[number within=section]{Lenma}{Lenma}
{%
	enhanced,
	breakable,
	colback = mylenmabg,
	frame hidden,
	boxrule = 0sp,
	borderline west = {2pt}{0pt}{mylenmafr},
	sharp corners,
	detach title,
	before upper = \tcbtitle\par\smallskip,
	coltitle = mylenmafr,
	fonttitle = \bfseries\sffamily,
	description font = \mdseries,
	separator sign none,
	segmentation style={solid, mylenmafr},
}
{th}

\tcbuselibrary{theorems,skins,hooks}
\newtcbtheorem[number within=chapter]{lenma}{Lenma}
{%
	enhanced,
	breakable,
	colback = mylenmabg,
	frame hidden,
	boxrule = 0sp,
	borderline west = {2pt}{0pt}{mylenmafr},
	sharp corners,
	detach title,
	before upper = \tcbtitle\par\smallskip,
	coltitle = mylenmafr,
	fonttitle = \bfseries\sffamily,
	description font = \mdseries,
	separator sign none,
	segmentation style={solid, mylenmafr},
}
{th}


%================================
% PROPOSITION
%================================

\tcbuselibrary{theorems,skins,hooks}
\newtcbtheorem[number within=section]{Prop}{Proposition}
{%
	enhanced,
	breakable,
	colback = mypropbg,
	frame hidden,
	boxrule = 0sp,
	borderline west = {2pt}{0pt}{mypropfr},
	sharp corners,
	detach title,
	before upper = \tcbtitle\par\smallskip,
	coltitle = mypropfr,
	fonttitle = \bfseries\sffamily,
	description font = \mdseries,
	separator sign none,
	segmentation style={solid, mypropfr},
}
{th}

\tcbuselibrary{theorems,skins,hooks}
\newtcbtheorem[number within=chapter]{prop}{Proposition}
{%
	enhanced,
	breakable,
	colback = mypropbg,
	frame hidden,
	boxrule = 0sp,
	borderline west = {2pt}{0pt}{mypropfr},
	sharp corners,
	detach title,
	before upper = \tcbtitle\par\smallskip,
	coltitle = mypropfr,
	fonttitle = \bfseries\sffamily,
	description font = \mdseries,
	separator sign none,
	segmentation style={solid, mypropfr},
}
{th}


%================================
% CLAIM
%================================

\tcbuselibrary{theorems,skins,hooks}
\newtcbtheorem[number within=section]{claim}{Claim}
{%
	enhanced
	,breakable
	,colback = myg!10
	,frame hidden
	,boxrule = 0sp
	,borderline west = {2pt}{0pt}{myg}
	,sharp corners
	,detach title
	,before upper = \tcbtitle\par\smallskip
	,coltitle = myg!85!black
	,fonttitle = \bfseries\sffamily
	,description font = \mdseries
	,separator sign none
	,segmentation style={solid, myg!85!black}
}
{th}




%================================
% EXAMPLE BOX
%================================

\newtcbtheorem[number within=section]{Example}{Example}
{%
	colback = myexamplebg
	,breakable
	,colframe = myexamplefr
	,coltitle = myexampleti
	,boxrule = 1pt
	,sharp corners
	,detach title
	,before upper=\tcbtitle\par\smallskip
	,fonttitle = \bfseries
	,description font = \mdseries
	,separator sign none
	,description delimiters parenthesis
}
{ex}

\newtcbtheorem[number within=chapter]{example}{Example}
{%
	colback = myexamplebg
	,breakable
	,colframe = myexamplefr
	,coltitle = myexampleti
	,boxrule = 1pt
	,sharp corners
	,detach title
	,before upper=\tcbtitle\par\smallskip
	,fonttitle = \bfseries
	,description font = \mdseries
	,separator sign none
	,description delimiters parenthesis
}
{ex}

%================================
% DEFINITION BOX
%================================

\newtcbtheorem[number within=section]{Definition}{Definition}{enhanced,
	before skip=2mm,after skip=2mm, colback=red!5,colframe=red!80!black,boxrule=0.5mm,
	attach boxed title to top left={xshift=1cm,yshift*=1mm-\tcboxedtitleheight}, varwidth boxed title*=-3cm,
	boxed title style={frame code={
					\path[fill=tcbcolback]
					([yshift=-1mm,xshift=-1mm]frame.north west)
					arc[start angle=0,end angle=180,radius=1mm]
					([yshift=-1mm,xshift=1mm]frame.north east)
					arc[start angle=180,end angle=0,radius=1mm];
					\path[left color=tcbcolback!60!black,right color=tcbcolback!60!black,
						middle color=tcbcolback!80!black]
					([xshift=-2mm]frame.north west) -- ([xshift=2mm]frame.north east)
					[rounded corners=1mm]-- ([xshift=1mm,yshift=-1mm]frame.north east)
					-- (frame.south east) -- (frame.south west)
					-- ([xshift=-1mm,yshift=-1mm]frame.north west)
					[sharp corners]-- cycle;
				},interior engine=empty,
		},
	fonttitle=\bfseries,
	title={#2},#1}{def}
\newtcbtheorem[number within=chapter]{definition}{Definition}{enhanced,
	before skip=2mm,after skip=2mm, colback=red!5,colframe=red!80!black,boxrule=0.5mm,
	attach boxed title to top left={xshift=1cm,yshift*=1mm-\tcboxedtitleheight}, varwidth boxed title*=-3cm,
	boxed title style={frame code={
					\path[fill=tcbcolback]
					([yshift=-1mm,xshift=-1mm]frame.north west)
					arc[start angle=0,end angle=180,radius=1mm]
					([yshift=-1mm,xshift=1mm]frame.north east)
					arc[start angle=180,end angle=0,radius=1mm];
					\path[left color=tcbcolback!60!black,right color=tcbcolback!60!black,
						middle color=tcbcolback!80!black]
					([xshift=-2mm]frame.north west) -- ([xshift=2mm]frame.north east)
					[rounded corners=1mm]-- ([xshift=1mm,yshift=-1mm]frame.north east)
					-- (frame.south east) -- (frame.south west)
					-- ([xshift=-1mm,yshift=-1mm]frame.north west)
					[sharp corners]-- cycle;
				},interior engine=empty,
		},
	fonttitle=\bfseries,
	title={#2},#1}{def}



%================================
% EXERCISE BOX
%================================

\makeatletter
\newtcbtheorem{question}{Question}{enhanced,
	breakable,
	colback=white,
	colframe=myb!80!black,
	attach boxed title to top left={yshift*=-\tcboxedtitleheight},
	fonttitle=\bfseries,
	title={#2},
	boxed title size=title,
	boxed title style={%
			sharp corners,
			rounded corners=northwest,
			colback=tcbcolframe,
			boxrule=0pt,
		},
	underlay boxed title={%
			\path[fill=tcbcolframe] (title.south west)--(title.south east)
			to[out=0, in=180] ([xshift=5mm]title.east)--
			(title.center-|frame.east)
			[rounded corners=\kvtcb@arc] |-
			(frame.north) -| cycle;
		},
	#1
}{def}
\makeatother

%================================
% SOLUTION BOX
%================================

\makeatletter
\newtcolorbox{solution}{enhanced,
	breakable,
	colback=white,
	colframe=myg!80!black,
	attach boxed title to top left={yshift*=-\tcboxedtitleheight},
	title=Solution,
	boxed title size=title,
	boxed title style={%
			sharp corners,
			rounded corners=northwest,
			colback=tcbcolframe,
			boxrule=0pt,
		},
	underlay boxed title={%
			\path[fill=tcbcolframe] (title.south west)--(title.south east)
			to[out=0, in=180] ([xshift=5mm]title.east)--
			(title.center-|frame.east)
			[rounded corners=\kvtcb@arc] |-
			(frame.north) -| cycle;
		},
}
\makeatother

%================================
% Question BOX
%================================

\makeatletter
\newtcbtheorem{qstion}{Question}{enhanced,
	breakable,
	colback=white,
	colframe=mygr,
	attach boxed title to top left={yshift*=-\tcboxedtitleheight},
	fonttitle=\bfseries,
	title={#2},
	boxed title size=title,
	boxed title style={%
			sharp corners,
			rounded corners=northwest,
			colback=tcbcolframe,
			boxrule=0pt,
		},
	underlay boxed title={%
			\path[fill=tcbcolframe] (title.south west)--(title.south east)
			to[out=0, in=180] ([xshift=5mm]title.east)--
			(title.center-|frame.east)
			[rounded corners=\kvtcb@arc] |-
			(frame.north) -| cycle;
		},
	#1
}{def}
\makeatother

\newtcbtheorem[number within=chapter]{wconc}{Wrong Concept}{
	breakable,
	enhanced,
	colback=white,
	colframe=myr,
	arc=0pt,
	outer arc=0pt,
	fonttitle=\bfseries\sffamily\large,
	colbacktitle=myr,
	attach boxed title to top left={},
	boxed title style={
			enhanced,
			skin=enhancedfirst jigsaw,
			arc=3pt,
			bottom=0pt,
			interior style={fill=myr}
		},
	#1
}{def}



%================================
% NOTE BOX
%================================

\usetikzlibrary{arrows,calc,shadows.blur}
\tcbuselibrary{skins}
\newtcolorbox{note}[1][]{%
	enhanced jigsaw,
	colback=gray!20!white,%
	colframe=gray!80!black,
	size=small,
	boxrule=1pt,
	title=\textbf{Note:-},
	halign title=flush center,
	coltitle=black,
	breakable,
	drop shadow=black!50!white,
	attach boxed title to top left={xshift=1cm,yshift=-\tcboxedtitleheight/2,yshifttext=-\tcboxedtitleheight/2},
	minipage boxed title=1.5cm,
	boxed title style={%
			colback=white,
			size=fbox,
			boxrule=1pt,
			boxsep=2pt,
			underlay={%
					\coordinate (dotA) at ($(interior.west) + (-0.5pt,0)$);
					\coordinate (dotB) at ($(interior.east) + (0.5pt,0)$);
					\begin{scope}
						\clip (interior.north west) rectangle ([xshift=3ex]interior.east);
						\filldraw [white, blur shadow={shadow opacity=60, shadow yshift=-.75ex}, rounded corners=2pt] (interior.north west) rectangle (interior.south east);
					\end{scope}
					\begin{scope}[gray!80!black]
						\fill (dotA) circle (2pt);
						\fill (dotB) circle (2pt);
					\end{scope}
				},
		},
	#1,
}

%%%%%%%%%%%%%%%%%%%%%%%%%%%%%%
% SELF MADE COMMANDS
%%%%%%%%%%%%%%%%%%%%%%%%%%%%%%


\newcommand{\thm}[2]{\begin{Theorem}{#1}{}#2\end{Theorem}}
\newcommand{\cor}[2]{\begin{Corollary}{#1}{}#2\end{Corollary}}
\newcommand{\mlenma}[2]{\begin{Lenma}{#1}{}#2\end{Lenma}}
\newcommand{\mprop}[2]{\begin{Prop}{#1}{}#2\end{Prop}}
\newcommand{\clm}[3]{\begin{claim}{#1}{#2}#3\end{claim}}
\newcommand{\wc}[2]{\begin{wconc}{#1}{}\setlength{\parindent}{1cm}#2\end{wconc}}
\newcommand{\thmcon}[1]{\begin{Theoremcon}{#1}\end{Theoremcon}}
\newcommand{\ex}[2]{\begin{Example}{#1}{}#2\end{Example}}
\newcommand{\dfn}[2]{\begin{Definition}[colbacktitle=red!75!black]{#1}{}#2\end{Definition}}
\newcommand{\dfnc}[2]{\begin{definition}[colbacktitle=red!75!black]{#1}{}#2\end{definition}}
\newcommand{\qs}[2]{\begin{question}{#1}{}#2\end{question}}
\newcommand{\pf}[2]{\begin{myproof}[#1]#2\end{myproof}}
\newcommand{\nt}[1]{\begin{note}#1\end{note}}

\newcommand*\circled[1]{\tikz[baseline=(char.base)]{
		\node[shape=circle,draw,inner sep=1pt] (char) {#1};}}
\newcommand\getcurrentref[1]{%
	\ifnumequal{\value{#1}}{0}
	{??}
	{\the\value{#1}}%
}
\newcommand{\getCurrentSectionNumber}{\getcurrentref{section}}
\newenvironment{myproof}[1][\proofname]{%
	\proof[\bfseries #1: ]%
}{\endproof}
\newcounter{mylabelcounter}

\makeatletter
\newcommand{\setword}[2]{%
	\phantomsection
	#1\def\@currentlabel{\unexpanded{#1}}\label{#2}%
}
\makeatother




\tikzset{
	symbol/.style={
			draw=none,
			every to/.append style={
					edge node={node [sloped, allow upside down, auto=false]{$#1$}}}
		}
}


% deliminators
\DeclarePairedDelimiter{\abs}{\lvert}{\rvert}
\DeclarePairedDelimiter{\norm}{\lVert}{\rVert}

\DeclarePairedDelimiter{\ceil}{\lceil}{\rceil}
\DeclarePairedDelimiter{\floor}{\lfloor}{\rfloor}
\DeclarePairedDelimiter{\round}{\lfloor}{\rceil}

\newsavebox\diffdbox
\newcommand{\slantedromand}{{\mathpalette\makesl{d}}}
\newcommand{\makesl}[2]{%
\begingroup
\sbox{\diffdbox}{$\mathsurround=0pt#1\mathrm{#2}$}%
\pdfsave
\pdfsetmatrix{1 0 0.2 1}%
\rlap{\usebox{\diffdbox}}%
\pdfrestore
\hskip\wd\diffdbox
\endgroup
}
\newcommand{\dd}[1][]{\ensuremath{\mathop{}\!\ifstrempty{#1}{%
\slantedromand\@ifnextchar^{\hspace{0.2ex}}{\hspace{0.1ex}}}%
{\slantedromand\hspace{0.2ex}^{#1}}}}
\ProvideDocumentCommand\dv{o m g}{%
  \ensuremath{%
    \IfValueTF{#3}{%
      \IfNoValueTF{#1}{%
        \frac{\dd #2}{\dd #3}%
      }{%
        \frac{\dd^{#1} #2}{\dd #3^{#1}}%
      }%
    }{%
      \IfNoValueTF{#1}{%
        \frac{\dd}{\dd #2}%
      }{%
        \frac{\dd^{#1}}{\dd #2^{#1}}%
      }%
    }%
  }%
}
\providecommand*{\pdv}[3][]{\frac{\partial^{#1}#2}{\partial#3^{#1}}}
%  - others
\DeclareMathOperator{\Lap}{\mathcal{L}}
\DeclareMathOperator{\Var}{Var} % varience
\DeclareMathOperator{\Cov}{Cov} % covarience
\DeclareMathOperator{\E}{E} % expected

% Since the amsthm package isn't loaded

% I prefer the slanted \leq
\let\oldleq\leq % save them in case they're every wanted
\let\oldgeq\geq
\renewcommand{\leq}{\leqslant}
\renewcommand{\geq}{\geqslant}

% % redefine matrix env to allow for alignment, use r as default
% \renewcommand*\env@matrix[1][r]{\hskip -\arraycolsep
%     \let\@ifnextchar\new@ifnextchar
%     \array{*\c@MaxMatrixCols #1}}


%\usepackage{framed}
%\usepackage{titletoc}
%\usepackage{etoolbox}
%\usepackage{lmodern}


%\patchcmd{\tableofcontents}{\contentsname}{\sffamily\contentsname}{}{}

%\renewenvironment{leftbar}
%{\def\FrameCommand{\hspace{6em}%
%		{\color{myyellow}\vrule width 2pt depth 6pt}\hspace{1em}}%
%	\MakeFramed{\parshape 1 0cm \dimexpr\textwidth-6em\relax\FrameRestore}\vskip2pt%
%}
%{\endMakeFramed}

%\titlecontents{chapter}
%[0em]{\vspace*{2\baselineskip}}
%{\parbox{4.5em}{%
%		\hfill\Huge\sffamily\bfseries\color{myred}\thecontentspage}%
%	\vspace*{-2.3\baselineskip}\leftbar\textsc{\small\chaptername~\thecontentslabel}\\\sffamily}
%{}{\endleftbar}
%\titlecontents{section}
%[8.4em]
%{\sffamily\contentslabel{3em}}{}{}
%{\hspace{0.5em}\nobreak\itshape\color{myred}\contentspage}
%\titlecontents{subsection}
%[8.4em]
%{\sffamily\contentslabel{3em}}{}{}  
%{\hspace{0.5em}\nobreak\itshape\color{myred}\contentspage}



%%%%%%%%%%%%%%%%%%%%%%%%%%%%%%%%%%%%%%%%%%%
% TABLE OF CONTENTS
%%%%%%%%%%%%%%%%%%%%%%%%%%%%%%%%%%%%%%%%%%%

\usepackage{tikz}
\definecolor{doc}{RGB}{0,60,110}
\usepackage{titletoc}
\contentsmargin{0cm}
\titlecontents{chapter}[3.7pc]
{\addvspace{30pt}%
	\begin{tikzpicture}[remember picture, overlay]%
		\draw[fill=doc!60,draw=doc!60] (-7,-.1) rectangle (-0.9,.5);%
		\pgftext[left,x=-3.5cm,y=0.2cm]{\color{white}\Large\sc\bfseries Chapter\ \thecontentslabel};%
	\end{tikzpicture}\color{doc!60}\large\sc\bfseries}%
{}
{}
{\;\titlerule\;\large\sc\bfseries Page \thecontentspage
	\begin{tikzpicture}[remember picture, overlay]
		\draw[fill=doc!60,draw=doc!60] (2pt,0) rectangle (4,0.1pt);
	\end{tikzpicture}}%
\titlecontents{section}[3.7pc]
{\addvspace{2pt}}
{\contentslabel[\thecontentslabel]{2pc}}
{}
{\hfill\small \thecontentspage}
[]
\titlecontents*{subsection}[3.7pc]
{\addvspace{-1pt}\small}
{}
{}
{\ --- \small\thecontentspage}
[ \textbullet\ ][]

\makeatletter
\renewcommand{\tableofcontents}{%
	\chapter*{%
	  \vspace*{-20\p@}%
	  \begin{tikzpicture}[remember picture, overlay]%
		  \pgftext[right,x=15cm,y=0.2cm]{\color{doc!60}\Huge\sc\bfseries \contentsname};%
		  \draw[fill=doc!60,draw=doc!60] (13,-.75) rectangle (20,1);%
		  \clip (13,-.75) rectangle (20,1);
		  \pgftext[right,x=15cm,y=0.2cm]{\color{white}\Huge\sc\bfseries \contentsname};%
	  \end{tikzpicture}}%
	\@starttoc{toc}}
\makeatother


\newcommand{\eps}{\epsilon}
\newcommand{\veps}{\varepsilon}
\newcommand{\ol}{\overline}
\newcommand{\ul}{\underline}
\newcommand{\wt}{\widetilde}
\newcommand{\wh}{\widehat}
\newcommand{\vocab}[1]{\textbf{\color{blue} #1}}
\providecommand{\half}{\frac{1}{2}}
\newcommand{\dang}{\measuredangle} %% Directed angle
\newcommand{\ray}[1]{\overrightarrow{#1}}
\newcommand{\seg}[1]{\overline{#1}}
\newcommand{\arc}[1]{\wideparen{#1}}
\DeclareMathOperator{\cis}{cis}
\DeclareMathOperator*{\lcm}{lcm}
\DeclareMathOperator*{\argmin}{arg min}
\DeclareMathOperator*{\argmax}{arg max}
\newcommand{\cycsum}{\sum_{\mathrm{cyc}}}
\newcommand{\symsum}{\sum_{\mathrm{sym}}}
\newcommand{\cycprod}{\prod_{\mathrm{cyc}}}
\newcommand{\symprod}{\prod_{\mathrm{sym}}}
\newcommand{\Qed}{\begin{flushright}\qed\end{flushright}}
\newcommand{\parinn}{\setlength{\parindent}{1cm}}
\newcommand{\parinf}{\setlength{\parindent}{0cm}}
% \newcommand{\norm}{\|\cdot\|}
\newcommand{\inorm}{\norm_{\infty}}
\newcommand{\opensets}{\{V_{\alpha}\}_{\alpha\in I}}
\newcommand{\oset}{V_{\alpha}}
\newcommand{\opset}[1]{V_{\alpha_{#1}}}
\newcommand{\lub}{\text{lub}}
\newcommand{\del}[2]{\frac{\partial #1}{\partial #2}}
\newcommand{\Del}[3]{\frac{\partial^{#1} #2}{\partial^{#1} #3}}
\newcommand{\deld}[2]{\dfrac{\partial #1}{\partial #2}}
\newcommand{\Deld}[3]{\dfrac{\partial^{#1} #2}{\partial^{#1} #3}}
\newcommand{\lm}{\lambda}
\newcommand{\uin}{\mathbin{\rotatebox[origin=c]{90}{$\in$}}}
\newcommand{\usubset}{\mathbin{\rotatebox[origin=c]{90}{$\subset$}}}
\newcommand{\lt}{\left}
\newcommand{\rt}{\right}
\newcommand{\bs}[1]{\boldsymbol{#1}}
\newcommand{\exs}{\exists}
\newcommand{\st}{\strut}
\newcommand{\dps}[1]{\displaystyle{#1}}

\newcommand{\sol}{\setlength{\parindent}{0cm}\textbf{\textit{Solution:}}\setlength{\parindent}{1cm} }
\newcommand{\solve}[1]{\setlength{\parindent}{0cm}\textbf{\textit{Solution: }}\setlength{\parindent}{1cm}#1 \Qed}

%From M275 "Topology" at SJSU
\newcommand{\id}{\mathrm{id}}
\newcommand{\taking}[1]{\xrightarrow{#1}}
\newcommand{\inv}{^{-1}}

%From M170 "Introduction to Graph Theory" at SJSU
\DeclareMathOperator{\diam}{diam}
\DeclareMathOperator{\ord}{ord}
\newcommand{\defeq}{\overset{\mathrm{def}}{=}}

%From the USAMO .tex files
\newcommand{\ts}{\textsuperscript}
\newcommand{\dg}{^\circ}
\newcommand{\ii}{\item}

% % From Math 55 and Math 145 at Harvard
% \newenvironment{subproof}[1][Proof]{%
% \begin{proof}[#1] \renewcommand{\qedsymbol}{$\blacksquare$}}%
% {\end{proof}}

\newcommand{\liff}{\leftrightarrow}
\newcommand{\lthen}{\rightarrow}
\newcommand{\opname}{\operatorname}
\newcommand{\surjto}{\twoheadrightarrow}
\newcommand{\injto}{\hookrightarrow}
\newcommand{\On}{\mathrm{On}} % ordinals
\DeclareMathOperator{\img}{im} % Image
\DeclareMathOperator{\Img}{Im} % Image
\DeclareMathOperator{\coker}{coker} % Cokernel
\DeclareMathOperator{\Coker}{Coker} % Cokernel
\DeclareMathOperator{\Ker}{Ker} % Kernel
\DeclareMathOperator{\rank}{rank}
\DeclareMathOperator{\Spec}{Spec} % spectrum
\DeclareMathOperator{\Tr}{Tr} % trace
\DeclareMathOperator{\pr}{pr} % projection
\DeclareMathOperator{\ext}{ext} % extension
\DeclareMathOperator{\pred}{pred} % predecessor
\DeclareMathOperator{\dom}{dom} % domain
\DeclareMathOperator{\ran}{ran} % range
\DeclareMathOperator{\Hom}{Hom} % homomorphism
\DeclareMathOperator{\Mor}{Mor} % morphisms
\DeclareMathOperator{\End}{End} % endomorphism

% Things Lie
\newcommand{\kb}{\mathfrak b}
\newcommand{\kg}{\mathfrak g}
\newcommand{\kh}{\mathfrak h}
\newcommand{\kn}{\mathfrak n}
\newcommand{\ku}{\mathfrak u}
\newcommand{\kz}{\mathfrak z}
\DeclareMathOperator{\Ext}{Ext} % Ext functor
\DeclareMathOperator{\Tor}{Tor} % Tor functor
\newcommand{\gl}{\opname{\mathfrak{gl}}} % frak gl group
\renewcommand{\sl}{\opname{\mathfrak{sl}}} % frak sl group chktex 6

% More script letters etc.
\newcommand{\SA}{\mathcal A}
\newcommand{\SB}{\mathcal B}
\newcommand{\SC}{\mathcal C}
\newcommand{\SF}{\mathcal F}
\newcommand{\SG}{\mathcal G}
\newcommand{\SH}{\mathcal H}
\newcommand{\OO}{\mathcal O}

\newcommand{\SCA}{\mathscr A}
\newcommand{\SCB}{\mathscr B}
\newcommand{\SCC}{\mathscr C}
\newcommand{\SCD}{\mathscr D}
\newcommand{\SCE}{\mathscr E}
\newcommand{\SCF}{\mathscr F}
\newcommand{\SCG}{\mathscr G}
\newcommand{\SCH}{\mathscr H}

% Mathfrak primes
\newcommand{\km}{\mathfrak m}
\newcommand{\kp}{\mathfrak p}
\newcommand{\kq}{\mathfrak q}

% number sets
\newcommand{\RR}[1][]{\ensuremath{\ifstrempty{#1}{\mathbb{R}}{\mathbb{R}^{#1}}}}
\newcommand{\NN}[1][]{\ensuremath{\ifstrempty{#1}{\mathbb{N}}{\mathbb{N}^{#1}}}}
\newcommand{\ZZ}[1][]{\ensuremath{\ifstrempty{#1}{\mathbb{Z}}{\mathbb{Z}^{#1}}}}
\newcommand{\QQ}[1][]{\ensuremath{\ifstrempty{#1}{\mathbb{Q}}{\mathbb{Q}^{#1}}}}
\newcommand{\CC}[1][]{\ensuremath{\ifstrempty{#1}{\mathbb{C}}{\mathbb{C}^{#1}}}}
\newcommand{\PP}[1][]{\ensuremath{\ifstrempty{#1}{\mathbb{P}}{\mathbb{P}^{#1}}}}
\newcommand{\HH}[1][]{\ensuremath{\ifstrempty{#1}{\mathbb{H}}{\mathbb{H}^{#1}}}}
\newcommand{\FF}[1][]{\ensuremath{\ifstrempty{#1}{\mathbb{F}}{\mathbb{F}^{#1}}}}
% expected value
\newcommand{\EE}{\ensuremath{\mathbb{E}}}
\newcommand{\charin}{\text{ char }}
\DeclareMathOperator{\sign}{sign}
\DeclareMathOperator{\Aut}{Aut}
\DeclareMathOperator{\Inn}{Inn}
\DeclareMathOperator{\Syl}{Syl}
\DeclareMathOperator{\Gal}{Gal}
\DeclareMathOperator{\GL}{GL} % General linear group
\DeclareMathOperator{\SL}{SL} % Special linear group

%---------------------------------------
% BlackBoard Math Fonts :-
%---------------------------------------

%Captital Letters
\newcommand{\bbA}{\mathbb{A}}	\newcommand{\bbB}{\mathbb{B}}
\newcommand{\bbC}{\mathbb{C}}	\newcommand{\bbD}{\mathbb{D}}
\newcommand{\bbE}{\mathbb{E}}	\newcommand{\bbF}{\mathbb{F}}
\newcommand{\bbG}{\mathbb{G}}	\newcommand{\bbH}{\mathbb{H}}
\newcommand{\bbI}{\mathbb{I}}	\newcommand{\bbJ}{\mathbb{J}}
\newcommand{\bbK}{\mathbb{K}}	\newcommand{\bbL}{\mathbb{L}}
\newcommand{\bbM}{\mathbb{M}}	\newcommand{\bbN}{\mathbb{N}}
\newcommand{\bbO}{\mathbb{O}}	\newcommand{\bbP}{\mathbb{P}}
\newcommand{\bbQ}{\mathbb{Q}}	\newcommand{\bbR}{\mathbb{R}}
\newcommand{\bbS}{\mathbb{S}}	\newcommand{\bbT}{\mathbb{T}}
\newcommand{\bbU}{\mathbb{U}}	\newcommand{\bbV}{\mathbb{V}}
\newcommand{\bbW}{\mathbb{W}}	\newcommand{\bbX}{\mathbb{X}}
\newcommand{\bbY}{\mathbb{Y}}	\newcommand{\bbZ}{\mathbb{Z}}

%---------------------------------------
% MathCal Fonts :-
%---------------------------------------

%Captital Letters
\newcommand{\mcA}{\mathcal{A}}	\newcommand{\mcB}{\mathcal{B}}
\newcommand{\mcC}{\mathcal{C}}	\newcommand{\mcD}{\mathcal{D}}
\newcommand{\mcE}{\mathcal{E}}	\newcommand{\mcF}{\mathcal{F}}
\newcommand{\mcG}{\mathcal{G}}	\newcommand{\mcH}{\mathcal{H}}
\newcommand{\mcI}{\mathcal{I}}	\newcommand{\mcJ}{\mathcal{J}}
\newcommand{\mcK}{\mathcal{K}}	\newcommand{\mcL}{\mathcal{L}}
\newcommand{\mcM}{\mathcal{M}}	\newcommand{\mcN}{\mathcal{N}}
\newcommand{\mcO}{\mathcal{O}}	\newcommand{\mcP}{\mathcal{P}}
\newcommand{\mcQ}{\mathcal{Q}}	\newcommand{\mcR}{\mathcal{R}}
\newcommand{\mcS}{\mathcal{S}}	\newcommand{\mcT}{\mathcal{T}}
\newcommand{\mcU}{\mathcal{U}}	\newcommand{\mcV}{\mathcal{V}}
\newcommand{\mcW}{\mathcal{W}}	\newcommand{\mcX}{\mathcal{X}}
\newcommand{\mcY}{\mathcal{Y}}	\newcommand{\mcZ}{\mathcal{Z}}



%---------------------------------------
% Bold Math Fonts :-
%---------------------------------------

%Captital Letters
\newcommand{\bmA}{\boldsymbol{A}}	\newcommand{\bmB}{\boldsymbol{B}}
\newcommand{\bmC}{\boldsymbol{C}}	\newcommand{\bmD}{\boldsymbol{D}}
\newcommand{\bmE}{\boldsymbol{E}}	\newcommand{\bmF}{\boldsymbol{F}}
\newcommand{\bmG}{\boldsymbol{G}}	\newcommand{\bmH}{\boldsymbol{H}}
\newcommand{\bmI}{\boldsymbol{I}}	\newcommand{\bmJ}{\boldsymbol{J}}
\newcommand{\bmK}{\boldsymbol{K}}	\newcommand{\bmL}{\boldsymbol{L}}
\newcommand{\bmM}{\boldsymbol{M}}	\newcommand{\bmN}{\boldsymbol{N}}
\newcommand{\bmO}{\boldsymbol{O}}	\newcommand{\bmP}{\boldsymbol{P}}
\newcommand{\bmQ}{\boldsymbol{Q}}	\newcommand{\bmR}{\boldsymbol{R}}
\newcommand{\bmS}{\boldsymbol{S}}	\newcommand{\bmT}{\boldsymbol{T}}
\newcommand{\bmU}{\boldsymbol{U}}	\newcommand{\bmV}{\boldsymbol{V}}
\newcommand{\bmW}{\boldsymbol{W}}	\newcommand{\bmX}{\boldsymbol{X}}
\newcommand{\bmY}{\boldsymbol{Y}}	\newcommand{\bmZ}{\boldsymbol{Z}}
%Small Letters
\newcommand{\bma}{\boldsymbol{a}}	\newcommand{\bmb}{\boldsymbol{b}}
\newcommand{\bmc}{\boldsymbol{c}}	\newcommand{\bmd}{\boldsymbol{d}}
\newcommand{\bme}{\boldsymbol{e}}	\newcommand{\bmf}{\boldsymbol{f}}
\newcommand{\bmg}{\boldsymbol{g}}	\newcommand{\bmh}{\boldsymbol{h}}
\newcommand{\bmi}{\boldsymbol{i}}	\newcommand{\bmj}{\boldsymbol{j}}
\newcommand{\bmk}{\boldsymbol{k}}	\newcommand{\bml}{\boldsymbol{l}}
\newcommand{\bmm}{\boldsymbol{m}}	\newcommand{\bmn}{\boldsymbol{n}}
\newcommand{\bmo}{\boldsymbol{o}}	\newcommand{\bmp}{\boldsymbol{p}}
\newcommand{\bmq}{\boldsymbol{q}}	\newcommand{\bmr}{\boldsymbol{r}}
\newcommand{\bms}{\boldsymbol{s}}	\newcommand{\bmt}{\boldsymbol{t}}
\newcommand{\bmu}{\boldsymbol{u}}	\newcommand{\bmv}{\boldsymbol{v}}
\newcommand{\bmw}{\boldsymbol{w}}	\newcommand{\bmx}{\boldsymbol{x}}
\newcommand{\bmy}{\boldsymbol{y}}	\newcommand{\bmz}{\boldsymbol{z}}

%---------------------------------------
% Scr Math Fonts :-
%---------------------------------------

\newcommand{\sA}{{\mathscr{A}}}   \newcommand{\sB}{{\mathscr{B}}}
\newcommand{\sC}{{\mathscr{C}}}   \newcommand{\sD}{{\mathscr{D}}}
\newcommand{\sE}{{\mathscr{E}}}   \newcommand{\sF}{{\mathscr{F}}}
\newcommand{\sG}{{\mathscr{G}}}   \newcommand{\sH}{{\mathscr{H}}}
\newcommand{\sI}{{\mathscr{I}}}   \newcommand{\sJ}{{\mathscr{J}}}
\newcommand{\sK}{{\mathscr{K}}}   \newcommand{\sL}{{\mathscr{L}}}
\newcommand{\sM}{{\mathscr{M}}}   \newcommand{\sN}{{\mathscr{N}}}
\newcommand{\sO}{{\mathscr{O}}}   \newcommand{\sP}{{\mathscr{P}}}
\newcommand{\sQ}{{\mathscr{Q}}}   \newcommand{\sR}{{\mathscr{R}}}
\newcommand{\sS}{{\mathscr{S}}}   \newcommand{\sT}{{\mathscr{T}}}
\newcommand{\sU}{{\mathscr{U}}}   \newcommand{\sV}{{\mathscr{V}}}
\newcommand{\sW}{{\mathscr{W}}}   \newcommand{\sX}{{\mathscr{X}}}
\newcommand{\sY}{{\mathscr{Y}}}   \newcommand{\sZ}{{\mathscr{Z}}}


%---------------------------------------
% Math Fraktur Font
%---------------------------------------

%Captital Letters
\newcommand{\mfA}{\mathfrak{A}}	\newcommand{\mfB}{\mathfrak{B}}
\newcommand{\mfC}{\mathfrak{C}}	\newcommand{\mfD}{\mathfrak{D}}
\newcommand{\mfE}{\mathfrak{E}}	\newcommand{\mfF}{\mathfrak{F}}
\newcommand{\mfG}{\mathfrak{G}}	\newcommand{\mfH}{\mathfrak{H}}
\newcommand{\mfI}{\mathfrak{I}}	\newcommand{\mfJ}{\mathfrak{J}}
\newcommand{\mfK}{\mathfrak{K}}	\newcommand{\mfL}{\mathfrak{L}}
\newcommand{\mfM}{\mathfrak{M}}	\newcommand{\mfN}{\mathfrak{N}}
\newcommand{\mfO}{\mathfrak{O}}	\newcommand{\mfP}{\mathfrak{P}}
\newcommand{\mfQ}{\mathfrak{Q}}	\newcommand{\mfR}{\mathfrak{R}}
\newcommand{\mfS}{\mathfrak{S}}	\newcommand{\mfT}{\mathfrak{T}}
\newcommand{\mfU}{\mathfrak{U}}	\newcommand{\mfV}{\mathfrak{V}}
\newcommand{\mfW}{\mathfrak{W}}	\newcommand{\mfX}{\mathfrak{X}}
\newcommand{\mfY}{\mathfrak{Y}}	\newcommand{\mfZ}{\mathfrak{Z}}
%Small Letters
\newcommand{\mfa}{\mathfrak{a}}	\newcommand{\mfb}{\mathfrak{b}}
\newcommand{\mfc}{\mathfrak{c}}	\newcommand{\mfd}{\mathfrak{d}}
\newcommand{\mfe}{\mathfrak{e}}	\newcommand{\mff}{\mathfrak{f}}
\newcommand{\mfg}{\mathfrak{g}}	\newcommand{\mfh}{\mathfrak{h}}
\newcommand{\mfi}{\mathfrak{i}}	\newcommand{\mfj}{\mathfrak{j}}
\newcommand{\mfk}{\mathfrak{k}}	\newcommand{\mfl}{\mathfrak{l}}
\newcommand{\mfm}{\mathfrak{m}}	\newcommand{\mfn}{\mathfrak{n}}
\newcommand{\mfo}{\mathfrak{o}}	\newcommand{\mfp}{\mathfrak{p}}
\newcommand{\mfq}{\mathfrak{q}}	\newcommand{\mfr}{\mathfrak{r}}
\newcommand{\mfs}{\mathfrak{s}}	\newcommand{\mft}{\mathfrak{t}}
\newcommand{\mfu}{\mathfrak{u}}	\newcommand{\mfv}{\mathfrak{v}}
\newcommand{\mfw}{\mathfrak{w}}	\newcommand{\mfx}{\mathfrak{x}}
\newcommand{\mfy}{\mathfrak{y}}	\newcommand{\mfz}{\mathfrak{z}}



\title{\Huge{Travaux dirigées}\\ Polynômes}
\author{\huge{A.Belcaid}}
\date{\today}

\begin{document}

\maketitle
% \newpage% or \cleardoublepage
% \pdfbookmark[<level>]{<title>}{<dest>}
\pdfbookmark[section]{\contentsname}{toc}
\tableofcontents
\pagebreak
\newcommand{\Rr}{\mathbb{R}};
\chapter{}

%{{{ Exercice 1
\section{Exercice 1} % (fold)
\label{sec:Exercice 1}
\qs{Division Euclidienne}{
Pour les trois cas listés, calculer la division euclidienne de $P$ par
$Q$.

\begin{enumerate}
  \item $P=X^4 + 5X^3 + 12X^2 + 19X - 7$ et $Q=X^2 + 3X-1$.
  \item $P=X^4-4X^3-9X^2+27X+38$ et $Q=X^2-X-7$.
  \item $P=X^5-X^2+2$ et $Q = X^2 +1$.
\end{enumerate}
}

\begin{myproof}

\begin{enumerate}
  \item Pour le premier cas on obtient:

    $$
    \polylongdiv[style=D]{X^4 + 5X^3 + 12X^2 + 19X - 7}{ X^2 + 3X-1}
    $$

  \item Pour le deuxième cas:
    $$
    \polylongdiv[style=D]{X^4-4X^3-9X^2+27X+38}{X^2-X-7}
    $$

  \item Finalement pour le troisième cas:

    $$
    \polylongdiv[style=D]{X^5-X^2+2}{X^2 +1}
    $$
\end{enumerate}

\end{myproof}

% section section name (end)
%}}}

%{{{ Exercice 2
\section{Exercice 2} % (fold)
\label{sec:Execice 2}
\qs{Expression du reste}
{

Soit $P\in \Rr[X]$, $a,b\in \Rr$, $a\neq b$. Sachant que le reste de la
division euclidienne de $P$ par $(X-a)$ vaut $1$ et que celui de $P$ par
$(X-b)$ vaut $-1$.

\begin{enumerate}
  \item Évaluer l'image $P(a)$ et $P(b)$?
  \item On note $R\in \Rr[X]$ le reste de la division euclidienne de $P$ par
    $(X-a)(X-b)$. Quel  sera le degré de $R$?
  \item En déduire l'expression de $R$.
\end{enumerate}
}

\begin{myproof}
  \begin{enumerate}
    \item 
  On sait que $$P(X)=(X-a)Q_1(x) + 1$$
  Ceci implique que 
  $$
  P(a) = 1
  $$

  De meme on obtient que 
  $$
  P(b) = -1
  $$
\item Puisque le cardinal de $(X-a)(x-b)$ est $2$ le reste $R$ doit être au plus de cardinal $1$.

  $$
  P(X) = (X-a)(X-b)Q(x) + \alpha X  + \beta
  $$

  On evalue cette expression en $a$ et en $b$, on trouve le systeme:

  $$
  \begin{cases}
    \alpha a + \beta &= 1\\
    \alpha b + \beta &= -1\\
  \end{cases}
  $$
  La résolution de système nous donne:
  $$
  \alpha = \dfrac{2}{a-b} \text{ et } \beta = \dfrac{-a-b}{a-b}
  $$
  
  Le reste recherché est donc
  $$
  \dfrac{2}{a-b}X + \dfrac{-a-b}{a-b}
  $$


  \end{enumerate}
\end{myproof}

% section section name (end)
%}}}

%{{{ Exercice 3
\newpage
\section{Exercice 3} % (fold)
\label{sec:Exercice 3}

\qs{}
{
On se propose de déterminer l'ensemble

\begin{equation*}
  E = \left\{P\in \Rr[X]\;\; P(X^2)=(X^3+1)P(X)\right\} 
\end{equation*}

\begin{enumerate}
  \item Démontrer que le polynôme nul ainsi\\  que le polynôme $X^3-1$ sont
    dans $E$.\\[4pt]
  \item Soit $P\in E$, non nul.
    \begin{enumerate}
      \small
    \item Démontrer que $P(1)=0$ puis que $P^{'}(0)=P^{''}(0)=0$.
    \item En effectuant la division euclidienne de $P$ par
      $X^3-1$, démontrer qu'il existe $\lambda\in \Rr$ tel que
      $$
      P(X) = \lambda (X^3-1)
      $$
    \end{enumerate}
  \item En déduire l'ensemble $E$.
\end{enumerate}
}
\begin{myproof}
  \begin{enumerate}
    \item On demontre que le polynôme nul et $X^3-1$ sont dans $E$..
      \begin{itemize}
        \item On a 
          $$ 0(X^2) = 0 = (X^3+1)0(X) $$ donc le polynome nul est dans $E$.
        \item Soit $P_1 = X^3 - 1$, on as:
          $$
          P_1(X^2) = X^6 - 1
          $$
          et 
          $$
          (X^3+1)P_1(X) = (X^3+1)(X^3-1) = X^6 - 1
          $$
          Ainsi on as aussi que $P_1$ est dans $E$.
      \end{itemize}
    \item On considere alors $P\in E$. selon la definition on aura:
      $$
      P(1) = (1 + 1) P(1)
      $$
      ce qui implique que 
      $$
      P(1) = 0
      $$
      On dérive la relation de l'équation on obtient:

      $$
      2P^{'}(X^2) = 3X^2P(X) + (X^3+1)P^{'}(X)
      $$
      ce qui prouve que $P^{'}(0)= 0$.

      On dérive on deuxième fois:

      $$
      4P^{''}(X^2) = 6XP(X) + 3X^2P^{''}(X) + 3X^2P^{'}(X) + (X^3+1)P^{''}(X)
      $$

      on injecte $0$ dans cette équation on trouve que:
      $$
      P^{''}(0) = 0
      $$
    \item Une analyse de dégrée de $P$ selon l'équation vérifiée implique que :
      $$
      2deg(P) = 3  + deg(P)
      $$
      Ce qui implique que 
      $$
      deg(P) = 3
      $$
      Ainsi l'expression de la division euclidienne d'un tel polynome par $X^3-1$ donnera :

      $$
      P = \lambda (X^3 - 1) + \beta\quad \lambda,\beta \in \Rr
      $$

      Sachant que $P(1) = 0$ on trouve forcement que $\beta =0$.
      Ainsi

      $$
      P = \lambda (X^3-1)
      $$

  \end{enumerate}
\end{myproof}
% section section name (end)
%}}}

%{{{ Exercice 4
\section{Exercice 4} % (fold)
\label{sec:Exercice 4}
\qs{Calcul PGCD}{
Pour chaque cas, déterminer le \textbf{PGCD} entre $P$ et $Q$.

\begin{enumerate}
  \item $P=X^4-3X^3 + X^2 + 4$ et $Q=X^3-3X^2+3X-2$.
  \item $P=X^5-X^4+2X^3-2X^2+2X-1$ et $Q=X^5-X^4+2X^2-2X+1$.
  \item $P=X^n -1$ et $Q=(X-1)^n$.
\end{enumerate}
}

\begin{myproof}
  pour chaque cas on as:
  \begin{enumerate}
    \item 
      $$
      \polylonggcd{X^4-3X^3 + X^2 + 4}{X^3-3X^2+3X-2}
      $$
      Ce qui donne que le PGCD est :

      $$
      X - 2
      $$
    \item 
      $$
      \polylonggcd{X^5-X^4+2X^3-2X^2+2X-1}{X^5-X^4+2X^2-2X+1}
      $$
      Ainsi le PGCD est
      $$
      X^2 - X + 1
      $$
    \item 

      On as 
      $$
      (X^n - 1) = (X - 1)\sum_{i=0}^{n-1}X^i
      $$
      Ainsi le seul facteur commun entre ces deux polynomes est:

      $$
      (X - 1)
      $$
  \end{enumerate}

\end{myproof}

% section section name (end)
%}}}

%{{{ Exercice 5
\qs{Formule de Bezout}{
Trouver deux polynômes $U$ et $V$ de $\Rr[X]$ tel que 
$$
AU + BV = 1
$$

où $A = X^7-X-1$ et $B=X^5-1$.

}
\begin{myproof}
  On utilise l'algorithme d'Euclide. On a

  $$
  \polylonggcd{X^7-X -1}{X^5-1}
  $$
  On remonte ensuite les calculs. On va partir plutot de 
  $$
  11 = -25(X^2 - X - 1) + (5X-7)(5X+2)
  $$
  Pour eviter de trainer des fractions. On trouve successivement:

  \begin{eqnarray*}
    11 &= & -25(X^2 - X - 1) + (5X - 7)\left((X^5-1) - (X^2-X-1)(X^3+X^2+2X+3)\right)\\[4pt]
       & =& \left(-5X^4 + 2X^3 - 3X^2-X -4\right)\left(X^2-X-1\right) + \left(5X -7\right)\left(X^5-1\right)\\[4pt]
       &=&(-5X^4 + 2X^3 - 3X^2 -X -4)(X^7-X-1) + (5X^6-2X^5+3X^4+X^3+4X^2+5X-7)(X^5-1)
  \end{eqnarray*}
  Finalement il suffit de diviser par $11$ pour trouver $U$ et $V$.
\end{myproof}
%}}}


%{{{ Exercice 6
\section{Exercice 6} % (fold)
\label{sec:Exercice 6}

% section section name (end)
\qs{}
{

  Soient $P$ et $Q$ des polynômes de $\mathbb{C}[X]$ non constants. Montrer  que
l'équivalence entre:

\begin{enumerate}
  \item $P$ et $Q$ ont un facteur commun.
  \item il existe $A,B\in \mathbb{C}[X]$, $A\neq0$, $B\neq 0$, tel que
    $$
    AP = BQ
    $$
  et $\text{deg}(A)< \text{deg}(Q),\quad \text{deg}(B)<\text{deg}(P)$ 
\end{enumerate}
}

\begin{myproof}
  On commence par la premiere implication
  \begin{enumerate}
    \item On suppose que $P$ et $Q$ ont un facteur commun $D$. On factorise alors 
      $$
      \left\{\begin{array}{lll}
          P&=&DB\\[4pt]
          Q&=&DA
        \end{array}
      \right.
      $$
      Ce qui implique que:
      $$
      AP = ADB = BDA = BQ
      $$
    \item Pour la reciproque, On suppose que $P$ et $Q$ sont premiers entre eux
      $$
      P \wedge Q = 1 \quad \text{ et } AP = BQ
      $$
      alors 
      $$
      P | BQ
      $$
      et par le theoreme de Gauss on aura
      $$
      P | B
      $$
      Ce qui est absurde.
  \end{enumerate}
\end{myproof}
%}}}

%{{{ Exercice 7
\section{Exercice 7} % (fold)
\label{sec:Exercice 7}

% section section name (end)
\qs{Racines}
{
Quel est pour $n\geq 1$ l'ordre de multiplicité de $2$ du polynôme:

$$
P_n(X)= n X^{n+2}-(4n+1)X^{n+1}+4(n+1)X^n-4X^{n-1}
$$
}
\begin{myproof}
  On verifie d'abord que $P_n(2)= 0$ et donc $2$ est une racine de $P_n$.\\

  On calcule maintenant la derivee 
  $$
  P^{'}_n(X) = n(n+2)X^{n+1} - (4n+1)(n+1)X^n + 4n(n+1)X^{n-1} - 4(n-1)X^{n-2}
  $$
  si $n=1$, le dernier terme est interprete comme le polynome nul. En particulier, on a:

  \begin{eqnarray}
    P^{'}_n(2)&=& n(n+2)2^{n+1} - (4n+1)(n+1)2^n + 4n(n+1)2^{n-1} -4(n-1)2^{n-2}\\[4pt]
              &=& 2^{n-2}\left(8n(n+2) -4(4n+1)(n+1) + 8n(n+1) - 4(n-1)\right)\\[4pt]
              &=&0
  \end{eqnarray}
  On derive une fois encore
  $$
  P^{''}_n(X) = n(n+1)(n+2)X^n -(4n+1)n(n+1)X^{n-1} + 4n(n+1)(n-1)X^{n-2} 4(n-1)(n-2)X^{n-3}
  $$
  d'ou l'on tire 
  $$
  P^{''}_n(2) = 2^{n-3}\left(8n(n+1)(n+2) - 4(4n+1)n(n+1)+ 8n(n+1)(n-1) -4(n-1)(n-2)\right)
  $$
  $$
P^{''}_n(2) = 2^n(2n-1)
  $$
  Puisque $2n-1$ ne s'anulle pas quand n est un entier, on as 

  $$
  P_n(2) = P^{'}_n(2) = 0
  $$
  et 
  $$
  P^{''}_n(2) \ne 0.
  $$

  Aini $2$ est une racine de multiplicite $2$.
\end{myproof}
%}}}

%{{{ 
\newpage
\section{Exercie 8} % (fold)
\label{sec:Exercie 8}

% section section name (end)
\qs{}
{
  Soit $P(X)=a_nX^n+\ldots+a_0$ un polynôme dans $\mathbb{Z}[X]$. On suppose aussi
que $P$ admet une racine rationnelle $r=\frac{p}{q}$ tel que $p\wedge q
=1$.
\begin{enumerate}
  \item Développer que la forme $P(r)=0$.
  \item Démontrer que $p\;|\; a_0$.
  \item Prouver que $q\;|\; a_n$
  \item En déduire que $P=X^5 - X^2+1$ n'admet pas de racines dans
    $\mathbb{Q}$.
\end{enumerate}
}
\begin{myproof}
  \begin{enumerate}
    \item 
      \begin{eqnarray*}
        P\left(\dfrac{p}{q}\right) &=& 0\\[4pt]
        a_np^n + a_{n-1}p^{n-1}q+\ldots+a_1pq^{n-1} + a_0q^n&=& 0
      \end{eqnarray*}
      On commence par isoler $a_0q^n$ et on trouve que:
      $$
      p\left(a_n p^{n-1} + a_{n-1}p^{n-2}q + \ldots + a_1q^{n-1}\right) = -a_0q^n
      $$
      Puisque $p \wedge q = 1 $, on aura $p | a_0q^n$.\\

      On isole aussi $a_np^n$, on trouve que:
      $$
      q\left(a_{n-1}p^{n-1} + \ldots+ a_0q^{n-1}\right) = -a_np^n
      $$
      De meme analyse, on conclut que 
      $$
      q | a_np^n \quad \text { puisque } p \wedge q = 1
      $$
      Par consequent, si le polynome $X^5 - X^2 +1$ admet une racine rationnelle $p/q$, alors $p|1$ et $q|1$. Ainsi
      $$
      \vert p \vert = 1 \quad \text{ et } \vert q \vert = 1
      $$
      Ainsi les seuls racines possibles sont $-1$ et $1$. Or, elles ne sont pas des racines de $P$ 
      Ainsi $P$ n'admet pas de racines rationnelle.
  \end{enumerate}
\end{myproof}
%}}}

%{{{Exercice 9
\section{Exercice 9}
\newcommand{\Cc}{\mathbb{C}}
% \newcommand{\Rr}{\mathbb{R}}
\qs{}
{
\begin{enumerate}
  \item Le polynôme $P(X) = X^4 + X^2 + 1$ est il irréductible  dans $\Rr[X]$?
dans $\Cc[X]$?
\item La relation $P\mathcal{R} Q \iff P\; \text{divise}\; Q$  est-elle une
relation d'ordre?
\end{enumerate}
}
\begin{myproof}
  On sait que les seuls polynômes \textbf{irréductibles} dans $\Rr[X]$
  \begin{itemize}
    \item Les polynômes de degré 1.
    \item Les polynômes de degré 2 avec un déterminant négatif.
  \end{itemize}
  et dans $\Cc[X]$ les polynômes de degré $1$.\\

  Ainsi le polynôme $X^4 + X^2 + 1$ ni irréductible ni dans $\Rr[X]$ ni dans $\Cc[X]$.


\end{myproof}
%}}}

%{{{
\section{Exercice 10}
\qs{}{

Pour chaque polynôme, donner la décomposition en facteurs irréductibles dans
$\Rr[X]$

\begin{enumerate}
  \item $P_1(X) = X^4 + 1$
  \item $P_2(X) = X^8 - 1$
  \item $P_3(X) = \left(X^2 - X+1\right)^2+1$
\end{enumerate}

}
\begin{myproof}
  Voici la décomposition  de chaque polynôme:

  \begin{enumerate}
    \item Pour le premier exemple, on résout  l'équation classique
      $$
      X^4 = -1
      $$
      Ça nous donne que 

      $$
      X^4 + 1 = (X-e^{\frac{i\pi}{4}})(X-e^{\frac{3i\pi}{4}})(X-e^{\frac{5i\pi}{4}})(X-e^{\frac{7i\pi}{4}})
      $$

      on regroupe les termes \textbf{conjugués} on trouve que

      $$
      X^4 + 1 =(X-e^{\frac{i\pi}{4}})(X-e^{\frac{7i\pi}{4}}) (X-e^{\frac{5i\pi}{4}})(X-e^{\frac{5i\pi}{4}})
      $$ $$ X^4 + 1 = (X^2 - \sqrt{2}X + 1)(X^2 + \sqrt{2}X +1)
      $$
      Les deux sont de degré deux et de déterminant négatifs. Ainsi ils sont irréductibles dans $\Rr[X]$.
    \item Pour le deuxième exemple on trouve que
      \begin{eqnarray*}
        X^8 - 1 &=& (X^4 - 1)(X^4 + 1 )\\
         &=& (X^2 - 1)(X^2 + 1)(X^4 + 1)\\
         &=& (X-1)(X+1)(X^2+1)(X^4 +1)
    \end{eqnarray*}
    Pour $X^4 + 1$ on utilise le même processus que la question $1$ et on trouve:
    $$
    X^8 - 1 = (X-1)(X+1)(X^2+1)(X^2 - \sqrt{2}X + 1)(X^2 + \sqrt{2}X +1)
    $$
  \item Pour le polynôme final
    $$
    (X^2 - X + 1)^2 + 1 = (X^2 - X + 1 -i)(X^2 - X + 1 + i)
    $$
    On factorise chaque polynôme de degré $2$ dans $\Cc$, on trouve que:

    $$
    (X^2 - X + 1)^2 + 1 = (X+i)(X-1-i)(X-i)(X-1+i)
    $$
    On regroupe les termes conjugués:

    $$
    (X^2 - X + 1)^2 + 1 = (X^2 + 1)(X^2-2X + 2)
    $$
  \end{enumerate}
\end{myproof}
%}}}

%{{{ Exercice 11
\section{Exercice 11} % (fold)
\qs{}{


Soit $P$ le polynôme définit par:

$$
P(X) = 2X^4 + X^2 -3
$$
\begin{enumerate}
  \item 
Décomposer $P$ en facteurs irréductibles dans $\Rr[X]$.
\end{enumerate}

}
\begin{myproof}
  Il s'agit d'un  polynôme spécial \textbf{bicarré} qui s'écrit sous la forme:

  $$
  P(X) = Q(X^2) \quad \text{ ou } Q(X) = X^2 + X - 3
  $$

  On commence alors par la décomposition de $Q$

  $$
  Q(X) = 2(X-1)(x + \frac{3}{2})
  $$

  On en déduit alors que 
  $$
  P(X) = 2(X^2- 1)(X^2 + \frac{3}{2}) = 2(X-1)(X+1)(X^2 + \frac{3}{2})
  $$
\end{myproof}
% section section name (end)
%}}}

%{{{ Exercice 12
\section{Exercice 12} % (fold)

\qs{}
{

Soit le polynôme $P(X) = X^4-6X^3+9X^2+9$.

\begin{enumerate}
  \item Décomposer $X^4-6X^3+9X^2$ en produit de facteurs irréductibles dans
    $\Rr[X]$.
  \item En déduire une décomposition de $P$ dans $\Rr[X]$.

  \item Même question pour $\Cc[X]$.
\end{enumerate}

}
\begin{myproof}
  \begin{enumerate}
\item
  On écrit simplement
  $$
  X^4 - 6X^3 + 9X^2 + 9 = X^2(X^2-6X+9) = X^2 (X-3)^2
  $$

\item L'astuce pour obtenir une identité remarquable est de considèrera que $9 = -(3i)^2$. Avec ceci on trouve que 

  \begin{eqnarray*}
    X^4-6X^3+9X^2 + 9 &=& \left(X(X-3)\right)^2 - (3i)^2\\
                      &=& (X(X-3) - 3i)(X(X-3) + 3i)\\
                      &=& (X^2 - 3X -3i)(X^2-3X+3i)
  \end{eqnarray*}
  Pour finalisme la décomposition il suffit de calculer le discriminant de chaque polynôme pour obtenir les racines.
\item  Dans $\Cc[X]$ la décomposition s'écrit comme
  $$
  P(X) = (X-\alpha_1)(X-\alpha_2)(X - \beta_1)(X- \beta_2)
  $$
  Ou $\alpha_1, \alpha_2, \beta_1$ et $\beta_2$ sont les racines calculées dans la question précédente.
\end{enumerate}
\end{myproof}
% section section name (end)

%}}}
%{{{ Exercice 13
\qs{Factorisation simultanée}{

On considère les deux polynômes suivants:

\begin{itemize}
  \item $P(X) = X^3 - 9X^2 + 26X -24$
  \item $Q(X) = X^3 - 7X^2 + 7X + 15$.
\end{itemize}
\begin{enumerate}
  \item Sachant que $P$ et $Q$ admettent une racine \textbf{commune} $a$,
    Quelle est la relation entre $(X-a)$ et $\text{pgcd}(P,Q)$?
  \item En appliquant l'algorithme d'Euclide, montrer  que le \textbf{pgcd} de $P$
    et $Q$ est $X-3$?
  

  \item Calculer le polynôme $P_1$ tel que
    $$ P = (X-3)P_1$$

  \item Même question pour $Q_1$ tel que:

    $$Q = (X-3)Q_1$$.

  \item En déduire une décomposition en facteurs\\ irréductibles dans
    $\Rr[X]$ de $P$ et $Q$.
\end{enumerate}
}
\begin{myproof}

  Si $a$ possède une racine commune  $a$, alors le polynome $(X-1)$ divise le \textbf{PGCD}(P, Q). Ainsi on calcul le pgcd de ces deux polynômes

$$
\polylonggcd{X^3 - 9X^2 + 26X  - 24}{X^3 - 7X^2 + 7X + 15}
$$

On trouve que le pgcd est $(X-3)$ Ainsi la valeur de $a=3$.\\

Maintenant qu'on as une racine commune on peut diminuer le degré de $P$ et de $Q$ en réalisant la division euclidienne.

$$
\polylongdiv[style=D]{X^3 - 9X^2 + 26 X -24}{X-3}
$$
Ainsi on as:

$$
P = (X-3)(X^2-6X+8) = (X-3)(X-2)(X-4)
$$

On reprend le meme processus pour $Q$

$$
\polylongdiv[style=D]{X^3-7X^2 + 7X + 15}{X-3}
$$

Ainsi 

$$
Q(X) = (X-3)(X^2-6X+8) = (X-3)(X-3)(X-5)
$$



\end{myproof}

%}}}
\end{document}

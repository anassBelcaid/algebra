
% Définition {{{ %
\begin{frame}[<+->]
  \frametitle{Définition}
  \small
  \begin{block}{Définition}
    One appelle une \textbf{\alert{application}} $f: E\rightarrow F$ entre deux ensembles $E$ et
    $F$, une correspondance qui associe a \textbf{\structure{tout}} élément
    $ x \in E$ un élément \textbf{\alert{unique}} $y\in F$   noté $f(x)$
  \end{block}
 \begin{center}
 \begin{tikzpicture}[scale=1, transform shape]
   \node[ellipse, minimum width=1.5cm, minimum height=2.5cm, draw=sexyRed!80,
     thick, label=left:$E$] at(0,0)  {};
   \node[ellipse, minimum width=1.5cm, minimum height=2.5cm, draw=Apricot,
     thick, label=right:$F$] at(3,0)  {};
   \node[point, fill=sexyRed!80, label=left:$x$] (A) at (0.2,0.5){};
   \node[point, fill=sexyRed!80] (B) at (0.1,0.0){};
   \node[point, fill=sexyRed!80] (C) at (-0.15,-0.4){};

   \node[point, fill=Apricot,xshift=3cm, label=above right:$f(x)$] (fA) at (0.2,0.5){};
   \node[point, fill=Apricot,xshift=3cm] (fB) at (0.1,0.0){};
   \node[point, fill=Apricot,xshift=3cm] (fC) at (0.15,-0.6){};

   \path[->,>=stealth, thick] (A) edge[bend left=45] (fA);
   \path[->,>=stealth, thick] (B) edge[bend left=35] (fB);
   \path[->,>=stealth, thick] (C) edge[bend right=35] (fB);
 \end{tikzpicture}
 \end{center}
 \pause

 \begin{itemize}
   \item $E$: \textbf{Ensemble de départ}.
   \item $F$: \textbf{Ensemble d'arrivée}.
 \end{itemize}
\end{frame}


\begin{frame}[<+->]
  \frametitle{Application sur $\mathbb{R}$}
  \begin{itemize}
    \item Si $E$ et $F$ sont des \textbf{sous ensembles}  de $\R$. On peut
      représenter $f : \R\rightarrow\R$ par son \alert{\textbf{graphe}}:
      \begin{equation}
        \Gamma_f = \left\{ \big(x,f(x)\big)\in E\times F\;|\; x\in E \right\}
      \end{equation}
  \end{itemize} 
\pause

\begin{center}
\begin{tikzpicture}[yscale=0.8, transform shape]
  \draw[ ->, >=stealth ] (-2,0)--(5,0);
  \draw[ ->, >=stealth ] (0,-1)--(0,4);
  \node at (5,-0.2) {$x$};
  \node at (-0.2,5) {$y$};
  \draw[very thick, Apricot] (-1,0.5) ..controls (1.5, 2) and (2,.9).. (4.5, 1.5);
  \node[point, fill=sexyRed!80, label=below:$x$]  (X) at (1,0){};
  \node[point, fill=sexyRed!80, label=left:$y$]  (Y) at (0,1.3){};
  \node[point, fill=sexyRed!80,label=above:$f(x)$]  (A) at (1,1.3){};
  \node[label=above:$\Gamma_f$]  at (3,1.1){};
  \path[draw] (X)--(A)--(Y);
\end{tikzpicture}
\end{center}
\end{frame}


\begin{frame}[t]
  \frametitle{Égalité et Composition}
 \begin{block}{Égalité}
   Deux applications $f$ et $g: E\rightarrow F$ sont dites \textbf{\alert{égales}} $f=g$ si:
   \begin{equation*}
     \forall x \in E\quad f(x) = g(x)
   \end{equation*}
 \end{block} 
 \pause

 \begin{block}{Composition}
   Soit les deux applications $f: E\rightarrow F$ et $g:F\rightarrow G$. On
   définit alors la \textbf{\alert{composée}}:
   \begin{equation*}
     \left( g\circ f\right)(x) = g\left( f(x)\right)
   \end{equation*}
 \end{block}
 \pause
 \begin{center}
 \begin{tikzpicture}[scale=1,>=stealth]
   \node[] (E) at (0,0) {$E$};  
   \node[] (F) at (3,0) {$F$};  
   \node[] (G) at (6,0) {$G$};  
   \path[->, draw] (E)--(F);
   \path[->, draw] (F)--(G);

   \path[thick,->,>=stealth,draw, Apricot] (E)edge[bend left=35]node[pos=0.5, label=above:$f$]{}(F);
   \path[thick,->,>=stealth,draw,Apricot] (F)edge[bend left=35]node[pos=0.5, label=above:$g$]{}(G);

   \path[thick,->,>=stealth,draw,sexyRed] (E)edge[bend right=25]node[pos=0.5,
     label=below:$g\circ f$]{}(G);
 \end{tikzpicture}
 \end{center}
\end{frame}

\begin{frame}[t]
  \frametitle{Identité}
  \small
 \begin{block}{Identité}
   Une application particulière est l'application \textbf{\textbf{identité}}:
   \begin{eqnarray*}
     \text{id}_E &:& E\rightarrow E\\
                && x \rightarrow x
   \end{eqnarray*}
 \end{block} 
 \pause

 \begin{block}{Mini Exercice}
   Soit $f: ]0,+\infty[\rightarrow ]0,+\infty[$ tel que $f(x)=\dfrac{1}{x}$, et
   $g: ]0,+\infty[\rightarrow \R$  tel que $g(x) = \dfrac{x-1}{x+1}$.
   \begin{itemize}
     \item Donner $f\circ\text{id}$, $\text{id}\circ g$, $g\circ f$ et $f\circ
       g$.
   \end{itemize}
 \end{block}
\end{frame}
% }}} Définition %

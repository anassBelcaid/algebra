% Cardinal {{{ %
%{{{ Définition cardinal
\begin{frame}[<+->]
  \frametitle{Ensembles finis}
  \begin{block}{Cardinal}
    Un ensemble $E$ est \textbf{\alert{fini}} si il existe une bijection entre
    $E$ et l'ensemble $\{1,2,\ldots, n\}$.\\[4pt]

    Dans ce cas l'entier $n$ s'appelle \alert{\textbf{cardinal}} de $E$ et il est
    noté $\Card E$ ou $\vert E\vert$.
  \end{block}
  
  Voici quelque exemples:
  \begin{enumerate}
    \item $E = \{\text{Rouge}, \text{Bleu}, \text{Vert}\}$ est de $\Card
      E=3$.\\[4pt]
    \item $\N$ n'est pas un ensemble fini.\\[4pt]
    \item Le cardinal de $\emptyset$ est $0$.\\[4pt]
  \end{enumerate}
\end{frame}
%}}}
% Propriétées  {{{ %
\begin{frame}[<+->]
  \frametitle{Propriétés Cardinal}
 \begin{block}{Propriétés Cardinal}
   \begin{itemize}
     \item Si $B\subset A$ et $A$ est \textbf{fini}. Alors $B$ est aussi fini et 
       \begin{equation}
         \Card B \leq \Card A
       \end{equation}
      \begin{equation}
        \Card (A-B) =\Card A - \Card B
      \end{equation}
      Ainsi si $\Card A = \Card B  \implies A = B$\\[8pt]
    \item Pour deux ensembles $A$ et $B$ \textbf{\alert{disjoints}}:
      \begin{equation}
        \Card(A\cup B) = \Card A + \Card B
      \end{equation}
  \end{itemize} 
 \end{block} 
\end{frame}
% }}} Propriétées  %
% Cardinal Union {{{ %
\begin{frame}[t]
  \frametitle{Cardinal union}
 \begin{block}{Cardinal Union}
      Pour deux ensembles quelconques
      \begin{equation}
        \Card(A\cup B) = \Card A + \Card B - \Card(A\cap B)
      \end{equation}
 \end{block} 
 \vspace*{1cm}
\begin{center}
 \begin{tikzpicture}[scale=0.7, transform shape]
   \node[ellipse,draw=sexyRed, minimum width=3cm, minimum height=4cm, thick,
   rotate=45, label=left:$E$] at (0,0) (E){};
   \node[xshift=1.5cm, ellipse,draw=Apricot, minimum width=3cm, minimum
   height=3cm, thick, label=right:$F$] at (0,0) (E){};
   \node[point, xshift=-0.5cm] (A) at (0,0){};
   \node[point, xshift=-0.5cm] (A) at (0,-1.2){};
   \node[point, xshift=-0.5cm] (A) at (-1,1.0){};
   \node[point, xshift=-0.5cm] (A) at (0,1){};
   \node[point, xshift=-0.5cm] (A) at (1,0.8){};
   \node[point, xshift=-0.5cm] (A) at (1,0){};
   \node[point, xshift=-0.5cm] (A) at (1.2,0.5){};
   \node[point, xshift=-0.5cm] (A) at (2,1){};
   \node[point, xshift=-0.5cm] (A) at (3,0){};
 \end{tikzpicture}
 \end{center}
 \begin{itemize}
   \item \textbf{\alert{Preuve:}} Utiliser la décomposition:
     \begin{equation*}
       A\cup B = A \cup \left(B - \left(A\cap B\right)\right)
     \end{equation*}
 \end{itemize}
\end{frame}
% }}} Cardinal Union %
% }}} Cardinal %
% Relation avec l'injectivité {{{ %

% Relation Injectivité {{{ %
\begin{frame}[t]
  \frametitle{Relation avec Injectivité, Surjectivité}
 \begin{block}{Proposition}
   Soit $E$ et $F$ deux ensembles finis et $f:E\rightarrow F$ une application.
   Alors:

   \begin{enumerate}
     \item Si $f$ est \alert{injective} alors $\Card E \leq \Card F$.\\[8pt]
     \item Si $f$ est \alert{surjective} alors $\Card E \geq \Card F$.\\[8pt]
     \item Si $f$ est \alert{bijective} alors $\Card E = \Card F$.\\[8pt]
   \end{enumerate}
 \end{block} 
 \pause
 \begin{block}{Démonstration}
   \begin{itemize}
     \scriptsize
     \item Supposons que $f$ est injective. Notons $F^{'}=f(E)\subset F$ l'image directe
       de $E$. Ainsi  chaque élément de $F^{'}$ admet un antécédent unique
       dans $E$. On conclut que $\Card F^{'} = \Card E \leq \Card F$.
     \item Supposons que $f$ est surjective, Ainsi tous les éléments de $F$
       admet au moins un antécédent, On conclut alors que $\Card E \geq \Card
       F$.
   \end{itemize} 
 \end{block}
\end{frame}
% }}} Relation Injectivité %

% Equivalence {{{ %
\begin{frame}[t]
  \frametitle{ Egalité cardinal}
 \begin{block}{Proposition}
  Soit $E$ et $F$ deux ensembles et $f:E\rightarrow F$ une application. Si 
  \begin{equation}
    \Card E = \Card F
  \end{equation}
  les trois propriétés suivantes sont équivalentes:
  \begin{enumerate}
    \item $f$ est injective.\\[8pt]
    \item $f$ est surjective.\\[8pt]
    \item $f$ est bijective.\\[8pt]
  \end{enumerate}
 \end{block} 
 \begin{itemize}
   \item \alert{\textbf{Indice:}} Prouver que $(1)\implies (2)\implies
     (3)\implies (1) $
 \end{itemize}
\end{frame}
% }}}  Equivalence %

% }}} Relation avec l'injectivité %
% Nombre d'applications {{{ %
% Nombre d'applications {{{ %
\begin{frame}[t]
  \frametitle{Nombre d'applications}
 \begin{block}{Proposition}
   Soit $E$ et $F$ tel que $\Card E = n$ et $\Card F=p$.\\[8pt]

   Alors le nombre d'applications \emph{différentes} entre $E$ et $F$ est
   $\mathbf{p^n}$
 \end{block} 
 \begin{itemize}
   \item \textbf{\alert{Indice Preuve:}} On fixe l'ensemble $F$ et on démontre
     par récurrence sur le cardinal de $E$.
 \end{itemize}
 \begin{block}{Exemple}
   \begin{itemize}
     \item Donner le nombre d'application entre $\{0,1,3\}$ et $\{0,1,2\}$.
     \item \textbf{Codage binaire}: Combien de nombre entiers peut on
       \textbf{\alert{coder}} sur $8$ bits. 
   \end{itemize}
 \end{block}
\end{frame}

\begin{frame}[<+->]
  \frametitle{Nombre d'injections}
 
  \begin{block}{Nombre d'injections}
   Soit $E$ et $F$ tel que $\Card E= n$ et $\Card F=p$. Alors le nombre
   \textbf{d'injections} est donné par:

   \begin{equation*}
     p\times (p-1) \times (p-2)\times \ldots \times\left(p-(n-1)\right)
   \end{equation*}
  \end{block}

  \begin{itemize}
    \item \alert{\textbf{Indice Preuve:}} Fixer l'ensemble $F$ et considérer une
      récurrence sur $E$.
  \end{itemize}
  \pause
  \begin{block}{Bijections}
   Le nombre de bijections entre deux ensembles de même cardinal est $n$ :

   \begin{equation*}
     n!
   \end{equation*}
  \end{block}
\end{frame}
% }}} Nombre d'applications %
% Nombre sous ensembles {{{ %

\begin{frame}[t]
  \frametitle{Nombre sous ensembles}
 \begin{block}{Nombre sous ensemble}
   Soit $E$ un ensemble fini tel que $\Card E = n$. 
   Le nombre de sous ensembles de $E$  est donné par:
   \begin{equation}
     \Card \mathcal{P}(E) = 2^n
   \end{equation}
 \end{block} 
 \pause
 \begin{block}{Exemple}
   Énumérer les éléments de $\mathcal{P}\big( \{1,2,3,4,5\}\big)$ selon leur
   cardinal.
 \end{block}
 \begin{itemize}
   \item \textbf{\alert{Indice preuve:}}: Utiliser une récurrence sur le
     cardinal de $E$.
 \end{itemize}
\end{frame}

% }}} Nombre sous ensembles %
% Coefficients de Binome {{{ %

% Coefficients de Binome {{{ %
\begin{frame}[t]
  \frametitle{Coefficients de Binôme}
  \begin{block}{Définition}
  Le nombre de parties contenant $\mathbf{k}$ éléments d'un ensemble de cardinal
  $n$ est noté $\mathbf{\binom{n}{k}}$ ou $C^{n}_{k}$.
  \end{block}
  \pause
  \begin{block}{Corrolaire}
    \begin{itemize}
      \item $\binom{n}{0} = 1$\\[8pt]\pause
      \item $\binom{n}{n} = 1$\\[8pt]\pause
      \item $\binom{n}{1} = n$\\[8pt]\pause
      \item $\binom{n}{k} = \binom{n}{n-k}$\\[8pt]\pause
      \item $\binom{n}{0}+\binom{n}{1}+\binom{n}{2}+\ldots \binom{n}{n}=2^n$.\pause
    \end{itemize}
  \end{block}
\end{frame}
% }}} Coefficients de Binome %

% Propriétées Binom {{{ %
\begin{frame}[<+->]
  \frametitle{Propriétées Coefficiens Binôme}
 \begin{block}{Proposition}

  \begin{equation}
    \binom{n}{k} = \binom{n-1}{k} + \binom{n-1}{k-1} \quad (0< k < n)
  \end{equation}
 \end{block} 

 \begin{itemize}
 \item \textbf{\alert{Indice preuve}}: Pour un un ensemble $E$ de cardinal
     $n$. Considérer un élément $x\in E$, Diviser les parties de taille $k$ en
     ceux qui contiennent $x$ et ce ceux qui ne le contiennent pas. 
 \end{itemize}
\end{frame}
% }}} Propriétées Binom %

% Proposition 2 {{{ %
\begin{frame}[<+->]
  \frametitle{Expression Coefficient Newton}
\begin{block}{Proposition}
  \begin{equation}
    \binom{n}{k} =  \dfrac{n!}{k!(n-k)!}
  \end{equation}
\end{block}
\pause

\begin{itemize}
  \item \alert{\textbf{Indice Preuve:}} Utiliser une récurrence sur $n$
    et la proposition précédante du cours.
\end{itemize}
\end{frame}

\begin{frame}[<+->]
  \frametitle{Mini exercices}
 \begin{enumerate}
   \small
   \item Donner le nombre d'injections entre un ensemble de cardina
     $\mathbf{n}$ et un autre de cardinal $n+1$.\\[8pt]
   \item Calculer le nombre main  de taille $5$ cartes d'un jeu de $32$
     cartes.\\[8pt]
   \item Calculer le nombre de listes de taille $3$ qu'on peut construire
     avec des chiffres ($<10$). Par exemple $(1,2,3)$ et $(2,2,3)$ mais
     pas $(10,2,3)$.\\[8pt]
 \end{enumerate} 

\end{frame}
% }}} Proposition 2 %
% }}} Coefficients de Binome %
% Formule de Newton {{{ %
% }}} Formule de Newton %


\documentclass{beamer}
\usetheme{ensam}
\usepackage{pgfplots}
\usepackage{subcaption}
\usepackage{amssymb}
\usepackage{acronym}
\usepackage{mynotations}
\usepackage{tikz}
\usepackage{polynom}
\usepackage{pgf-pie}
\usetikzlibrary{calc}
\usepackage{amsmath}
\usepackage {algorithmic}
\usepackage{algorithm}
\usepackage{eqparbox}
\usepackage[font=scriptsize]{caption}
\usetikzlibrary{bayesnet,positioning,calc}
\usepgfplotslibrary{groupplots}
\tikzstyle{obs} = [latent,fill=lightBlue]
\tikzstyle{default}=[draw=sexyRed,thick,rounded corners,text width=0.5in,font=\scriptsize,align=center]
\usepgfplotslibrary{colorbrewer}
\definecolor{ForestGreen}{RGB}{34,139,34}
\newcommand{\comment}[1]{\textcolor{ForestGreen}{#1}}
%algorithmic comment
\renewcommand\algorithmiccomment[1]{%
  \hfill\comment{\#\scriptsize\eqparbox{COMMENT}{#1}}%
}
\renewcommand{\algorithmicrequire}{\textbf{Input:}}
\renewcommand{\algorithmicensure}{\textbf{Output:}}
\title{Groupes}
\author{\underline{A.Belcaid}}
\institute{\small Université Euro Méditerranéenne de Fès} 


% Customzation {{{ %
% }}} Customzation %

%tikz bayesian theme
\usetikzlibrary{bayesnet,positioning,calc}
\tikzstyle{obs} = [latent,fill=lightBlue]
\tikzstyle{default}=[draw=sexyRed,thick,rounded corners,text width=0.5in,font=\scriptsize,align=center]
\DeclareMathOperator{\argmin}{argmin}

\pgfplotsset{every tick label/.append style={font=\tiny}}





% add bibliography
\date{\today}
\newcommand{\setK}{\mathbb{K}}

\begin{document}
\maketitle

\begin{frame}
\tableofcontents
\end{frame}


% Définitions {{{ %
\section{Définitions}%
\label{sec:introduction}
% Définition {{{ %
\subsection{Définition}
\begin{frame}[<+->]
  \frametitle{Définition}

  \begin{block}{Définition}
    Une \textbf{\alert{polynôme}} à coéfficients dans $\setK$ est expression de
    la forme:

    \begin{equation}
      P(x) = a_n x^n\;+\; a_{n-1}x^{n-1}\;+\ldots\;+\;a_1x\;+a_0
    \end{equation}
    avec $\mathbf{n}\in \Nn$ et $a_0, a_1,\ldots, a_n\in \setK$,
  \end{block}

  \begin{figure}[htpb]
  \begin{center}
  \begin{tikzpicture}[scale=1, transform shape]
  \begin{groupplot}[group style={group size=3 by 1}, height=4cm, width=4.5cm]
    \nextgroupplot[title={$y=x^3$}]
    \addplot[domain=-5:5, samples=100,color=sexyRed, thick]{x^3};
    \nextgroupplot[title={$y=-2x^3+8x^2+3$}]
    \addplot[domain=-10:10, samples=100,color=sexyRed, thick]{-2*x^3+8*x^2+3};
    \nextgroupplot[title={$y=x^4 - 4*x^3 + 1$}]
    \addplot[domain=-5:5, samples=100,color=sexyRed, thick]{x^4-4*x^3+1};
  \end{groupplot}  
  \end{tikzpicture}
  \end{center}
  \caption{Exemples de polynômes}%
  \label{fig:}
  \end{figure}
\end{frame}
% }}} Définition %
% Notation {{{ %
\begin{frame}[<+->]
  \frametitle{Notation}
 \begin{block}{Noations}
   \begin{itemize}
     \small
     \item L'ensemble des polynômes est noté $\setK[X]$.\\[4pt]
     \item Si \textbf{tous les coefficients} sont \textbf{\alert{nuls}}, alors
       $P$ est appelé \textbf{\alert{polynôme nul}} et il est noté
       $\mathbf{0}$.\\[4pt]
     \item Le \textbf{degré}  d'un polynôme, note $\text{deg} P$, est le plus
       \textbf{grand}  entier $n$ tel que $a_n$ est non nul.\\[4pt]
      \item Par convention, on note le degré du polynôme nul par: $deg(0) =
        -\infty$
      \item Un polynôme de degré $0$ s'écrit comme $P(x)=a_0$ est appelé un
        \textbf{\alert{polynôme constant}}.
   \end{itemize}
 \end{block} 
 \begin{block}{Exemples}
   \small
   \begin{enumerate}
     \item $P(x) = 2x^3 -1$.\\[4pt]    
     \item $P(x) = x^4 - 3x^2 + 4$.\\[4pt]
     \item $P(x) = 100 $.\\[4pt]
   \end{enumerate}
 \end{block}
\end{frame}
% }}} Notation %
% Opérations  {{{ %
\subsection{Opérations sur les polynômes}%
\begin{frame}[t]
  \frametitle{Opérations sur les polynômes}

  \begin{block}{Égalité}
    \small
    Soient deux polynômes $P=\sum_{i}^na_i x^i$ et $Q=\sum_i^n b_ix^i$.\\[4pt]
    L'\textbf{\alert{égalité}}
  entre $P$ et $Q$ implique l'égalité entre \textbf{\alert{tous}} les
  coefficients de  $P$ et $Q$.

  \begin{equation}
    P = Q \iff a_i = b_i\quad \forall i \in [0, n]
  \end{equation}
  \end{block}
\pause 
  \begin{block}{Addition}
    \small
    Soit un polynôme $P=\sum_i ^n a_i x^i$ de degré $n$ et un polynôme
    $Q=\sum_i^m a_i x^i$ de degré $m$.\\[4pt]

    La \alert{\textbf{somme}}  $P+Q$ est aussi un polynôme dont les coefficients
    sont donnés par: 

    \begin{equation}
      (P+Q)(x) = \sum_i^{\max(n,m)} c_i x^i\quad \text{tel que}\;\; \alert{c_i =
      a_i +b_i}
    \end{equation}
  \end{block}
\end{frame}

\begin{frame}[t]
  \frametitle{Opérations sur les polynômes}

  \begin{block}{Multiplication par un scalaire}
    
    Soit $P(x)= \sum_i a_i x^i$ un polynôme de $\setK[x]$ et $\lambda$ un
    scalaire dans $\setK$.\\[4pt]

    La multiplication de $P$ par le scalaire $\lambda$ est aussi un polynôme
    note $\lambda P$ dont les coefficients:

    \begin{equation}
      (\lambda P)(x) = \sum_i b_i x^i  \quad\text{tel que}\;\;
      \alert{b_i = \lambda a_i}
    \end{equation}
  \end{block}
\pause

  \begin{block}{Multiplication}
    \small
    Soit $P(x)=\sum_i^n a_i x^i$ un polynôme de degré $n$ et $Q(x)=\sum_i^m b_i
    x^i$ un polynôme de degré $m$.\\[4pt]

    Le produit $(PQ)$ est aussi un \textbf{polynôme} donné par la formule
    suivante:

    \begin{equation}
      (PQ)(x) = \sum_k^{\alert{n+m}}c_k x^k \quad \text{tel que}\;\; \alert{c_k=
      \sum_{i+j=k} a_ib_j}
    \end{equation}



  \end{block}
  
\end{frame}

% }}} Opérations  %
% Exemples {{{ %
\begin{frame}[<+->]
  \frametitle{Exemples}
 \begin{block}{Exemple}
   Soit le polynôme $P(X)= -5x^3 + 2x -5$ et $Q(X) = -2x^2+ 9x$.\\[8pt]
   \begin{enumerate}
     \item Ese qu'on possède que $P = Q$.\\[4pt]
       \item Calculer le polynôme $\mathbf{4} P$.\\[4pt]
     \item Evaluler l'expression du polynôme $\mathbf{P + Q}$.\\[4pt]
     \item Donner le \textbf{coefficient } de $x^3$ du polynôme $P\times
       Q$.\\[4pt]
   \end{enumerate}
 \end{block} 

\end{frame}
% }}} Exemples %
% Structure des polynomes {{{ %

\begin{frame}[<+->]
  \frametitle{Propriétées}
 
  \begin{block}{Proposition }
    Soient $P,\;Q$ et  $R$ des polynômes de $\setK[X]$. On a alors:\\[4pt]
\small
    \begin{itemize}
      \item $P + Q = Q + P$ \scriptsize{\alert{Commutativité}}
      \item $P + 0 = P$ \scriptsize{\alert{élement neutre}}
      \item $P + \left(Q + R\right) = \left( P + Q\right) +
        R$ \scriptsize{\alert{Associativité}}
      \item $1.P = P$ 
      \item $P\times Q = Q\times P$.
      \item $\left(P\times Q\right)\times R = P\times \left(Q\times R\right)$
      \item $P\times\left( Q + R\right) = P\times Q + P\times R$
    \end{itemize}
  \end{block}
\end{frame}

\begin{frame}[t]
  \frametitle{Propriété degré}
 Soient deux polynômes $P$ et $Q$ deux polynômes dans $\setK$. 

 \begin{block}{Degré Produit}
  \begin{equation}
    \text{deg}\left(P\times Q\right) \alert{=} \text{deg} P + \text{deg} Q
  \end{equation} 
 \end{block}

 \pause
 \vspace*{1cm}

 \begin{block}{Degré Somme}
   \begin{equation}
     \text{deg}\left(P\;+\;Q\right)\alert{\leq} \max(\text{deg} P, \text{deg} Q)
 \end{equation}
 \end{block}

\end{frame}
% }}} Structure des polynomes %

% Vocabulaire {{{ %
\subsection{Vocabulaire}
\begin{frame}[<+->]
  \frametitle{Vocabulaire}
  Voici un vocabulaire additionnel sur les polynômes:\\[8pt]

  \begin{itemize}
    \small
    \item Un polynôme contenant un \textbf{seul terme} de la forme
      $\mathbf{a_kx^k}$ ($a_k \neq 0$) est appelé un \textbf{\alert{monôme}}.\\[4pt]

    \item  Pour un polynôme de la forme $P(x) = a_nx^n +\ldots + a_1x + a_0$, on
      appelle le \textbf{\alert{terme dominant}} le monôme $a_n x^n$ et le
      coefficient $a_n$ est appelé le \textbf{\alert{coefficient
      dominant}}.\\[4pt]

    \item  En cas où le coefficient dominant est $\mathbf{1}$, on dit que le
      polynôme est \textbf{\alert{unitaire}}.
  \end{itemize}

  \begin{block}{Exemples}
    \begin{itemize}
      \item $P(x) = \underComment{3x^4}{\scriptsize terme
        dominant}\;+\;4x^3\;+\;
        \underComment{2x^2}{\scriptsize monôme} + 1$\\[6pt]
      \item $P(x) = x^2 - 1$.
    \end{itemize} 
  \end{block}
\end{frame}
% }}} Vocabulaire %

% Mini Exercices 1 {{{ %

\begin{frame}[t]
  \frametitle{Mini Exercices 1}
 \begin{block}{Mini Exercices 1}
  Soit $P(x) =3x^3 - 2$ , $Q(x) = x^2+x-1$ et $R(x) = ax+b$.\\[8pt]
   \begin{enumerate}
     \small
     \item Calculer $P +  Q$, $P\times Q$, $(P+Q)\times R$.\\[6pt]
     \item Donner $a$ et $b$ pour \textbf{\alert{minimiser}} le degré de
       $P-QR$.\\[6pt]
      \item Montrer que $\text{deg}P\neq \text{deg}Q$, alors
        $\text{deg}(P+Q)=\max{\text{deg}P, \text{deg}Q}$. \\[6pt]

      \item Soit $P_1=(X+1)^4$,
        \begin{itemize}
          \item Développer $P_1$.
          \item Quel est son terme dominant? $P_1$ 
          \item $P_1$ est il unitaire?
        \end{itemize}

   \end{enumerate}
 \end{block} 
\end{frame}
% }}} Mini Exercices 1 %


% }}} Introduction %
% Arithmétiques sur les polynomes {{{ %
\section{Arithmétiques sur les polynômes}%
\label{sec:arithmétiques_sur_les_polynômes}

% Division euclidienne {{{ %
\begin{frame}[t]
  \frametitle{Division Euclidienne}
  \small
  Les opérations \textbf{arithmétiques} portent une grande
  \alert{\textbf{similarité}} avec celle des entiers relatifs $\mathbb{Z}$. Par
  exemple on peut généraliser la notion de \textbf{\alert{division
  euclidienne}}:


  \begin{block}{Division}
    Soit $P$ et $Q$ deux polynômes de $\setK[X]$, on dit que $P$
    \textbf{\alert{divise}}   $Q$ et on note $\mathbf{P\vert Q}$, s'il existe un polynôme $R\in \setK[X]$ tel
    que:
    \begin{equation}
      Q = P\times R
    \end{equation}
    On dit aussi que $Q$ est un \textbf{multiple}  de $P$.
  \end{block}
\pause
  \begin{block}{Relations communes}
    \begin{itemize}
      \item $ P \vert P$\\[4pt]
      \item $ 1 \vert P$\\[4pt]
    \item $ P \vert 0$\\[4pt]
    \end{itemize}   
  \end{block}
  
\end{frame}
% }}} Division euclidienne %
% Relation d'ordre {{{ %
\begin{frame}[t]
  \frametitle{Relation d'ordre (ou presque!)}
  
\small
  \begin{block}{Propriété}
    Soient $P$, $Q$ et $R$ trois polynômes de $\setK[X]$, on a alors:

    \begin{itemize}
      \item $ P \vert P$\\[8pt]
      \item $ P \vert Q$ et $Q\vert P$ alors \alert{\textbf{il existe}} un réel
        $\alert{\lambda} \in \setK$ tel que $ P = \alert{\lambda} Q$.\\[8pt]
      \item $\left(P\vert Q\right)$ et $\left(Q \vert R\right)\quad \implies P \vert R$.\\[8pt]
      \item Si $P | Q$ et $P\;\vert R$ alors $P\;|\; \left(\alert{\alpha\times Q
            + \beta\times 
        R}\right)\quad
        \forall \alpha, \beta \in \setK$\\[8pt]
    \end{itemize}
  \end{block}
\end{frame}
% }}} Relation d'ordre %
% Division Euclidienne {{{ %
\begin{frame}[t]
  \frametitle{Division euclidienne}
 \begin{block}{Théorème}
   \small
   Soient $A$ et $B$ deux polynômes de $\setK[X]$ avec $B\neq 0$. Alors il
   existent deux polynômes \textbf{\alert{uniques}} $Q$ et $R$ tel que:

   \begin{equation}
     \label{eq:euc_div}
     A = Q\times B + R\quad \text{tel que }\quad \text{deg} R \alert{\mathbf{<}}
     \text{deg} B
   \end{equation}
 \end{block} 

 \begin{itemize}
   \small
   \item L'écriture dans \eqref{eq:euc_div} est appelée \textbf{\alert{division
     euclidienne}}.\\[4pt]
   \item $Q$ est appelé le \textbf{\alert{Quotient}}.\\[4pt]
   \item $R$ est note le \textbf{\alert{Reste}}. \\[4pt]
    \item On remarque que si $R=0   \iff B\;|\; A$.
 \end{itemize}

 \vspace*{1cm}
 Proposer une démonstration par récurrence sur le degré de $A$.
\end{frame}

\begin{frame}[t]
  \frametitle{Exemples}
  \begin{block}{Méthode de calcul}
    \begin{itemize}
      \item  Calculer le résultat de la division euclidienne entre $A$ et $B$
        pour les cas suivants:\\[8pt]

        \begin{itemize}
          \item $A= 2X^4 - X^3 -2X^2 +3X -1$ et $B=X^2 - X  + 1$.\\[4pt]
        \end{itemize}
    \end{itemize} 
  \end{block}
  \pause
  \begin{center}
  \polylongdiv[style=D]{2x^4 -x^3 -2x^2 +3x -1}{x^2-x+1}
\end{center}
\end{frame}


\begin{frame}[t]
  \frametitle{Exemples}
  \begin{block}{Méthode de calcul}
    \begin{itemize}
      \item  Calculer le résultat de la division euclidienne entre $A$ et $B$
        pour les cas suivants:\\[8pt]

        \begin{itemize}
          \item $A = X^4 -3X^3 + X + 1$ et $B = X^2 +2$.
        \end{itemize}
    \end{itemize} 
  \end{block}
  \pause
  \begin{center}
  \polylongdiv[style=D]{x^4 -3x^3 +x + 1}{x^2 + 2}
\end{center}
\end{frame}
% }}} Division Euclidienne %

% PGCD {{{ %
\subsection{pgcd}
\begin{frame}[<+->]
  \frametitle{PGCD}
  
  \begin{block}{Définition}
    \small
    Pour deux polynômes $P$ et $Q$ dans $\setK[X]$, avec $P\neq 0$ et $Q\neq 0$.
    Il existe alors un \textbf{unique}  polynôme \textbf{\alert{unitaire}} de
    plus grand degré qui divise à la fois $P$ et $Q$.\\

    \begin{itemize}
      \item Cet unique polynôme est appelé \textbf{\alert{pgcd}} (plus grand
        commun diviseur) et il est noté $\mathbf{\text{pgcd}(P,Q)}$.
    \end{itemize}
  \end{block}
  \pause
  \begin{block}{Remarques}
    \small
    \begin{itemize}
      \item $\text{pgcd}(P,Q)$ est un polynôme \textbf{\alert{unitaire}}.\\[4pt]
      \item Si $P\;\vert\; Q$ et $P\neq 0$ alors $\text{pgcd}(P,Q) =
        \alert{\frac{1}{a_n}}P$ où $\mathbf{a_n}$ est le coefficient dominant de $P$.
      \item Pour tout $\lambda \in \setK$ on a:
        \begin{equation}
          \text{pgcd}(P, Q) = \text{pgcd}(\lambda P, Q)
        \end{equation}
    \end{itemize}  
  \end{block}
\end{frame}


\begin{frame}[t]
  \frametitle{Méthode de calcul du PGCD}
 \begin{block}{Algorithme d'Euclide}
   \small
   Vous avez appris en arithmétique dans $\Zz$ que:
   \begin{equation}
     \text{pgcd}(a, b) = \text{pgcd}(b, r)
   \end{equation}
   où $r$ est le reste de la division euclidienne de $a$ et $b$.
 \end{block} 

 La génération de cet algorithme s'applique aussi aux \textbf{\alert{polynômes}}
 ainsi on a:

 \begin{block}{Algorithme Euclide (version polynôme)}
   \small
   Soient $A$ et $B$ deux polynômes de $\setK[X]$ et $(Q, R)$ le résultat de la
   division euclidienne de $A$ par $B$ (i.e \alert{$A = Q\times B + R$}). Alors on a:
\pause
   \begin{equation}
     \text{pgcd}( A, B)  = \text{pgcd}( B, R)
   \end{equation}
 \end{block}
\end{frame}
% }}} PGCD %
% Méthode calcul pgcd {{{ %
\begin{frame}[t]
  \frametitle{Exemple de calcul}
  \begin{block}{Exemple}
Utiliser l'algorithme d'Euclide pour calculer le \textbf{pgcd} des deux
polynômes: 
\begin{itemize}
  \small
  \item $ A = X^4 -1$.
  \item $ B = X^3 - 1$.
\end{itemize}
\end{block}
\begin{center}
  \polylonggcd[style=C]{x^4-1}{x^3-1}
\end{center}
\pause
\begin{itemize}
  \item \textbf{\alert{Conclusion:}}  Le pgcd est le dernier reste \textbf{non
    nul}. Alors $$\text{pgcd}(x^4-1, x^3-1) = \mathbf{x-1}$$. 
\end{itemize}
\end{frame}

\begin{frame}[t]
  \frametitle{Exemple de calcul}
  \begin{block}{Exemple}
Utiliser l'algorithme d'Euclide pour calculer le \textbf{pgcd} des deux
polynômes: 
\begin{itemize}
  \small
  \item $ A = X^3 -2X^2 + 3$.
  \item $ B = X^2 - 1$.
\end{itemize}
\end{block}
 \pause
\begin{center}
  \small
  \polylonggcd[style=C]{x^3- 2x^2 + 3 }{x^2 -1}
\end{center}
\pause
\begin{itemize}
  \item \textbf{\alert{Conclusion:}}  Le pgcd est le dernier reste \textbf{non
    nul}. Alors $$\text{pgcd}(x^3 - 2x^2 + 3, x^2-1) = \mathbf{x+1}$$. 
\end{itemize}
\end{frame}
% }}} Méthode calcul pgcd %
% Polynomes premiers {{{1 %
\subsection{Polynômes premiers}

\begin{frame}[t]
  \frametitle{Polynômes premiers}
  
  \begin{block}{Définition}
    \small
    Soit $P$ et $Q$ deux polynômes de $\setK[X]$. On dit que les $P$ et $Q$ sont
    \textbf{\alert{premiers}} entre eux si
    $$
    \text{pgcd}( P, Q) = 1
    $$
  \end{block}
  \pause
\begin{block}{Remarque}
  \small
  Pour deux polynômes quelconque, $P$ et $Q$ (qui ne sont forcément premiers
  entre eux). On peut construire deux polynômes $P^{'}$ et $Q^{'}$ qui sont
  premiers entre eux. Il suffit de calculer le \textbf{pgcd}(P,Q).\\

  \begin{equation}
    K = \text{pgcd}(P,Q)\implies \begin{array}{ll}
      P = K \times \alert{P^{'}}\\
      Q = K \times \alert{Q^{'}}
    \end{array}
  \end{equation}
  Alors forcément $P^{'}$ et $Q^{'}$ sont premiers entre eux.
\end{block}
\end{frame}
% }}} Arithmétiques sur les polynomes %
% Théorème de Bezout {{{ %
\subsection{Théorème de Bézout}
% Théorème {{{ %
\begin{frame}[t]
  \frametitle{Théorème de Bézout}
\begin{block}{Théorème}
  Soient $P$ et $Q$ deux polynômes dans $\setK[X]$ tel que $P\neq 0$ and $Q\neq
  0$. Soit alors $D=\text{pgcd}(P,Q)$ leur pgcd. Alors il exisent deux polynômes
  $U$ et $V$ dans $\setK[X]$ tel que:

  \begin{equation}
    P\alert{\mathbf{U}} + Q\alert{\mathbf{V}} = K
  \end{equation}
\end{block}  

\pause
\begin{itemize}
  \small
\item La preuve découle directement en \alert{remontant} l'algorithme d'Euclide.
\end{itemize}
\pause

\begin{block}{Corollaire}
  Si $P$ et $Q$ dans $\setK[X]$ sont \textbf{premiers entre eux}. alors il exisent deux
  polynômes $U$ et $V$ tel que:

  \begin{equation}
    P\alert{\mathbf{U}} + Q\alert{\mathbf{V}} = 1
  \end{equation}
\end{block}

\end{frame}

\begin{frame}[<+->]
  \frametitle{Corollaires Bézout}
 \begin{block}{Corollaire}
  Soient $P$, $Q$ et $R$ tel que $P\neq 0$ et $Q\neq 0$, alors on a la propriété
  suivante:

  \begin{equation}
    R\;\vert\; P \quad \text{et}\quad R\;|\; Q \implies R\;|\;
    \alert{\text{pgcd}(P,Q)}
  \end{equation}
 \end{block} 

 \pause

 \begin{block}{Corollaire}
   Soient $P$, $Q$ et $R$ des polynômes de $\setK[X]$, alors
   \begin{equation}
     P\;|\; QR \quad \text{et}\quad \text{pgcd}(P,Q)=1 \implies P\;|\;R
   \end{equation}
 \end{block}
\end{frame}
% }}} Théorème %
% }}} Théorème de Bezout %
% PPCM {{{ %
\subsection{PPCM}
\begin{frame}[t]
  \frametitle{PPCM}
 \begin{block}{Définition}
   \small
   Soit $P$, $Q$ deux polynômes  \textbf{non nuls} de $\setK[X]$, alors il
   existe un \textbf{unique}  polynôme $\mathbf{M}$ \alert{\textbf{unitaire}} de plus petit degré tel que:
   $$ P\;|\; M \quad et\quad Q\;|\; M$$.
     Le polynôme $M$ est appelé le \textbf{\alert{ppcm}} (plus petit
       commun multiple) et noté $\text{ppcm}(P, Q)$.
 \end{block} 
 \pause

 \begin{block}{Exemple}
   \begin{itemize}
     \item Donner le \textbf{ppcm} de:
  \begin{enumerate}
    \item  $P = x^2(x-2)^2(x^2+1)^4$.\\[4pt]
    \item  $Q=(x+1)(x-2)^3(x^2+1)^3$.
  \end{enumerate}
   \end{itemize} 
 \end{block}
\end{frame}
\begin{frame}
  \frametitle{Propriété PPCM}
 \begin{block}{Remarque}
  Le \textbf{ppcm} est le plus petit des polynômes qui vérifie cette propriété.
  Ainsi:
  \begin{equation}
    P\;|\; M\quad Q\;|\; M\implies \text{ppcm}(P, Q)\;|\;M 
  \end{equation}
 \end{block}
\end{frame}
% }}} PPCM %
% Mini Exercices {{{ %
\begin{frame}[t]
  \frametitle{Mini exercice}
 \begin{block}{Mini exercices}
   \small
   \begin{enumerate}
     \item Donner les diviseurs de $P(x) = X^4 + 2X^2 +1$.\\[4pt]
     \item Montrer que $(X-1)$ divise $X^n - 1$ (pour $n\geq 1$).\\[4pt]
     \item Calculer les divisions euclidienne des polynomes suivants: 
       \begin{itemize}
         \item $P(X)=X^4 -1$ et $Q(X) = X^3-1$.\\[4pt]
         \item $P(X)=4X^3 +2X^2-X-5$ et $Q(X)=X^2+X$.
       \end{itemize}
 \item Déterminer les pgcd de $P(X) = X^5 + X^3+X^2 + 1$ et
   $Q(X)=2X^3+3X^2+2X+3$.
   \begin{itemize}
     \item Trouver les polynômes $U$ et $V$ de Bézout
   \end{itemize}
 \end{enumerate}
 \end{block} 
\end{frame}
% }}} Mini Exercices %
% Racine et Factorisation {{{ %
\section{Racine d'un polynôme, Factorisation}%
\label{sec:racine_d_un_polynôme_factorisation}
% Racine Polynome {{{ %
\subsection{Racine d'un polynôme}
\begin{frame}[t]
  \frametitle{Racine d'un polynôme}
  \begin{block}{Définition}
  Soit $P\in \setK[X]$ et \alert{$\alpha \in \setK$}. On dit que $\alpha$ est
  une \textbf{\alert{racine}}  de $\mathbf{P}$ si:

  \begin{equation}
    \label{eq:poly_root}
    P(\alpha ) = 0
  \end{equation}
\end{block}
\pause

\begin{block}{Proposition}
  \begin{equation}
    P(\alpha) = 0 \quad \iff \quad (X - \alpha)\;\text{\alert{divise}}\; P
  \end{equation}
\end{block}
\pause

\begin{itemize}
  \small
  \item \textbf{Preuve} : Appliquer la division euclidienne de $P$ par $X -
    \alpha$.
\end{itemize}
\end{frame}
% }}} Racine Polynome %
% Dérivée d'un Polynômes {{{ 
\begin{frame}[t]
  \frametitle{Dérivée}
  
  \begin{block}{Proposition}
    Soit $P\in \setK[X]$ un polynôme, La dérivée de $P$ est aussi un polynôme
    dont l'expression est donnée par:

    \begin{eqnarray*}
      P^{'}(X) &=& na_{n} X^{n-1} + (n-1)a_{n-1}X^{n-2}+\ldots + 2a_2X + a_1\\
               & = & \sum_{k=1}^{n} ka_k\;X^{k-1}
    \end{eqnarray*}

  \end{block}
\end{frame}
% }}} Dérivée d'un Polynômes %
% Propriétés {{{ %
\begin{frame}[t]
  \frametitle{Propriétés des racines}
 \begin{block}{Définition}
   \small
   Soit $\alpha$ une \textbf{racine} du polynôme $P\in \setK[X]$, on dit que
   $\alpha$ est de \textbf{\alert{multiplicité}}  $\mathbf{k} \in\Nn$ si $k$ est
   le plus grand entier tel que:

   \begin{equation}
     \left( X - \alpha \right)^{\alert{k}} \;\text{divise}\; P 
   \end{equation}
 \end{block} 
 \begin{itemize}
   \small
   \item Si $k=1$,  on parle d'une racine \textbf{\structure{simple}}.
   \item Si $k=2$, on dit que la racine est \textbf{\structure{double}}.
   \item D'une manière générale, on dit que la racine est \textbf{d'ordre $k$ }
 \end{itemize}
 \pause
 \begin{block}{Proposition}
   \small
   On démontre \textbf{l'équivalence}  entre les propriétés suivantes:

   \begin{enumerate}
     \item $\alpha$ est une racine d'ordre $k$ de P.\\[4pt]
     \item il existe $Q\in \setK[X]$ tel que:
       $
       P = \left( X - \alpha \right)^k Q\;\quad\; \text{avec: } Q(\alpha) = 0
       $
     \item $\forall k \in {0,\ldots, k-1}\;\quad\; P^{k}(\alpha) = 0$ et
       $P^{k}(\alpha) \neq 0$. 
   \end{enumerate}
 \end{block}

\end{frame}
% }}} Propriétés
% Théorème d'Alebert-Gauss {{{ %
\begin{frame}[t]
  \frametitle{Théorème de d'Alembert-Gauss}
  \begin{block}{Théorème}
    Tout polynôme à \textbf{\alert{coefficients complexes}}  de degré $n\geq 1$
    a au moins une racine dans $\Cc$.\\

    Si on compte chaque racine avec son \textbf{degré}, alors il admet
    exactement  $\alert{n}$ racines.
  \end{block}
  \pause

  \begin{block}{Exemple}
    \small
    Pour un polynôme de deuxième degré $P(X) = aX^2 + bX + c$ à coefficients
    réels: $a\neq 0, b,c \in \Rr$ 
    \begin{enumerate}
      \small
      \item $\Delta >0\implies x_1 = \dfrac{-b + \sqrt{\Delta}}{2a} 
        \;\text{et}\; x_2 = \dfrac{-b - \sqrt{\Delta}}{2a}$ 
      \item $\Delta <0\implies x_1 = \dfrac{-b + i\sqrt{\Delta}}{2a} 
        \;\text{et}\; x_2 = \dfrac{-b - i\sqrt{\Delta}}{2a}$ 
      \item $\Delta =0$  une seule racine \textbf{\alert{double}}  $x =
        \dfrac{-b}{2a}$.
    \end{enumerate}
  \end{block}
\end{frame}
% }}} Théorème d'Alebert-Gauss %
% Korps plus faible {{{ %
\begin{frame}[t]
  \frametitle{D'autre corps}
  \begin{itemize}
    \item Pour des corps différent des \textbf{nombres complexes}, nous avons un
   résultat \textbf{\alert{plus faible}}:
  \end{itemize}


 \begin{block}{Théorème}
   Soit $P\in \setK[X]$ de degré $n\geq 1$, alors $P$ admet \textbf{\alert{au
   plus}} $n$ racines dans $\setK$.
 \end{block} 
 \pause
 
 \begin{block}{Exemple}
   Soit le polynôme $ P(X) = 3X^3 - 2X^2 + 6X -4$.\\

   Ce polynôme peut être ecrit comme : $P(X) = 3(X-\frac{2}{3}(X^2+2)) $
   \begin{itemize}
     \item On peut remarquer qu'il n'admet qu'une seule racine de $\Rr$,
       $\alpha=\frac{2}{3}$.\\[4pt]
     \item Or, dans $\Cc$ il admet \textbf{trois}  racines $\left\{\frac{2}{3},
       -i\sqrt{2}, i\sqrt{2}\right\}$
   \end{itemize}
 \end{block}
\end{frame}
% }}} Korps plus faible %
% Polynomes irréductibles {{{ %
\subsection{ Polynômes irréductibles}
\begin{frame}[<+->]
  \frametitle{Polynômes irréductibles}
 \begin{block}{Définition}
   \small
   Un polynôme $P\in \setK[X]$ de degré $n\geq 1$ est dit
   \textbf{\alert{irréductible}} si pour chaque $Q\in \setK[X]$ divisant $P$
   alors on a:

   \begin{itemize}
     \item $Q\in \setK^{*}$ (Q est une constant).\\[4pt]
     \item  $\exists \lambda \in \setK^{*}$ tel que $Q = \lambda P$.
   \end{itemize}
 \end{block} 
 \pause

 \begin{block}{Remarques}
   \begin{enumerate}
     \small
     \item Les seuls diviseurs d'un polynôme irréductible sont des
       \textbf{constantes}, soit \textbf{lui-même}( \alert{à une constante
       près}).\\[4pt]
      \item La notion de polynôme irréductible remplace celle des nombres
        \textbf{\alert{premiers}} dans $\Zz$.\\[4pt]
      \item Dans le cas où $P$ est \textbf{\alert{réductible}}, on peut alors le
        factoriser comme produit de deux polynômes (ou plus)
        \begin{equation}
          P = AB\quad \text{où}\quad \text{deg}A\geq 1
          \;\text{et}\;\text{deg}B\geq 1.
        \end{equation}
   \end{enumerate}
 \end{block}
\end{frame}
% }}} Polynomes irréductibles %
% Résultat irreductible {{{ %
\begin{frame}[<+->]
  \frametitle{Polynômes irréductibles}
 \begin{block}{Exemple}
   \begin{itemize}
     \small
   \item Tous les polynômes de degré $1$ sont forcément
     \textbf{\textbf{irréductible}}.\\[4pt]
   \item $X^2-1$ est réductible car $X^2-1 = (X-1)(X+1)$\\[4pt]
   \item $X^2 +1$ est irréductible dans $\Rr[X]$.
   \item Cependant, il est \textbf{\alert{réductible}} dans $\Cc[X]$ car 
     \begin{equation}
       X^2 + 1 = (X- i)(X+i)
     \end{equation}
   \end{itemize}
 \end{block} 
 \pause

 \begin{block}{Lemme d'Euclide}
   Pour un polynôme \textbf{\alert{irréductible}} $P\in\setK[X]$, et $A,B\in
   \setK[X]$ on a:

   \begin{equation}
     P\;|\;AB \implies P\;|\; A\quad \text{ou}\quad P\;|\; B
   \end{equation}
 \end{block}
\end{frame}
% }}} Résultat irreductible %
% Factorisaiton {{{ %
\begin{frame}[<+->]
  \frametitle{Factorisation}
 
  \begin{block}{Théorème}
    tout polynôme non constant $P\in \setK[X]$ se factorise en produit de
    polynômes irréductibles:

    \begin{equation}
      P =\lambda P_1^{r_1}P_2^{r_2}\ldots P_m^{r_m} =\lambda \prod_{k=1}^{m} P_k^{r_k}
    \end{equation}
    où $\lambda \in \setK^{*}$, $m\in\Nn^{*}$ et $r_i\in \Nn^{
    *}$ et les $P_i$ sont \textbf{\alert{réductibles}}.
  \end{block}
  \pause
  \begin{block}{Corollaire}
    
    \begin{itemize}
      \item Les seuls polynômes dans $\Cc[X]$ sont les polynômes de degré
        $\mathbf{1}$.\\[4pt]
      \item Pour un polynôme de $\Cc[X]$ de degré $n\geq 1$, la factorisation
        s'écrit comme:

        \begin{equation}
          P=\lambda (X-\alpha_1)^{k_1}(X-\alpha_2)^{k_2}\ldots(X-\alpha_r)^{k_r}
        \end{equation}
    \end{itemize}
  \end{block}
\end{frame}
% }}} Factorisaiton %
% }}} Racine et Factorisation %
%Factorisation Dans R et C{{{ %
\subsection{Factorisation dans $\Rr$ et $\Cc$.}
\begin{frame}[t]
  \frametitle{Factorisation dans $\Rr$ et $\Cc$}
 \begin{block}{Factorisation dans $\Cc$}
   \small
   Les seuls polynômes irréductibles dans \alert{$\Cc[X]$} sont les polynômes
   de degré $1$. Ainsi tout polynôme est décompose selon la forme suivante:

   \begin{equation}
     P(X) = \lambda\left(X- \alpha_1\right)^{r_1}\left(X-\alpha_2\right)^{r_2}\ldots \left(X- \alpha_k\right)^{r_k}
   \end{equation}
 où les $\left(\alpha_i\right)_{1\leq i \leq k}$ sont les
 \textbf{\alert{racines}}  de $P$ et les $r_i$ leurs degré de
 \textbf{multiplicité}.
 \end{block} 
 \pause

 \begin{block}{Décomposition dans $\Rr$}
   \small
   Malheureusement, on as pas cette décomposition puisque un polynôme de degré
   $\alert{\geq 2}$ peut aussi être \textbf{irréductible}. Ainsi obtient une
   décomposition plus faible:

   \begin{equation}
     \label{eq:factor_R}
     P(X) = \lambda\left(X- \alpha_1\right)^{r_1}\left(X-
     \alpha_2\right)^{r_2}\ldots \left(X- \alpha_k\right)^{r_k} Q_1^{l_1}\ldots
     Q_m^{l_m}
   \end{equation}
   où comme dans $\Cc$, les $\alpha_i$ sont les racines, en plus
   $\left(Q_i\right)_{1\leq i \leq m }$ sont des polynômes irréductibles.
 \end{block}
\end{frame}
% }}} Factorisation Dans R et C %
% Exemple décomposition {{{ %
\begin{frame}[t]
  \frametitle{Exemple Décomposition}

  \begin{block}{Exemple 1}
    \small
     On considère le polynôme $P(X) = 2X^4(X-1)^3(X^2+1)^2(X^2 + X + 1)$.
     \begin{itemize}
       \item Peut on décomposer le polynôme $P$ dans $\Rr[X]$.
       \item Exprimer les $\alpha_i$, $r_i$, $Q_i$ et $l_i$ selon l'expression
         \eqref{eq:factor_R}.
       \item Décomposer ce polynôme dans $\Cc$.

     \end{itemize}
   \end{block}
     \pause
     \begin{block}{Mini Exercice}
       \begin{itemize}
         \small
       \item Trouver le polynôme (\alert{de degré minimal}), tel que
         $\frac{1}{2}$ est une racine simple, $\sqrt{2}$ est une racine double
         de ce polynôme, et $i$ est une racine triple.\\[4pt]
       \item Soit $P \in \Cc[X]$. Montrer que si $P$ admet une racine double,
         alors $P$ et $P^{'}$ ne sont pas \textbf{premiers} entre eux.\\[4pt]
       \item Factoriser  le polynôme $Q(X) = 3(X^2-1)(X^2-X+\frac{1}{4})$ dans
         $\Rr[X]$.
       \end{itemize}
  \end{block}
  
\end{frame}
% }}} Exemple décomposition %

% Fractions rationnelles {{{ %
\section{Fractions rationnelles}%
\label{sec:fractions_rationnelles}



% Définition {{{ %
\subsection{Définition}
\begin{frame}[t]
  \frametitle{Définition}
\begin{block}{Définition}
  On appelle une \textbf{\alert{fraction continue}}  dans $\setK[X]$ une
  expression de la forme:

  \begin{equation}
    F  = \dfrac{P}{Q}
  \end{equation}
  où $P$ et $Q$ sont deux polynômes de $\setK[X]$ et $Q \neq 0$.
\end{block}  
\pause
\begin{block}{Exemples }
 \begin{enumerate}
   \small
   \item $F_1(X) = \dfrac{X^2 + 1}{3X^2 - 2X + 1}$.\\[4pt]
   \item $F_2(X) = \dfrac{X-1}{X^2 - 1}$.\\[4pt]
   \item $F_3(X) = \dfrac{X^2 - 1}{(X^2 + 2X + 1}$.\\[4pt]
 \end{enumerate} 
\end{block}
\end{frame}
% }}} Définition %
% Décomposition en élémént simples {{{ %
\subsection{Décomposition en éléments simples}
\begin{frame}[<+->]
  \frametitle{Décomposition en éléments simples}
\begin{block}{Décomposition en éléments simples}
  \small
  Soit $F = \dfrac{P}{Q}$ une fraction continue tel que:

  \begin{itemize}
    \small
    \item $\text{pgcd}(P, Q) = 1$ \\[4pt]
    \item $Q = (X - \alpha_1)^{r_1}(X-\alpha_2)^{r_2}\ldots (X -
      \alpha_k)^{r_k}$
  \end{itemize}
  Alors $F$ peut se décomposer en éléments simples d'une manière
  \textbf{\alert{unique}} comme suit:

  \begin{equation}
    F = E  + \underComment{\sum_{i=1}^{\alert{k}} }{{\scriptsize Somme racines}}
    \underComment{ \sum_{j = 0}^{\alert{r_k-1}} }{{\scriptsize Somme degré}}
    \dfrac{a_{ij}}{(X- \alpha_i)^{r_i - j}}
  \end{equation}

  \begin{itemize}
    \item $E$ s'appelle la \textbf{\alert{partie polynomiale}}.
    \item $\dfrac{a_{ij}}{(X-\alpha_i)^j}$ sont les \textbf{\alert{éléments
      simples}}.
  \end{itemize}
\end{block}
\end{frame}
% }}} Décomposition en élémént simples %
% Exemple Décomposition {{{ %
\begin{frame}[<+->]
  \frametitle{Exemple décomposition}
 \begin{block}{Exemple 1}
  \small 
   Soit la fraction $F(X) = \dfrac{1}{X^2 + 1}$.\\[4pt]
      Vérifier alors que:
       \begin{equation}
         F(X) = \dfrac{a}{X + i} + \dfrac{b}{X - i} \quad \text{où }
         a=\frac{1}{2}i \;\text{et}\; b= -\frac{1}{2}i
       \end{equation}
 \end{block} 
 \pause
 \begin{block}{Exemple 2}
   \small
   Soit la fraction $F(X) = \dfrac{X^4 - 8X^2 + 9X - 7}{(X-2)^2(X+3)}$.\\

   \begin{itemize}
     \small
     \item Vérifier que:
       \begin{equation}
         F(X) = X + 1  + \dfrac{-1}{(X-2)^2} + \dfrac{2}{(X-2)}+ \dfrac{-1}{(X+3)}
       \end{equation}
   \end{itemize}
 \end{block}
\end{frame}
% }}} Exemple Décomposition %
% Méthode de calcul {{{ %
\subsection{Méthode de calcul}
\begin{frame}[<+->]
  \frametitle{Méthode de calcul}
 \begin{block}{Méthode de calcul}
   \begin{enumerate}
     \small
     \item Calculer la division euclidienne de $P$ par $Q$.\\[4pt]
     \item Déterminer la partie polynomiale de cette division.\\[4pt]
     \item Factoriser le dénominateur $Q$ en facteur irréductibles.\\[4pt]
     \item Ecrire la formule générale de la décomposition.\\[4pt]
     \item Utiliser \textbf{l'égalité polynomiale}  pour calculer les résultats.
   \end{enumerate}
 \end{block} 
\end{frame}
% }}} Méthode de calcul %
% Exemple factorisation fraction {{{ %
\begin{frame}[t]
  \frametitle{Exemple détaillé (1)}
 \begin{block}{Exemple}
  \small 
   Décomposer la faction:
   \begin{equation}
     F(X) =\dfrac{P}{Q}= \dfrac{X^5 - 2X^3 +4X^2 - 8X +11}{X^3 - 3X + 2}
   \end{equation}
 \end{block} 
\pause

\begin{block}{Solution}
  \begin{enumerate}
    \small
  \item \textbf{\alert{Division euclidienne}} 
    \begin{equation}
      X^5 - 2X^3 + 4X^2 -8X + 11 = \underComment{(X^2+1)}{ {E} } Q(X) + 2X^2 - 5X + 9.
    \end{equation}
    \pause
  \item \textbf{\alert{Décomposition de $Q$}}: on peut vérifier que $Q$ admet
    deux racines $1$ et $-2$.

    \begin{equation}
      Q(X) = (X-1)^2(X+2)
    \end{equation}
  \end{enumerate}
\end{block}
\end{frame}

\begin{frame}[<+->]
  \frametitle{Exemple détailé (2)}
  
   \begin{enumerate}
     \small
   \item \textbf{\alert{Ecriture théorique}}: 

     \begin{equation}
       \small
       \dfrac{P(X)}{Q(X)} = E(X) + \dfrac{a}{(X-1)^2} + \dfrac{b}{(X-1)} +
       \dfrac{c}{X+2}
     \end{equation}
     \pause

   \item \textbf{\alert{Détermination de coefficients}} 
\begin{equation}
  \small
  \label{eq:fraction_decomp_1}
  \dfrac{a}{(X-1)^2} + \dfrac{b}{(X-1)} +
  \dfrac{c}{X+2} = \dfrac{(b+c)X^2 +(a+b+c)X + 2a-2b+c}{(X-1)^2(X+2)}
\end{equation}
L'expression en \alert{\eqref{eq:fraction_decomp_1}} doit être égale à
$\dfrac{2X^2 -5X + 9}{(X-1)^2(X+2)}$
   \end{enumerate} 
   \small
   On conclut alors que
   \begin{equation}
     \left\{ \begin{array}{lcl}
         b+ c & = & 2\\ 
         a + b + c & = & -5\\
         2a -2b +c & = & 9
     \end{array}\right.
   \end{equation}
   On trouve que:  $\alert{a=2}$, $\alert{b=-1}$  et $\alert{c=3}$
\end{frame}
% }}} Exemple factorisation fraction %
% }}} Fractions rationnelles %



\end{document}

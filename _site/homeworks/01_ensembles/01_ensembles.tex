\documentclass[10pt, a4paper, twocolumn]{homework}
\title{Ensembles et Applications}
\date{08 Octobre 2020}
\author{A.Belcaid}


\begin{document}
\maketitle

% Ensembles {{{ %
%{{{ empty set 
\exercice
Montrer que $\emptyset \subset X$, pour tout ensemble $X$.
%}}}
%{{{ Simple proprieties
\exercice
  Soit $A,B, \in \mathcal{P}(E)$,  montrer par contraposition les assertions suivantes, $E$ \'etant
un ensemble:
\begin{enumerate}
\item $(A\cap B=A\cup B)\Rightarrow A=B$,
\item $ (A\cap B=A\cap C \text{ et } A\cup B=A\cup C)\implies B=C$.
\end{enumerate}
%}}}
% Fold description {{{ %
\exercice
Soient $E$ et $F$ deux ensembles, $f:E\rightarrow F$.\\

D\'emontrer que:

\begin{itemize}
  \item $\forall A,B \in \mathcal{P}(E) \quad (A\subset B)\Rightarrow (f(A)\subset f(B))$
  \item $\forall A,B \in \mathcal{P}(E) \quad f(A\cap B)\subset f(A)\cap f(B)$
  \item $\forall A,B \in \mathcal{P}(E) \quad f(A\cup B) = f(A)\cup f(B)$
  \item $\forall A,B \in \mathcal{P}(F) \quad f^{-1}(A\cup B) = f^{-1}(A)\cup f^{-1}(B)$
  \item $\forall A \in \mathcal{P}(F) \quad f^{-1}(F\setminus A)=E\setminus f^{-1}(A)$
\end{itemize}

% }}} Fold description %
%%{{{ Propriétés
%\exercice
%Soient $E$ un ensemble et $A, B, C$ trois parties de $E$ telles que
%\begin{enumerate}
%  \item $A \cup B = A \cup C$
%  \item $A \cap B = A \cap C$
%\end{enumerate}
%Montrer que $B = C$.
%%}}}
% Propriétes {{{ %
\exercice
Soient $E$ un ensemble et $A, B, C$ trois parties de $E$.\\
Montrer que
$$ (A \cup B) \cap (B \cup C) \cap (C \cup A) =
(A \cap B) \cup (B \cap C) \cup (C \cap A)$$
% }}} Propriétes %
%{{{ ensemble parties
\exercice
Est-il vrai que:

\begin{enumerate}
  \item $\mathcal{P} (A \cap B) = \mathcal{P} (A) \cap \mathcal{P} (B)$ ?
  \item  $\mathcal{P} (A \cup B) = \mathcal{P} (A) \cup \mathcal{P} (B)$ ?
\end{enumerate}
%}}}
%{{{ Propriétes Inclusion


\exercice
 Démontrer les relations suivantes~:
$$ A \cup(B \cap C) = (A \cup B) \cap(A \cup C)$$
 $$A \cap(B \cup C) = (A \cap B) \cup(A \cap C).$$
%}}}
%{{{ Propriétés Inclusion
\exercice
Montrer que si $F$ et $G$ sont des sous-ensembles de $E$~:
\begin{enumerate}
  \item $(F \subset G \iff F \cup G = G )$
  \item $(F \subset G \iff \complement F \cup G = E).$
\end{enumerate}
En déduire que~: 
\begin{enumerate}
  \item $ (F \subset G \iff F \cap G = F)$
  \item $(F \subset G \iff F \cap\complement G = \emptyset).$ 
\end{enumerate}
%}}}
%{{{ Produit cartésien
\exercice
 Soit $A= \{a_1, a_2, a_3, a_4\}$ et $B= \{b_1, b_2, b_3, b_4 ,b_5\}$.
 \begin{itemize}
   \item Écrire le produit cartésien $A \times B$.
   \item Quel est le nombre de parties de $A \times B$~?
 \end{itemize}
% }}} Ensembles %
 %{{{ Différence symétrique
 \exercice
Soit un ensemble $E$ et deux parties $A$ et $B$ de~$E$. 
\begin{enumerate}
\item Démontrer que $A \triangle B = (A \setminus B) \cup(B \setminus A)$.
\item Démontrer que pour toutes les parties $A$, $B$, $C$ de $E$ on a
$(A \bigtriangleup B) \bigtriangleup C = A \triangle(B \triangle C)$.
\item Démontrer qu'il existe une unique partie~$X$ de~$E$ telle que 
pour toute partie~$A$ de~$E$, $A \triangle X = X \triangle A = A$.
\item Démontrer que pour toute partie $A$ de $E$, il existe une partie~$A'$
de~$E$ et une seule telle que $$A \triangle A' = A' \triangle A = X$$.

\textit{ \textbf{Indice:}  \scriptsize Il pourra être commode 
d'utiliser la notion de fonction caractéristique.}
\end{enumerate}
%}}}
%{{{ Equation ensembliste
\exercice
Soient $A, B \subset E$.\\
R\'esoudre les \'equations \`a l'inconnue $X \subset E$
\begin{enumerate}
\item $A \cup X = B$.
\item $A \cap X = B$.
\end{enumerate}
%}}}
%}}}
% Applications {{{ %
 %{{{ Fonction indicatrice
 \exercice
Soit $A\subset E$, on appelle fonction caractéristique de~$A$
l'application~$f$ de~$E$ dans l'ensemble à deux éléments $\{0, 1\}$, telle
que~:
$$f(x)=\begin{cases}
0&  \text{ si } x\notin A \cr 1& \text{ si } x \in A \cr
\end{cases}$$
Soit $A$ et $B$ deux parties de $E$, $f$ et $g$ leurs fonctions
caractéristiques.\\
Montrer que les fonctions suivantes sont les fonctions
caractéristiques d'ensembles que l'on déterminera~:
\begin{enumerate}
\item $1-f$.
\item $fg$.
\item $f+g-fg$.
\item $f + g - 2fg$
\end{enumerate}
%}}}
% Symetrie Composition {{{ %
\exercice
Soient $f : \Rr \rightarrow \Rr$ et $g : \Rr \rightarrow \Rr$ telles que $f(x) =
3x+1$ et $g(x)=x^2-1$.
\begin{itemize}
  \item  A-t-on $f\circ g=g\circ f\;$?
\end{itemize}

% }}} Exercice %
% image directe {{{ %
\exercice
Soit l'application de $\mathbb{R}$ dans $\mathbb{R}$, 
$f\colon x\mapsto x^2$.
\begin{enumerate}
\item Déterminer les ensembles suivants:
  \begin{itemize}
    \item $f([-3,-1])$
    \item $f([-2,1])$
    \item $f([-3,-1]\cup[-2,1])$
    \item $f([-3,-1]\cap[-2,1])$
  \end{itemize}
   
\item Mêmes questions avec les ensembles:
  \begin{itemize}
    \item $f^{-1}(\mathopen]-\infty,2])$
    \item $f^{-1}([1,+\infty\mathclose[)$
    \item $f^{-1}(\mathopen]-\infty,2]\cup\nolinebreak{}[1,+\infty\mathclose[)$
    \item $f^{-1}(\mathopen]-\infty,2]\cap\nolinebreak{}[1,+\infty\mathclose[)$
  \end{itemize}
\end{enumerate}
% }}} image directe %
%{{{ Coupe dans R2
\exercice
On définit les cinq ensembles suivants :
\begin{eqnarray*}
A_1 & = & \left\{(x,y)\in\mathbb{R}^2\,,\; x+y<1\right\}\\
A_2 & = & \left\{(x,y)\in\mathbb{R}^2\,,\; x+y>-1\right\}\\
A_3 & = & \left\{(x,y)\in\mathbb{R}^2\,,\; |x+y|<1\right\}\\
A_4 & = & \left\{(x,y)\in\mathbb{R}^2\,,\; |x|+|y|<1\right\}\\
A_5 & = & \left\{(x,y)\in\mathbb{R}^2\,,\; |x-y|<1\right\}\\
\end{eqnarray*}

\begin{enumerate}
\item Représenter ces cinq ensembles.
\item En déduire une démonstration géométrique de
$$(|x+y|<1\;\mbox{ et }\;|x-y|<1) \Leftrightarrow |x|+|y|<1.$$
\end{enumerate}
%}}}
% Composée {{{ %
\exercice 
Soit $X$ un ensemble. Pour $f \in \mathcal{F} (X, X)$, on d\'efinit


\begin{equation*}
  f^{n} = \left\{
    \begin{array}{ll}
      \text{id} & \text{si } n=0\\[4pt]
      f^{n+1} = f^n \circ f & n \in \N
    \end{array}
  \right.
\end{equation*}


\begin{enumerate}
\item Montrer que $\forall n \in \Nn$ $$f^{n + 1} = f \circ f^n$$
\item Montrer que si $f$ est bijective alors $\forall n \in \Nn$: $$ (f^{-1})^n
 = (f^n)^{-1}$$
 \end{enumerate}
% }}} Composée %
% }}} Applications %
% Injection et Bijection {{{ %
%{{{ Exemple d'application injective et surjectives
\exercice
Donner des exemples d'applications de $\Rr$ dans $\Rr$ (puis de $\Rr^{2}$ dans $\Rr$)
 injective et non surjective, puis surjective et non injective.
%}}}
 %{{{ Injectivité
%{{{ Déterminer injectivité
\exercice
Soit $f  : \Rr \rightarrow \Rr$ d\'efinie par $f(x) = x^3-x$.
\begin{itemize}
  \item $f$ est-elle injective? surjective?
  \item D\'eterminer $f^{-1}([-1,1])$ et $f(\Rr_+)$.
\end{itemize}
%}}}
% Detemriner Injectivité {{{ %
\exercice
Les fonctions suivantes sont-elles injectives? surjectives? bijectives?
$$ f : \Zz\rightarrow\Zz, \ n\mapsto 2n \quad ; \quad f : \Zz\rightarrow\Zz ,\ n\mapsto -n $$
$$ f:\Rr\rightarrow\Rr ,\ x\mapsto x^2 \quad ; \quad f : \Rr\rightarrow\Rr_+ ,\ x\mapsto x^2 $$
$$   f : \Cc\rightarrow\Cc ,\ z\mapsto z^2.$$
% }}} Detemriner Injectivité %
%{{{ Déterminer l'injectivité
\exercice
Les applications suivantes sont-elles injectives, surjectives, bijectives?
\begin{enumerate}
\item $f : {\Nn} \to {\Nn}, {n} \mapsto {n + 1}$
\item $g : {\Zz} \to {\Zz}, {n}\mapsto{n + 1}$
\item $h : {\Rr^2} \to {\Rr^2}, {(x, y)}\mapsto{ (x + y, x-y)}$
\item $k : {\Rr \setminus \left\{ 1\right\}} \to {\Rr}, {x}\mapsto{\frac{x + 1}{x - 1}}$
\end{enumerate}
%}}}
%{{{ Déterminer injectivité
\exercice
Soit $f  : \Rr \rightarrow \Rr$ d\'efinie par $f(x) = 2x/(1+x^2)$.
\begin{enumerate}
    \item $f$ est-elle injective? surjective?
    \item Montrer que $f(\Rr)=[-1,1]$.
    \item Montrer que la restriction $g  : [-1,1] \rightarrow [-1,1]$  $g(x) = f(x)$
est une bijection.
    \item Retrouver ce r\'esultat en \'etudiant les variations de $f$.
\end{enumerate}
%}}}
%{{{ Déterminer injectivité
\exercice
On consid\`ere quatre ensembles $A,B,C$ et $D$ et des applications $f:A\rightarrow B$, $g:B\rightarrow
C$, $h:C\rightarrow D$.\\
Montrer que:
$$g\circ f\text{ injective } \Rightarrow f\text{ injective,}$$
$$g\circ f\text{ surjective } \Rightarrow g\text{ surjective.}$$
%}}}
%{{{ Propriété injectivité
\exercice
Soit $f : X \rightarrow Y$. Montrer que
\begin{enumerate}
\item $\forall B \subset Y \, \, f (f^{-1} (B)) = B \cap f (X)$.
\item $f$ est surjective $\iff \forall B \subset Y \, \, f (f^{-1} (B)) = B $.
\item $f$ est injective $\iff \forall A \subset X \, \, f^{-1} (f(A)) = A $.
\item $f$ est bijective $\iff \forall A \subset X \, \,f (\complement A) = \complement f (A).$
\end{enumerate}
%}}}
%{{{ Propriété injectivité
\exercice
Soit $f : X \rightarrow Y$. Montrer que les trois propositions suivantes sont
\'equivalentes:
\begin{enumerate}
\item[i. ] $f$ est injective.
\item[ii. ] $\forall A, B \subset X \, \, f (A \cap B) = f (A) \cap f (B)$.
\item[iii. ]$\forall A, B \subset X \, \, A \cap B = \emptyset \implies
f (A) \cap f (B) = \emptyset$.
\end{enumerate}
%}}}
%{{{ Test injectivité
\exercice
Soit $f : X \rightarrow Y$.On note ${\hat{f}} : {\mathcal{P} (X)} \to {\mathcal{P} (Y)},
{A} \mapsto {f (A)}$ et ${\tilde{f}} : {\mathcal{P} (Y)} \to {\mathcal{P} (X)},
{B} \mapsto {f^{-1} (B)}$.\\
Montrer que:
\begin{enumerate}
\item $f$ est injective $\iff \hat{f}$ est injective.
\item $f$ est surjective $\iff \tilde{f}$ est injective.
\end{enumerate}
%}}}
% }}} Injection et Bijection %
%}}}
% Ensemble finis {{{ %
% }}} Ensemble finis %
% Relation d'équivalence{{{ %
% Relation 1 {{{ %
\exercice
Soit $E=\Nn\times\Nn$, on d\'efinit $\mathcal{R}$ par:\\

$$(a,b)\mathcal{R}(a',b')\Leftrightarrow a+b'=b+a'$$
\begin{enumerate}
  \item Montrer que $\mathcal{R}$ est une relation d'équivalence.
\end{enumerate}
% }}} Relation 1 %

%{{{ Relation 2
\exercice
Dans $\Rr^2$ on d\'efinit la relation $\mathcal{R}$ par :
$$(x,y)\mathcal{R}(x',y')\Leftrightarrow y=y'.$$
\begin{enumerate}
    \item Montrer que $\mathcal{R}$ est une relation d'\'equivalence.
    \item D\'eterminer la classe d'\'equivalence de $(x,y)\in \Rr^2$.
\end{enumerate}
%}}}
% Relation 3 {{{ %
\exercice
\begin{enumerate}
  \item
Montrer que la relation $\mathcal{R}$ d\'{e}finie sur $\Rr$ par :
$$x\mathcal{R} y\Longleftrightarrow xe^{y}=ye^{x}$$
est une relation d'\'{e}quivalence.
\item
Pr\'{e}ciser, pour $x$ fix\'{e} dans $\Rr$, le nombre d'\'{e}l\'{e}ments de
la classe de $x$ modulo $\mathcal{R}$.
\end{enumerate}
% }}} Relation 3 %
% Relation 4 {{{ %
\exercice
\'Etudier les propri\'et\'es des relations suivantes. Dans le cas
d'une relation d'\'equivalence, pr\'eciser les classes ; dans le cas
d'une relation d'ordre, pr\'eciser si elle est totale, si l'ensemble admet
un plus petit ou plus grand \'el\'ement.
\begin{enumerate}
    \item Dans $\mathcal{P}(E)$:
      $$A\mathcal{R}_1 B \iff A\subset B\quad$$

    \item Dans $\mathcal{P}(E)$ $$A\mathcal{R}_2 B \iff A\cap
      B=\emptyset$$
    \item Dans $\Zz$:
$$ a\mathcal{R}_3 b \iff \exists n\in \Nn \ \,a-b=3n$$
\end{enumerate}
% }}} Relation 4 %
% Relation d'ordre {{{ %
\exercice
Soit $ (E, \leq)$ un ensemble ordonn\'e.\\
On d\'efinit sur $\mathcal{P} (E)\setminus\left\{ \emptyset \right\}$ la relation
$\prec$ par 
$$X \prec Y \quad \text{ ssi } \quad (X = Y \ \text{ ou } \ 
\forall x \in X \  \forall y \in Y \  x \leq y).$$ 
\begin{enumerate}
  \item  V\'erifier que c'est une relation d'ordre.
\end{enumerate}
% }}} Relation d'ordre %
%%{{{ Relation et application
%\exercice

%Soit $E$ ordonn{\'e} tel que:
%\begin{center}
%  toute partie non vide et major{\'e}e\footnote{admet une borne sup{\'e}rieure.}
%\end{center}
%\begin{itemize}
%  \item  Montrer que toute partie non vide et minor{\'e}e admet une borne inf{\'e}rieure.
%\end{itemize}
%%}}}
% }}} Relation d'équivalence %
%%{{{Challenge
%\exercice

%$A$ et $B$ sont des parties d'un ensemble $E$.\\

%Montrer que :

%\begin{enumerate}
%\item  $(A\Delta B=A\cap B)\Leftrightarrow(A=B=\varnothing)$.
%\item  $(A\cup B)\cap(B\cup C)\cap(C\cup A)=(A\cap B)\cup(B\cap C)\cup(C\cap A)$.
%\item  $A\Delta B=B\Delta A$.
%\item  $(A\Delta B)\Delta C=A\Delta(B\Delta C)$.
%\item  $A\Delta B=\varnothing\Leftrightarrow A=B$.
%\item  $A\Delta C=B\Delta C\Leftrightarrow A=B$.
%\end{enumerate}

%%}}}
\end{document}

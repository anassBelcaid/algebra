\documentclass[10pt, a4paper, twocolumn]{homework}
\usepackage[inline]{enumitem}

% Link of the exercices
% http://www.bibmath.net/ressources/index.php?action=affiche&quoi=bde/algebre/groupe&type=fexo
\title{Espace Vectoriels}
\date{27 Janvier 2021}
\author{A.Belcaid}



\begin{document}
\maketitle

% Sous espace vectoriels {{{ %
% Vérification {{{ %
\exercice{1}

Pour chaque cas vérifier si $E$ est un espace vectoriel?

\begin{enumerate}
  \item $E = \left\{(x,y,z)\;|\; x+y +3z = 0\right\}$
  \item $E = \left\{(x,y,z)\;|\; x+y +3z = 2\right\}$
  \item $E = \left\{(x,y,z,t)\;|\; x=y=2z=4t\right\}$
  \item $E = \left\{(x,y)\;|\; xy=0\right\}$
  \item $E = \left\{(x,y)\;|\; x=y^2\right\}$
  \item $F = \left\{(x,y,z)\;|\; 2x+3y-5z=0\right\}$ et\\ $G
    =\left\{(x,y,z)\in \Rr^3\;|\; x-y+z=0\right\}.$\\[4pt]
    Vérifier $E = F\cap G$
  \item Même question mais pour $E = F \cup G$

\end{enumerate}
% }}} Vérification %
% Vérification bis {{{ %
\exercice{1}
Détéreminer si les ensembles suivants sont des sous espaces vectoriels?

\begin{enumerate}
  \item $E_1 = \left\{P\in \Rr[X]\;|\; P(0) = P(2)\right\}$
  \item $E_2 = \left\{P\in \Rr[X]\;|\; P^{'}(0) = 2\right\}$
  \item Pour $A\in \Rr[X]$ non-nul fixé, $E_3=\left\{P\in
    \Rr[X]\;|\; A|P\right\}$
  \item $\mathcal{D}$ l'ensemble des fonctions dérivable de $\Rr$ vers
    $\Rr$.
\end{enumerate}
% }}} Vérification bis %
% Union de deux espace {{{ %
\exercice{3}

Soit $E$ un espace vectoriel et $F$ et $G$ deux sous-espace vectoriels de $E$.
Montrer que $F\cup G$ est un sous espace vectoriel si et seulement si
$F\subset G$ ou $G\subset F$.
% }}} Union de deux espace %
% }}} Sous espace vectoriels %
% Combinaison linéaire {{{ %
% Combinaison linéaire 1 {{{ %
\exercice{1}
Pour chaque cas vérifier si $u$ est une combinaison\\ linéaire des
$(u_i)_i$?
\begin{enumerate}
  \item $u=(1,2)$, $u_1=(1,-2)$ et $u_2=(2,3)$
  \item $u=(1,2)$, $u_1=(1,-2)$ et $u_2=(2,3)$, $u_3=(-4,5)$
  \item $u=(2,5,3)$, $u_1=(1,3,2)$ et $u_2=(1,-1,4)$
\end{enumerate}
% }}} Combinaison linéaire 1 %
% les baques {{{ %
\exercice{2}

\begin{itemize}
  \item Emilie achète pour sa maman une baque contenant $2g$ d'or, $5g$ de
    cuivre et $4g$ d'agent. Il paie $6200$.
  \item Paulin achète une bague contenant $3g$ d'or, $5g$ de cuivre et $1g$
    d'argent. Il paie $5300$.
  \item Frédéric achète une bague contenant $5g$ d'or, $12g$ de cuivre et
    $9g$ d'argent. Combien va-t-il payer?
\end{itemize}
% }}} les baques %
% }}} Combinaison linéaire %
% Famille libre {{{ %
% Vérification famille libre {{{ %
\exercice{1}
Les familles suivantes sont-elle libres?

\begin{enumerate}
  \item $u=(1,2,3)$ et $v=(-1,4,6)$
  \item $u=(1,2,-1)$ $v=(1,0,1)$  et $w=(0,0,1)$.
  \item $u=(1,2,-1)$,$v=(1,0,1)$ et $w=(-1,2,-3)$.
  \item $u=(1,2,3,4)$, $v=(5,6,7,8)$, $w=(9,10,11,12)$\\ et $z=(13,14,15,16$.
\end{enumerate}
% }}} Vérification famille libre %
% Vérification prime {{{ %
\exercice{1}

On considère dans $\Rr^3$ les vecteurs $v_1=(1,1,0)$ et $v_2=(4,1,4)$ et
$v_3=(2,-1,4)$.\\
\begin{enumerate}
  \item Montrer que la famille $(v_1,v_2)$ est libre. Faire de même pour
    $(v_1, v_3)$, puis pour $(v_2, v_3)$.
  \item La famille $(v_1, v_2, v_3)$ est-elle libre?
\end{enumerate}
% }}} Vérification prime %
% Polynomes échelonnés {{{ %
\exercice{1}

Soit $\left(P_1,\ldots, P_n\right)$ une famille de polynômes de $\Cc[X]$ non
nuls à degrés \textbf{échelonnés}. C'est à dire 
$$
\text{deg}(P_1) < \text{deg}(P_2) < \ldots <\text{deg}(P_n)
$$
\begin{enumerate}
  \item Montrer que $\left(P_1, \ldots, P_n\right)$ est une famille libre.
\end{enumerate}
% }}} Polynomes échelonnés %
% Comlétion {{{ %
\exercice{2}

On considère les vecteurs suivants:

$$ v_1 = (1,-1,1), \;\; v_2=(2,-2,2),\;\; v_3=(2,-1,2).$$
\begin{enumerate}
  \item Peut-on trouver un vecteur $w$ tel que $(v_1, v_2, w)$ soit libre?
    Si oui, construisez-en un.
  \item Même question en remplaçant $v_2$ par $v_3$.
\end{enumerate}

% }}} Comlétion %
% }}} Famille libre %
% Espace vectoriel engendré {{{ %
% Span 1 {{{ %
\exercice{1}
Donner un système d'équations résumant les espaces vectoriels engendrés par les
vecteurs suivants:

\begin{enumerate}
  \item $u = (1,2,3)$
  \item $u_1 = (1,2,3)$ et $u_2=(-1,0,1)$
  \item $u_1=(1,2,0)$, $u_2=(2,1,0)$ et $u_3=(1,0,1)$.
\end{enumerate}
% }}} Span 1 %
% Générateur {{{ %
\exercice{1}
Trouver un système générateur des sous-espaces vectoriels suivants

\begin{enumerate}
  \item $F=\left\{(x,y,z)\;|\; x + 2y -z=0\right\}$
  \item $G = \left\{(x,y,z)\;|\; x-y+z=0 \text{ et } 2x-y-z=0\right\}$
\end{enumerate}
% }}} Générateur %
% Egalité générateur {{{ %
\exercice{1}

Dans les exemples suivants, démontrer que les sous-espaces $F$ et $G$ sont
égaux.

\begin{enumerate}
  \item $u_1=(1,1,3)$ , $u_2=(1,-1,-1)$,$v_1=(1,0,1)$, \\$v_2=(2,-1,0)$.

    $$
    F=\text{Vect}(u_1,u_2) \text{ et } G=\text{Vect}(v_1,v_2)
    $$

  \item $u_1=(2,3,-1)$ , $u_2=(1,-1,-2)$,$v_1=(3,7,0)$, \\$v_2=(5,0,-7)$.

    $$
    F=\text{Vect}(u_1,u_2) \text{ et } G=\text{Vect}(v_1,v_2)
    $$

  \item $v_1=(1,1,-2)$, $v_2=(1,-4,3)$.
    $$
    F=\left\{(x,y,z)\;|\; x+y+z=0\right\} \text{ et  }
    G=\text{Vect}(u_1,v_1)
    $$
\end{enumerate}
% }}} Egalité générateur %
% Espace Supplémentaires {{{ %
\exercice{2}

On considère dans $\Rr^4$ les cinc vecteurs suivants $v_1=(1,0,0,1)$,
$v_2=(0,0,1,0)$, $v_3=(0,1,0,0)$,\\ $v_4=(0,0,0,1)$ et $v_5=(0,1,0,1)$.\\

Pour chaque cas, vérifier si les sous espaces vectoriels sont
\textbf{supplémentaires}?

\begin{enumerate}
  \item $\text{Vect}(v_1,v_2)$ et $\text{Vect}(v_3)$?
  \item $\text{Vect}(v_1,v_2)$ et $\text{Vect}(v_4,v_5)$?
  \item $\text{Vect}(v_1,v_3,v_4)$ et $\text{Vect}(v_2,v_5)$?
  \item $\text{Vect}(v_1,v_4)$ et $\text{Vect}(v_3,v_5)$?
\end{enumerate}
% }}} Espace Supplémentaires
% Somme {{{ %
\exercice{3}
Soit $E=\Rr^4$. On considère $(u_1, u_2, u_3, u_4)$ une famille libre et on
pose:

$$
\small
F=\text{Vect}(u_1 + u_2, u3),\; G=\text{Vect}(u_1+u_3,
u_4),\;H=\text{Vect}(u1+u4, u_2)
$$

\begin{enumerate}
  \item Démonter que $F\cap G = F\cap H = G \cap H = \{0\}$
  \item La somme $F+G+H$ est-elle directe?
\end{enumerate}
% }}} Somme %
% }}} Espace vectoriel engendré %
% Application linéaires {{{ %
% Vérification {{{ %
\exercice{1}

Pour chaque cas, vérifier s'il s'agit d'une application linéaire?
\begin{enumerate}
  \item $f:\Rr^2\rightarrow \Rr^3, (x,y)\rightarrow (x+y,\; x-2y,\; 0)$
  \item $f:\Rr^2\rightarrow \Rr^3, (x,y)\rightarrow (x+y,\; x-2y,\; 1)$
  \item $f:\Rr^2\rightarrow \Rr, (x,y)\rightarrow x^2 - y^2$
  \item $f:\Rr[x]\rightarrow \Rr^2, P\rightarrow (P(0), P^{'}(1))$
\end{enumerate}
% }}} Vérification %
% Noyau et image {{{ %
\exercice{1}

Soit $f:\Rr^2\rightarrow \Rr^3$ l'application définie par:
\begin{equation*}
  f(x,y)=(x+y,\; x-y,\; x+y)
\end{equation*}

\begin{enumerate}
  \item Détéreminer le noyau de $f$.
  \item Calculer son image.
  \item $f$ est-elle injective? surjective?
\end{enumerate}
% }}} Noyau et image %
% Projections {{{ %
\exercice{2}

Soit $E$ un espce vectoriel et $p,\;q$ deux projecteurs de $E$ tel que $p\neq
0$, $q\neq 0$ et $p\neq q$.

\begin{enumerate}
  \item Démonter que $(p,q)$ est une famille libre dans l'espace
    $\mathcal{L}(E)$ des fonctions linéaires entre $E$ et $E$.
\end{enumerate}

\textbf{Indice}: Si $q$ est une projection, quelle sera $q^2$.
% }}} Projections %
% }}} Application linéaires %

\end{document}

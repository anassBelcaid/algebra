\documentclass[10pt, a4paper, twocolumn]{homework}
\usepackage[inline]{enumitem}

% Link of the exercices
% http://www.bibmath.net/ressources/index.php?action=affiche&quoi=bde/algebre/groupe&type=fexo
\title{Polynômes}
\date{08 novembre 2020}
\author{A.Belcaid}



\begin{document}
\maketitle


% Division Euclidienne {{{ %
% Calcul {{{ %
\exercice{1}
Pour les trois cas listés, calculer la division euclidienne de $P$ par
$Q$.

\begin{enumerate}
  \item $P=X^4 + 5X^3 + 12X^2 + 19X - 7$ et $Q=X^2 + 3X-1$.
  \item $P=X^4-4X^3-9X^2+27X+38$ et $Q=X^2-X-7$.
  \item $P=X^5-X^2+2$ et $Q = X^2 +1$.
\end{enumerate}

% }}} Calcul %
% Division produit {{{ %
\exercice{2}

Soit $P\in \Rr[X]$, $a,b\in \Rr$, $a\neq b$. Sachant que le reste de la
division euclidienne de $P$ par $(X-a)$ vaut $1$ et que celui de $P$ par
$(X-b)$ vaut $-1$.

\begin{enumerate}
  \item Évaluer l'image $P(a)$ et $P(b)$?
  \item On note $R\in \Rr[X]$ le reste de la division euclidienne de $P$ par
    $(X-a)(X-b)$. Quel  sera le degré de $R$?
  \item En déduire l'expression de $R$.
\end{enumerate}

% }}} Division produit %
% Détérmination d'un ensemble polynôme {{{ %
\exercice{2}
On se propose de déterminer l'ensemble

\begin{equation*}
  E = \left\{P\in \Rr[X]\;\; P(X^2)=(X^3+1)P(X)\right\} 
\end{equation*}

\begin{enumerate}
  \item Démontrer que le polynôme nul ainsi\\  que le polynôme $X^3-1$ sont
    dans $E$.\\[4pt]
  \item Soit $P\in E$, non nul.
    \begin{enumerate}
      \small
    \item Démontrer que $P(1)=0$ puis que $P^{'}(0)=P^{''}(0)=0$.
    \item En effectuant la division euclidienne de $P$ par
      $X^3-1$, démontrer qu'il existe $\lambda\in \Rr$ tel que
      $$
      P(X) = \lambda (X^3-1)
      $$
    \end{enumerate}
  \item En déduire l'ensemble $E$.
\end{enumerate}
% }}} Détérmination d'un ensemble polynôme %
% }}} Division Euclidienne %
% Calcul PGCD {{{ %
% Exercice simple de calcul PGCD {{{ %
\exercice{1}

Pour chaque cas, déterminer le \textbf{PGCD} entre $P$ et $Q$.

\begin{enumerate}
  \item $P=X^4-3X^3 + X^2 + 4$ et $Q=X^3-3X^2+3X-2$.
  \item $P=X^5-X^4+2X^3-2X^2+2X-1$ et $Q=X^5-X^4+2X^2-2X+1$.
  \item $P=X^n -1$ et $Q=(X-1)^n$.
\end{enumerate}
% }}} Exercice simple de calcul PGCD %
% Equation de Bézout {{{ %
\exercice{1}
Trouver deux polynômes $U$ et $V$ de $\Rr[X]$ tel que 
$$
AU + BV = 1
$$

où $A = X^7-X-1$ et $B=X^5-1$.
% }}} Equation de Bézout %
% Facteur commun {{{ %
\exercice{1}
Soient $P$ et $Q$ des polynômes de $\Cc[X]$ non constants. Montrer  que
l'équivalence entre:

\begin{enumerate}
  \item $P$ et $Q$ ont un facteur commun.
  \item il existe $A,B\in \Cc[X]$, $A\neq0$, $B\neq 0$, tel que
    $$
    AP = BQ
    $$
  et $\text{deg}(A)< \text{deg}(Q),\quad \text{deg}(B)<\text{deg}(P)$ 
\end{enumerate}
% }}} Facteur commun %  
% }}} Calcul PGCD %
% Racines {{{ %
% Ordre de multiplicité {{{ %
\exercice{1}
Quel est pour $n\geq 1$ l'ordre de multiplicité de $2$ du polynôme:

$$
P_n(X)= n X^{n+2}-(4n+1)X^{n+1}+4(n+1)X^n-4X^{n-1}
$$
% }}} Ordre de multiplicité %
% Racine sur Q {{{ %
\exercice{2}
Soit $P(X)=a_nX^n+\ldots+a_0$ un polynôme dans $\Zz[X]$. On suppose aussi
que $P$ admet une racine rationnelle $r=\frac{p}{q}$ tel que $p\wedge q
=1$.
\begin{enumerate}
  \item Développer que la forme $P(r)=0$.
  \item Démontrer que $p\;|\; a_0$.
  \item Prouver que $q\;|\; a_n$
  \item En déduire que $P=X^5 - X^2+1$ n'admet pas de racines dans
    $\Qq$.
\end{enumerate}
% }}} Racine sur Q %
% }}} Racines %
% Réductibilité {{{ %
% Définition simple réductible {{{ %
\exercice{1}

\begin{enumerate}
  \item Le polynôme $P(X) = X^4 + X^2 + 1$ est il irréductible  dans $\Rr[X]$?
dans $\Cc[X]$?
\item La relation $P\mathcal{R} Q \iff P\; \text{divise}\; Q$  est-elle une
    relation d'ordre?
\end{enumerate}
% }}} Définition simple réductible %
% Décomposition simple {{{ %
\exercice{1}
Pour chaque polynôme, donner la décomposition en facteurs irréductibles dans
$\Rr[X]$

\begin{enumerate}
  \item $P_1(X) = X^4 + 1$
  \item $P_2(X) = X^8 - 1$
  \item $P_3(X) = \left(X^2 - X+1\right)^2+1$
\end{enumerate}
% }}} Décomposition simple %
% Décomposition simple {{{ %
\exercice{2}
Soit $P$ le polynôme définit par:

$$
P(X) = 2X^4 + X^2 -3
$$
\begin{enumerate}
  \item 
Décomposer $P$ en facteurs irréductibles dans $\Rr[X]$.
\end{enumerate}
% }}} Décomposition simple %
% Décomposition compliquée {{{1 %
\exercice{3}
Soit le polynôme $P(X) = X^4-6X^3+9X^2+9$.

\begin{enumerate}
  \item Décomposer $X^4-6X^3+9X^2$ en produit de facteurs irréductibles dans
    $\Rr[X]$.
  \item En déduire une décomposition de $P$ dans $\Rr[X]$.

  \item Même question pour $\Cc[X]$.
\end{enumerate}
% }}} Réductibilité %
% Décomposition  {{{ %
\exercice{3}
On considère les deux polynômes suivants:

\begin{itemize}
  \item $P(X) = X^3 - 9X^2 + 26X -24$
  \item $Q(X) = X^3 - 7X^2 + 7X + 15$.
\end{itemize}
\begin{enumerate}
  \item Sachant que $P$ et $Q$ admettent une racine \textbf{commune} $a$,
    Quelle est la relation entre $(X-a)$ et $\text{pgcd}(P,Q)$?
  \item En appliquant l'algorithme d'Euclide, montrer  que le \textbf{pgcd} de $P$
    et $Q$ est $X-3$?
  

  \item Calculer le polynôme $P_1$ tel que
    $$ P = (X-3)P_1$$

  \item Même question pour $Q_1$ tel que:

    $$Q = (X-3)Q_1$$.

  \item En déduire une décomposition en facteurs\\ irréductibles dans
    $\Rr[X]$ de $P$ et $Q$.
\end{enumerate}
% }}} Décomposition  %
\end{document}

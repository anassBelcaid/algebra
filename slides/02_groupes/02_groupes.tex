\documentclass{beamer}
\usetheme{ensam}
\usepackage{pgfplots}
\usepackage{subcaption}
\usepackage{amssymb}
\usepackage{acronym}
\usepackage{mynotations}
\usepackage{tikz}
\usepackage{pgf-pie}
\usetikzlibrary{calc}
\usepackage{amsmath}
\usepackage {algorithmic}
\usepackage{algorithm}
\usepackage{eqparbox}
\usepackage[font=scriptsize]{caption}
\usetikzlibrary{bayesnet,positioning,calc}
\tikzstyle{obs} = [latent,fill=lightBlue]
\tikzstyle{default}=[draw=sexyRed,thick,rounded corners,text width=0.5in,font=\scriptsize,align=center]
\usepgfplotslibrary{colorbrewer}
\definecolor{ForestGreen}{RGB}{34,139,34}
\newcommand{\comment}[1]{\textcolor{ForestGreen}{#1}}
%algorithmic comment
\renewcommand\algorithmiccomment[1]{%
  \hfill\comment{\#\scriptsize\eqparbox{COMMENT}{#1}}%
}
\renewcommand{\algorithmicrequire}{\textbf{Input:}}
\renewcommand{\algorithmicensure}{\textbf{Output:}}
\title{Groupes}
\author{\underline{A.Belcaid}}
\institute{\small Université Euro Méditerranéenne de Fès} 


% Customzation {{{ %
% }}} Customzation %

%tikz bayesian theme
\usetikzlibrary{bayesnet,positioning,calc}
\tikzstyle{obs} = [latent,fill=lightBlue]
\tikzstyle{default}=[draw=sexyRed,thick,rounded corners,text width=0.5in,font=\scriptsize,align=center]
\DeclareMathOperator{\argmin}{argmin}

\pgfplotsset{every tick label/.append style={font=\tiny}}





% add bibliography
\date{\today}

\begin{document}
\maketitle


% Définition {{{ %
% Définition {{{ %
\begin{frame}[<+->]
  \frametitle{Définition}
 \begin{block}{Groupe}
   \small
   Un \textbf{\alert{groupe}}  $\left(G, *\right)$ est un ensemble $G$ muni
   d'une \textbf{opération} $\mathbf{*}$( dite \structure{\textbf{loi de
   composition}}) vérifiant les propriétés suivantes:
   \begin{enumerate}
     \item \alert{\textbf{Loi Interne}}:  $\forall x,y \in G,\quad x*y\in G$.\\[8pt]
     \item \alert{\textbf{Associativité}}: $\forall x, y,z \in G \quad (x*y)*z =
       x*(y*z)$.\\[8pt]
     \item \alert{\textbf{Elément neutre}}: $\exists e\in G\;\text{tel que}
       \forall x \in G\quad x*e = e*x = x$\\[8pt]

     \item \alert{\textbf{Elément inverse}}: $
       \forall x \in G\;\; \exists x^{'}\in G\;\; \text{tel que} x*x^{'} = x^{'}*x = e$\\[8pt]
   \end{enumerate}
 \end{block} 
 Si on plus, l'opération $*$ vérifie:
 \begin{equation}
   \forall x, y\in G\quad x*y = y*x
 \end{equation}
 On dit que le groupe $G$ est \textbf{\alert{abélien}} (\textbf{commutatif}).
\end{frame}


\begin{frame}[<+->]
  \frametitle{Unicité}
 \begin{block}{Unicité Elément neutre}
   L'élément neutre est \textbf{\alert{unique}}.\\[8pt]
   \scriptsize
   Supposons qu'on possède deux éléments uniques $e_1$ et $e_2$ dans $(G, *)$.

   \begin{equation}
     \left\{\begin{array}{lll}
         e_1\;*\; e_2 &=& e_1\\[4pt]
         e_1\;*\; e_2 &=& e_2\\
       \end{array}
     \right.
   \end{equation}
 \end{block} 

 \begin{block}{Unicité Inverse}
   De même, on prouve que l'\textbf{\structure{inverse}} d'un élément $x$ est
   \textbf{\alert{unique}}.
   \scriptsize

   Supposons que pour un $x\in G$, on possède deux inverses $x_1$ et $x_2$.
   alors on peut évaluer l'expression:

   \begin{equation}
     x_1\;*x\;*x_2 = \left\{ \begin{array}{lllll}
         \left(x_1\;*x\right)\;* x_2 & = & e \;* x_2 & =& x_2\\[4pt]
         x_1\;*\left(x\;* x_2\right) & = & x_1 \;* e & =& x_1\\
       \end{array}
     \right.
   \end{equation}
 \end{block}
\end{frame}
% }}} Définition %
% Exemples {{{ %

% Exemples simples {{{ %
\begin{frame}[t]
  \frametitle{Exemples classiques}
 \begin{block}{Mini Exercices}
   \begin{enumerate}
     \item 
   Vérifier que les ensembles muni des opérations suivantes sont des
   \textbf{groupes}:

   \begin{enumerate}
     \item $\left(\mathbb{Q}, \times\right)$\\[6pt]
     \item $\left(\mathbb{Z}, +\right)$ \\[6pt]
   \end{enumerate}
  Justifier pourquoi les ensembles suivants ne sont pas des groupes:
  \begin{enumerate}
    \item $\left(\mathbb{Z}, \times\right)$\\[6pt]
    \item $\left(\mathbb{N}, +\right)$
  \end{enumerate}
   \end{enumerate}


 \end{block} 
\end{frame}
% }}} Exemples simples %


% }}} Exemples %
% Puissances {{{ %
\begin{frame}[<+->]
  \frametitle{Puissance}
  \begin{block}{Puissance}
    Pour un groupe $\left(G,*\right)$ et un élément $x\in G$, on peut définir
    l'opération $ x\;*\; x$ par $\mathbf{\alert{x^2}}$. Plus généralement:
    \begin{equation}
      x^n  = \left\{
        \begin{array}{lll}
          e & \text{si}& n=0\\[6pt]
          \underbrace{x*x\ldots*x}_{n \text{ fois}}& \text{pour}& n>0
        \end{array}
      \right.
    \end{equation}
    On possède alors les propriétés suivantes:
    \begin{enumerate}
      \item $ x^m\;*\; x^n = x^{n+m}$\\[4pt]
      \item $ \left(x^m\right)^n = x^{nm}$\\[4pt]
      \item $\left(x\;*\;y\right)^{-1} = \mathbf{\alert{y^{-1}\;*\;
        x^{-1}}}$\\[4pt]
      \item \textbf{\alert{Si G est abélien}}, $(x*y)^n = x^n\;*\; y^n$
    \end{enumerate}
  \end{block}
\end{frame}


\begin{frame}[t]
  \frametitle{Mini exercice}
 \begin{block}{Mini Exercice}
   \small
   \begin{enumerate}
     \item Soit $f_{a,b}: \mathbb{R}\rightarrow \mathbb{R}$ l'application
       définie par: $x\rightarrow ax+b$.\\
       Montrer que le groupe:
       \begin{equation}
         \Big(\mathcal{F}=\{f_{a,b}\;|\; a\in\mathbb{R}^{*}, b\in\mathbb{R}\},
         \circ\Big)
       \end{equation}
       est un groupe non commutatif.
      \item Soit $G = ]-1, 1[$, Pour $x,y \in G$ on définit:
        \begin{equation}
          x*y = \dfrac{x+y}{1 + xy}
        \end{equation}
        Montrer que $\left(G, *\right)$ forme un groupe.
   \end{enumerate}
 \end{block} 
\end{frame}
% }}} Puissances %
% Exemple des matrices {{{ %
% }}} Exemple des matrices %
% }}} Définition %
% Sous-groupes {{{ %
% Définition {{{ %

\begin{frame}[t]
  \frametitle{2- Sous Groupes}
  \begin{itemize}
    \scriptsize
    \item Souvent on travaille avec des ensembles $G$ qui sont des parties des
      groupes classiques comme $\left(\mathbb{R}, \times\right)$. Il existe
      alors une méthode \alert{\textbf{plus simple}} pour démontrer que $G$ est un groupe.
  \end{itemize}
\pause
  \begin{block}{Sous groupe}
    \small
    Soit $\left(G,\; *\right)$ un groupe. Une \textbf{partie}
    $\mathbf{H}\subset G$ est un \textbf{\alert{sous groupe}} de $G$
    si:

    \centering
    \begin{itemize}
      \item $\mathbf{e} \in H$\\[8pt]
      \item $\forall x, y \in H,\quad x\;*\;y \in H$\\[8pt]
      \item $\forall x\in H,\quad x^{-1}\in H$.
    \end{itemize}

  \end{block}
  \pause
  \begin{block}{Remarque}
   Une méthode plus simple pour prouver que $H$ est un \textbf{sous groupe} est:
   \begin{itemize}
     \item $H \neq \emptyset$
     \item $\forall x, y \in H\quad x\;*y^{-1} \in H$.
   \end{itemize}
  \end{block}
\end{frame}
% }}} Définition %
% Exemples {{{ %
\begin{frame}[<+->]
  \frametitle{2-2 Exemples}
 \begin{block}{Exemples}
   \begin{itemize}
     \item $\mathbf{\left(\mathbb{R}^{*}_+, \times\right)}$ est un \textbf{sous groupe}  de
       $\left(\mathbb{R}^{*}, \times\right)$
       \begin{itemize}
         \scriptsize
         \item $1 \in \mathbb{R}_+^{*}$\\[4pt]
         \item $\forall x,y \in \mathbb{R}^{*}_+\quad x\times y \in
           \mathbb{R}^{*}_+.$
         \item $\forall x \in \mathbb{R}_+^{*}\quad \dfrac{1}{x} \in
           \mathbb{R}^{*}_+$ \\[8pt]
       \end{itemize}
     \item $\mathbf{\left(\mathbb{Z}, +\right)}$ est un sous groupe de
       $\left(\mathbb{R}, +\right)$.
     \item Soit l'ensemble $\mathbb{U} = \left\{z\in\mathbb{C}\;|\; \vert z\vert
       = 1\right\}$\\[8pt]
       \begin{itemize}
         \item Sachant que $\left(\mathbb{C},\times \right)$ est un groupe,
           prouvez que $\alert{\mathbb{U}}$ muni de $\times$ est aussi un
           groupe.
       \end{itemize}
   \end{itemize} 
 \end{block} 

   \begin{figure}[htpb]
   \begin{center}
   \begin{tikzpicture}[scale=0.4,>=stealth]
     \draw[->, thick] (-3, 0)--(3,0);
     \draw[->, thick] (0, -3)--(0,3);
     \draw[sexyRed, thick] (0,0) circle[radius=1cm];
     \node at (1.2, 0.4) {\scriptsize $\mathbb{U}$};
   \end{tikzpicture}
   \end{center}
   \end{figure}
\end{frame}
% }}} Exemples %
% Sous groupes de Z {{{ %

\begin{frame}[t]
  \frametitle{Sous groupes de $\mathbb{Z}$}
 \begin{block}{Théorème}
   Les \textbf{seuls} \alert{sous groupes} de $\mathbb{Z}$, sont les
   $\mathbf{n\mathbb{Z}}$ pour $n\in\mathbb{N}$.
   \begin{equation}
     n\mathbb{Z} = \left\{k.n \;|\; k\in \mathbb{Z}\right\}
   \end{equation}
 \end{block} 
 \begin{itemize}
   \item Lister les éléments de $3\mathbb{Z}$.\\[8pt]
   \item Donner une démonstration de ce théorème. 
 \end{itemize}
\end{frame}
% }}} Sous groupes de Z %
% Mini exercices {{{ %
\begin{frame}[t]
  \frametitle{Mini exercices}
 \begin{block}{Mini Exercices}
   \begin{enumerate}
     \item Montrer que $\left\{2^n\;|\; n\in \mathbb{Z}\right\}$ est un sous
       groupe de $\left(\mathbb{R}^{*}, \times\right)$\\[8pt]
     \item Montrer que si $H$ et $H^{'}$ sont deux sous groupes de $\left(G,
       *\right)$, alors $\alert{H\cap H^{'}}$ est aussi un sous groupe.\\[8pt]
     \item Montrer que $\alert{5\mathbb{Z}\cup 8\mathbb{Z}}$ n'est pas un sous
       groupe. 
   \end{enumerate}
 \end{block} 
\end{frame}
% }}} Mini exercices %
% }}} Sous-groupes %
% Morphisme de groupes {{{ %
% Définition {{{ %
\begin{frame}[t]
  \frametitle{2.3 Morphisme de groupes}
\begin{block}{Morphisme}
  \small
  Soit $\Group{G}{*}$ et $\Group{G^{'}}{\diamond}$ deux \textbf{groupes}. Une
  application $f\;:\; G\Longrightarrow G^{'}$ est un \textbf{\alert{morphisme de
  groupes}} si:
  \begin{equation}
    \forall x, y \in G \quad   f(x\; *\;y) = f(x)\diamond f(y)
  \end{equation}
\end{block}  
\pause

Les deux exemples classiques que vous connaissez sont:

\begin{itemize}
  \small
  \item $G = (\Rr, +)$ et $ G^{'}=\left(\Rr^{*}_+, \times\right)$. Le morphisme
    classique qui transforme l'\textbf{addition} en \textbf{multiplication}
    est la fonction \textbf{\alert{exponentielle}} \pause
    \begin{equation}
      \forall x, y \in \Rr\quad \exp(x \structure{+} y)= \exp(x)\alert{\times} \exp(y)
    \end{equation}
  \item Inversement, la fonction \textbf{\alert{logarithme}}  est un morphisme
    de groupe entre $G^{'}$ et $G$.

    \begin{equation}
      \log(x\structure{\times} y) =  \log(x) \alert{+} \log(y)
    \end{equation}
\end{itemize}
\end{frame}
% }}} Définition %
% Propriétées {{{ %
%{{{Inverse et élement neutre
\begin{frame}[t]
  \frametitle{Propriétés(1)}
 \begin{block}{Proposition}
   \small
   Soit $f\;:\; G\Longrightarrow G^{'}$ un morphisme de groupe. On note $e_{G}$
   ($e_{G^{'}}$) l'élément neutre de $G$($G^{'}$). Alors:

   \begin{itemize}
     \item \begin{equation}
         f(e_{G}) = e_{G^{'}}
     \end{equation}
   \item 
     \begin{equation}
       \forall x \in G\quad  f(x^{-1}) = \left(f\left(x\right)\right)^{-1}
     \end{equation}
   \end{itemize}
 \end{block} 

 \begin{itemize}
   \item Pour l'exemple de logarithme, on sait déjà que:
     \begin{equation}
       \log(\structure{1}) = \alert{0}
     \end{equation}
   \item Donner une \alert{\textbf{preuve}} des deux équations.
 \end{itemize}
\end{frame}
%}}}
% Composition et inverse {{{ %
\begin{frame}[t]
  \frametitle{Propriétés (2) }
  \begin{itemize}
    \small
  \item Soient deux morphismes $f:\;G\Longrightarrow G^{'}$ et $g:\;G^{'}\Longrightarrow G^{''}$
    \begin{block}{Composée}
         La composée \alert{$\mathbf{g\circ f}$ est un morphisme} de groupe entre $G$ et
   $G^{''}$.\\[8pt]
    \end{block}
    \pause
    \item 
      \begin{block}{Inverse}
   Si l'application est \textbf{\alert{bijective}}. Alors $\mathbf{f^{-1}}$ est aussi un morphisme de
   groupe entre $G^{'}$ et $G$.\\[8pt]
    \end{block}

   Dans ce cas, on dit que:
   \begin{itemize}
     \item $f$ est un \textbf{\alert{isomorphisme}}.\\[8pt]
     \item Les deux groupes $G$ et $G^{'}$ sont \textbf{\alert{isomorphes}}
   \end{itemize}
 \end{itemize}
\end{frame}
% }}} Composition et inverse %

% }}} Propriétées %
% Noyaux et image {{{ %
% Noyau {{{ %
\begin{frame}[t]
  \frametitle{Noyau et Image}
  \begin{itemize}
    \small
    \item Si $f: G\longrightarrow G^{'}$ est un morphisme de groupe. Alors on
      identifie deux \textbf{sous groupes} importants:

      \begin{block}{Noyau}
        \small
        Le \textbf{\alert{noyaux}} (\emph{kernel}) de $f$ est
        \begin{equation}
          \text{\textbf{Kern}}\;f = \left\{ x\in G\;|\; f(x) = e_{G^{'}}\right\}
          = f^{-1}\left(\left\{ e_{G^{'}}\right\}\right)
        \end{equation}
      \end{block}
      \pause
      \vspace*{1cm}
      \begin{block}{Image}
        \small L'\alert{\textbf{image}} de $f$ est:
        \begin{equation}
          \text{\textbf{Img}}\;f = \left\{ f(x)\;|\; x\in G\right\}\subset G^{'}
        \end{equation}
      \end{block}
  \end{itemize}
\end{frame}

\begin{frame}[t]
  \frametitle{Propriétés}
 Voici quelque propriétés du noyau et l'image de $f$:.

   \small
   \begin{block}{Propriétés}
 \begin{enumerate}
 \item $\text{\textbf{Kern}}\;f$ est un \textbf{sous groupe}  de $G$.\\[8pt]
 \item $\text{\textbf{Img}}\; f$ est un \textbf{sous groupe}  de $G^{'}$.\\[8pt]
 \item $f$ est \alert{\textbf{injectif}} si et seulement si $\text{\textbf{Kern}}\;f =
   \left\{e_G\right\}$\\[8pt]

 \item $f$ est \alert{\textbf{surjectif}} si et seulement si $\text{\textbf{Img}}\;f
   =G^{'}$\\[8pt]
 \end{enumerate}
 \end{block}
 \begin{itemize}
   \item Preuve:
 \end{itemize}
\end{frame}
% }}} Noyau %

\begin{frame}[t]
  \frametitle{Mini Exercices}
 \begin{block}{Mini Exercices}
   \small
   \begin{enumerate}
     \small
     \item Soit $f\;:\; \left(\Zz, +\right)\longrightarrow \left(\Qq^{*},
       \times\right)$ définie par $f(n)= 2^n$.
       \begin{itemize}
         \item Montrer que $f$ est un morphisme de groupe.\\[4pt]
         \item $f$ est elle \textbf{injective}, \textbf{surjective}?
       \end{itemize}
       \small
     \item Soit $\left(G, *\right)$ un groupe et $f\;:\; G\longrightarrow G$
       l'application définie par $f(x) = x^2$.
       \begin{itemize}
         \small
         \item Montrer que si $\left(G,*\right)$ est \textbf{commutatif}, alors
           $f$ est un morphisme de groupe.\\[4pt]
         \item Montrer la réciproque.
       \end{itemize}
       \small
     \item Montrer qu'il n'existe pas de morphisme $f:(\Zz, +)\longrightarrow
       (\Zz, +)$ tel que:
       \begin{equation*}
         f(2) = 3
       \end{equation*}
   \end{enumerate}
 \end{block} 
\end{frame}
% }}} Noyaux et image %
% }}} Morphisme de groupes %
% Groupe Z / nZ {{{ %
% Ensemble Z/nZ {{{ %
\begin{frame}[t]
  \frametitle{Groupe $\Zz/ n\Zz$ }
  \begin{itemize}
    \small
    \item Pour un $n\geq 1$, on rappelle que:
      \begin{equation*}
        \Zz/ n\Zz = \left\{ \overline{0}, \overline{1}, \overline{2}, \ldots, \overline{n-1}\right\}
      \end{equation*}
    où $\overline{p}$ désigne la classe \alert{\textbf{d'équivalence}} de $p$
    modulo $n$.
  \item Sur cet ensemble, on peut définir l'opérateur d'\alert{\textbf{addition}} entre les
    classes par:

    \begin{equation}
      \overline{p} + \overline{q} = \overline{p + q}
    \end{equation}

    On prouve que:

    \begin{block}{Groupe}
      $\left(\Zz/n\Zz, +\right)$ est un groupe commutatif.
    \end{block}
  \end{itemize} 
\end{frame}
% }}} Ensemble Z/nZ %
% Groupe cyclique de cardinal fini {{{ %

\begin{frame}[t]
  \frametitle{Groupe cyclique de cardinal fini}
 \begin{block}{Définition}
   \small
   Un groupe $\left(G, *\right)$ est dit \textbf{\alert{cyclique}} si

   \begin{equation*}
     \exists! a \in G\quad \forall x \in G, \exists k\in \Zz \text{ tel que } x =
     a^n
   \end{equation*}
ou que le groupe $G$ est engendré par l'élément $a$.
 \end{block} 

 \begin{block}{Exemple}
   \small
   Le groupe $\left(\Zz/n\Zz, +\right)$ est un groupe \textbf{cyclique} engendré
   par l'élément $a = \overline{1}$.\\

   On peut prouver que $\forall k \in \Zz/n\Zz$ on a:
   \begin{equation*}
     k = \underComment{\overline{1} + \overline{1} + \ldots, \overline{1}}{$k$ fois}
   \end{equation*}
 \end{block}
 \begin{block}{Théorème}
   \small
   Soit $\left(G,*\right)$ un groupe de cardinal fini $n$. Alors $G$ est
  \textbf{\alert{isomorphe}}  à $\Zz/n\Zz$.
 \end{block}
\end{frame}
% }}} Groupe cyclique de cardinal fini %
% }}} Groupe Z / nZ %


\end{document}
